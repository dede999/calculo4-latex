\documentclass[12pt,openany, letterpaper]{book}
\usepackage[utf8]{inputenc}
\usepackage[portuguese]{babel}
\usepackage[margin = 2cm]{geometry}
\usepackage[colorlinks=false]{hyperref} %conectar o arquivo em PDF atravéz de links
\usepackage{framed} % frame enviroment
\newtheorem{theorem}{Teorema}[section]
\newtheorem{lemma}{Lema}[section]
\newtheorem{corollary}{Corolário}[theorem]
\newtheorem{definition}{Definição}[section]
\newtheorem{demonstration}{Demonstração}
\newtheorem{obs}{Observação}
\newtheorem{prop}{Proposição}
\usepackage[rightcaption]{sidecap}
\usepackage{graphicx} %acrescentar imagens
\usepackage{amsmath} %ambiente matemático
\usepackage{dsfont} % coloquei esse pacote aqui, mas não lembro ao certo, no que ele ajuda
\usepackage{multicol} %permite que eu escreva o texto em mais de uma coluna
\newcommand{\LI}[1][n]{\lim_{{#1} \rightarrow \infty}}
\newcommand{\soma}[2][n]{\sum_{{#1} = #2}^\infty}
\newcommand{\defini}[3][\mathds{R}]{#2: #3 \rightarrow {#1}}
\newcommand{\E}[1]{Exercício #1}
\newcommand{\DS}{\displaystyle{}}
\newcommand{\Resolve}{\begin{center} \textbf{Resolução} \end{center}}
\newcommand{\IT}[1]{Item {#1}}
\newcommand{\diff}[2]{\frac{d{#1}}{d{#2}}}
\newcommand{\difff}[2]{\frac{\partial {#1}}{\partial {#2}}}
\newcommand{\cmw}{Critério M de Weierstrass }
\newcommand{\IP}{\int_{-\pi}^{\pi}}

%modelo de início de capítulo encontrado no link https://pt.sharelatex.com/learn/Defining_your_own_commands
\makeatletter 
\def\thickhrulefill{\leavevmode \leaders \hrule height 1.2ex \hfill \kern \z@}
\def\@makechapterhead#1{
  \vspace*{10\p@}%
  {\parindent \z@ \centering \reset@font
        \thickhrulefill\quad 
        \scshape\bfseries\textit{\@chapapp{}  \thechapter}  
        \quad \thickhrulefill
        \par\nobreak
        \vspace*{10\p@}%
        \interlinepenalty\@M
        \hrule
        \vspace*{10\p@}%
        \Huge \bfseries #1 \par\nobreak
        \par
        \vspace*{10\p@}%
        \hrule
        \vskip 40\p@
  }}


\title{Calculo IV - IME 2015}
\author{André Luiz Abdalla Silveira \thanks{graças à ajuda e as lousas do Professor Antônio C. Asperti}}
\date{Segundo Semestre -- 2015}

\begin{document}

\maketitle

\tableofcontents


\chapter*{Introdução}
\label{chap:c0}

\hspace{5mm} Eu criei esse livrinho pra ter um resumo e material de estudo. Espero que dê os resultados esperados para mim e vc também que estiver lendo esse material. 

Esse PDF não tem intenção nenhuma de ser uma segunda lousa, quanto menos substituir as aulas, pois selecionei partes que são importantes para mim, mas podem não ser para outros. Essa é unicamente uma forma de me organizar e estudar, e que PODE ajudar a outras pessoas em outros momentos.\\

Professor: 	Antonio Carlos Asperti (\href{mailto:asperti@ime.usp.br}{asperti@ime.usp.br})

Livro usado: Um curso de Cálculo -- Volumes 2 e 4
\vspace{10mm}

Provas:
\begin{tabular}{|c|c|c|c|c|}
\hline
 Prova & Data & Dia & Peso & Correção \\ \hline 
 1 & 14/09 & 2a & 1 &  p. \pageref{C:P1}\\ \hline
 2 & 21/10 & 4a & 1 &  p. \pageref{C:P2}\\ \hline
 3 & 02/12 & 4a & 2 &  p. \pageref{C:P3}\\ \hline
 SUB & 07/12 & 2a & ?? & p. \pageref{C:SUB} \\\hline
\end{tabular}

\begin{figure}
\centering
\includegraphics[width=1.05\textwidth]{lucky}
\caption{Boa Sorte, pessoal}
\label{fig:lucky}
\end{figure}


\part{Teoria}
\chapter{Integrais Impróprias}
\label{chap:c1}
CONTEUDO5Figura
\hspace{5mm} Assunto tratado entre 03/08/2015 e 05/08/2015
\section{Conceitos}
\label{sec:s11}

\hspace{5mm} São expressões do tipo $\displaystyle{\int_a^{+ \infty}f(x)dx}$, $\displaystyle{\int_{-\infty}^{a} f(x)dx}$, $\displaystyle{\int_{-\infty}^{+ \infty}f(x)dx}$\\
\vspace{15pt}

Definição: Seja $f:{[a,+\infty)}\rightarrow \mathds{R}$ e suponha que $f$ seja integrável em $[a,t]$ para todo $t>a$. Assim definimos $$ \int_a^{+\infty}f(x)dx = \lim_{t\rightarrow\infty}\int_a^t f(x)dx$$ desde que exista o limite e seja finito.

Neste caso, o o limite é a \underline{integral imprópria} de $f$ estendido ao intervalo $[a,+\infty)$.

Se o limite existir e for finito, diremos se tratar de uma \underline{integral convergente}. Caso contrário, é uma \underline{integral divergente}.

Exemplo 1:
\begin{align*}
\int_1^{+\infty}\frac{1}{x^2}dx &= \lim_{t\rightarrow + \infty}\int_1^t \frac{dx}{x^2}=\LI[t]\left[\frac{x^{-2+1}}{-2+1}\right]_1^t = \lim_{t\rightarrow + \infty}\left[-\frac{1}{x}\right]_1^t = -\lim_{t\rightarrow + \infty}\left[\frac{1}{t}-1\right]=1
\end{align*}

Exemplo 2:
\begin{align*}
\int_1^{+\infty}\frac{dx}{x}= \lim_{t\rightarrow + \infty}\int_1^t\frac{dx}{x}= \left.\lim_{t\rightarrow + \infty}\ln x \right|_1^t = \lim_{t\rightarrow + \infty}\ln t - 0 = +\infty
\end{align*}

\section{Função definida por uma integral}
\label{sec:s12}

\hspace{5mm} Seja $f \in (-\infty, +\infty) \rightarrow \mathds{R}$ contínua e tal que $\displaystyle{\int_{-\infty}^x f(t)dt}$ exista $\displaystyle{\forall x \in \mathds{R}}$. Podemos afirmar que $F:\mathds{R}\rightarrow\mathds{R}$, $\displaystyle{F(x)=\int_{-\infty}^x f(t)dt}$. Fixe $a \in \mathds{R}$. Então:

\begin{align*}
\int_a^x f(t)dt &= \int_a^b f(t)dt + \int_b^x f(t)dt \\
F(x) = \lim_{a\rightarrow - \infty}\int_a^xf(t)dt &= \underbrace{\int_{-\infty}^b f(t)dt}_{constante} + \underbrace{\int_b^x f(t)dt}_{H(x)}\\
H(x) &= \int_b^x f(t)dt \\
F(x) &= \underbrace{\int_{-\infty}^b f(t)dt}_{constante}+ \underbrace{H(x)}_{derivavel} 
\end{align*}

$H$ é derivável e $H'(x) = f(x)$ (T.F.C.), logo $F$ é derivável e $F'(x)=H'(x)=f(x)$

\vspace{20pt}
Exemplo 1: Dada $f:\mathds{R}\rightarrow\mathds{R}$, $\hbox{f}(t)
= \left\{ \begin{array}{rll}
2 & \hbox{se} &  |t| \leq 1 \\
0 & \hbox{se} &  |t|  > 1 \\
\end{array}\right.$, esboçe o gráfico de $\displaystyle{F(x)=\int_{-\infty}^x f(t)dt}$
\begin{figure}[ht]
\centering
\includegraphics[width=0.5\textwidth]{enun}
\caption{gráfico de f(x): exemplo 1}
\label{fig:en1}
\end{figure}

Resolução:
\begin{itemize}
\item se $\displaystyle{x<-1 \Rightarrow F(x) = \int_{-\infty}^x 0\hspace{1mm}dt = 0}$
\item se $\displaystyle{ -1 \leq x < 1 \Rightarrow F(x) = \int_{-\infty}^{-1} f(t) dt + \int_{-1}^x f(t)dt = 0 + \int_{-1}^x \hspace{1mm}dt = 2x+2}$
\item se $\displaystyle{ x \geq 1 \Rightarrow F(x) = \int_{-\infty}^{-1} f(t) dt + \int_{-1}^1 f(t)dt + \int_1^{+\infty}f(t)dt = 0 + 4+ \int_1^x 0 \hspace{1mm}dt = 4}$
\end{itemize}

$\hbox{F}(x)
= \left\{ \begin{array}{rll}
0 & \hbox{se} &  x <-1 \\
2x+2 & \hbox{se} &  -1 \leq x < 1 \\
4 & \hbox{se} & x \geq 1 \\
\end{array}\right.$

\begin{figure}[ht]
\centering
\includegraphics[width=0.5\textwidth]{resp}
\caption{Gráfico Resultate da Operação}
\label{fig:res1}
\end{figure}
\vspace{20pt}

\hspace{5mm}Tente você mesmo: Esboçe o gráfico da função $\displaystyle{F(x) = \int_{-\infty}^x f(t)dt}$, onde $\hbox{f}(t)
= \left\{ \begin{array}{rll}
\displaystyle{\frac{1}{t}} & \hbox{se} &  t \geq 1 \\
0 & \hbox{se} &  t < 1 \\
\end{array}\right.$

\section{Critério de Comaparação}
\label{sec:13}

\hspace{5mm} Ferramenta Importante na classificação de integrais impróprias

\begin{theorem}[Critério de Comparação]
Sejam $f$ e $g$ funções integráveis em $[a;t]$ para todo $t\geq a$ e tais que $0\leq f(x)\leq g(x)\ \forall x \geq a$.  Assim temos que:
\begin{itemize}
\item $\displaystyle{\int_a^{\infty} g(x)dx}$ é convergente $\Rightarrow$ $\displaystyle{\int_a^{\infty} f(x)dx}$ é convergente
\item $\displaystyle{\int_a^{\infty} f(x)dx}$ é divergente $\Rightarrow$ $\displaystyle{\int_a^{\infty} g(x)dx}$ é divergente
\end{itemize}
\vspace{10pt}
\begin{corollary}
Seja $f:[a;+\infty) \rightarrow \mathds{R}$ integrável em $[a;t]$ para todo $t \geq a$, então vale que se $\displaystyle{\int_a^{+\infty}|f(x)| dx}$ for convergente, então $\displaystyle{\int_a^{+\infty} f(x) dx}$ também será.
\end{corollary}
\end{theorem}
\vspace{10pt}

Suponha que $F(x)$ seja crescente em $[a;+\infty)$, prove que $\displaystyle{\lim_{x \rightarrow +\infty} F(x) }$ será finito \footnote{nesse caso seria igual a $M = sup\{F(x)| x \in [a;+\infty)\}$ } ou não.

\begin{demonstration}
Sendo $f(x)$ integrável em $[a,t] \ \ \forall t \geq a$ e $f(x) \geq 0$, a função $\displaystyle{F(x) = \int_a^x\ f(t)\ dt}$ é crescente. De fato, sejam $x_1$ e $x_2 \in [a,+\infty)$, então:

\begin{align*}
F(x_2)-F(x_1) &= \int_a^{x_2} f(t)dt - \int_a^{x_1} f(t)dt \\
&= \int_{x_1}^{x_2} f(t)dt \geq 0 \Rightarrow F(x_2) \geq F(x_1)
\end{align*}

Como $\displaystyle{\int_a^{+\infty} g(t)dt}$ é convergente, então $\displaystyle{\exists \lim_{t \rightarrow +\infty} \int_a^t g(x)\ dx} $ (será $M \geq 0$). E como $0 \leq f(x) \leq g(x)$, então $\forall t \geq a \Rightarrow F(x) = \int_a^t f(x)dx \leq \int_a^t g(x)dx \leq \int_a^{+\infty} f(x)dx = M$
\end{demonstration}
\vspace{10pt}

Pelo que se vê acima, $\displaystyle{\lim_{t\rightarrow+\infty} F(t) = \lim_{t\rightarrow+\infty} \int_a^t f(x)dx}$ exite e é finito. Logo, $\displaystyle{\int_a^{+\infty} f(t)dt}$ é \textbf{convergente}.

Interessante destacar que se tivermos a integral indefinida de $\displaystyle{f(x) =  \frac{g(x)}{x^{\alpha}}}$ com $g(x)$ limitado \footnote{$\displaystyle{\lim_{x \rightarrow +\infty} g(x)} = L \neq \infty $}, então sua classificação depende únicamente da classificação de $1 /x^\alpha$ 

\section{Descobrindo se é ou não convergente -- Exemplo de exercícios}
\label{sec:s14}

\subsection*{Exemplo 1}
\label{sub:ex141}

Mostrar que $\displaystyle{\int_0^{+\infty} e^{-x} \cos(\sqrt x) dx} $ é  convergente.
\begin{align*}
f(x) &= e^{-x} \cos(\sqrt x) \\
|f(x)| &= e^{-x}  |\cos(\sqrt x)| \leq e^{-x} = g(x) \\
\end{align*}
Se $\displaystyle{\int_0^\infty \ g(x)dx}$ for convergente, $\displaystyle{\int_0^{+\infty}f(x)dx }$ também o será.
$$ \int_0^{+\infty}e^{-x}dx = \lim_{t \rightarrow +\infty} \left.-e^{-x} \right|_0^t = - \lim_{t \rightarrow +\infty} \left[\frac{1}{e^t}-1 \right] = 1$$

$\displaystyle{\int_0^{+\infty}g(x)dx }$ é convergente, logo $\displaystyle{\int_0^{+\infty}f(x)dx }$ também  é convergente.

\subsection{Exemplo 2}
\label{sub:ex142}

Mostre que $$ \int_1^{\infty} \frac{x^2+1}{x^3+1} $$ é divergente.

\begin{align}
\frac{x^2+1}{x^3+1} \geq \frac{x^2}{x^3+1} = \frac{1}{x+\frac{1}{x^2}} = \frac{1}{x(1+\frac{1}{x^2})}\\
x^2 \geq 1 \Rightarrow \frac{1}{x^2} \leq 1 \\
1 + \frac{1}{x^2} \leq 2 \Rightarrow \frac{1}{1+\frac{1}{x^2}} \geq \frac{1}{2}\\
\frac{x^2+1}{x^3+1} \geq \frac{1}{x}\left(\frac{1}{1+\frac{1}{x^2}}\right) \geq \frac{1}{2x}
\end{align}

Como $ \displaystyle{\int_1^{\infty} \frac{1}{2x}} $ é divergente, $\displaystyle{ \int_1^{\infty} \frac{x^2+1}{x^3+1}} $ também é

\subsection{Exemplo 3}
\label{sub:ex143}

\hspace{5mm}Seja $\alpha \in \mathds{R} | \alpha >0 $, então mostre que $\displaystyle{\int_1^\infty \frac{dx}{x^\alpha}} $:
\begin{enumerate}
\item [a.] é convergente se $\alpha > 1$\\
\item [b.] é divergente se $\alpha \in (0,1) $
\end{enumerate}

Para $0  < \alpha \neq 1$ temos que $$\displaystyle{\int_1^\infty \frac{dx}{x^\alpha} = \lim_{t \rightarrow +\infty} \int_1^t x^{-\alpha}\ dx = \lim_{t \rightarrow +\infty} \left[ \frac{x^{-\alpha + 1}}{-\alpha + 1}\right]_1^t = \lim_{t \rightarrow +\infty} \left( \frac{t^{1-\alpha}-1}{1-\alpha}\right)}$$

\begin{enumerate}
\item [a.] $\alpha > 1 \Rightarrow 1 - \alpha < 0$. Quando t tende a infinito, $t^{1-\alpha}$ tende a 0: assim temos uma integral indefinida convergente $$\displaystyle{\int_1^\infty \frac{dx}{x^\alpha}} = \frac{1}{\alpha - 1}$$
\item [b.] $0 < \alpha < 1 \Rightarrow 1 - \alpha > 0$. Quando t tende a infinito, $t^{1-\alpha}$ tende a infinito também: assim temos uma integral indefinida divergente.
\end{enumerate}

\subsection{Exemplo 4}
\label{sub:ex144}

\hspace{5mm} Verifique se são convergentes:
\begin{enumerate}
\item [a.] $\displaystyle{\int_1^{\infty} \frac{\ln x}{x \ln \left(x+1\right)} dx}$
\item [b.] $\displaystyle{\int_{10}^{\infty}\frac{x^5-3}{\sqrt{x^{20}+x^{10}+1}} dx}$
\end{enumerate}

\begin{enumerate}
\item [a.] Seja  $\displaystyle{g(x) = \frac{\ln x}{\ln {x+1}}}$ uma função limitada $\displaystyle{\left(\lim_{x \rightarrow +\infty} \frac{\ln x}{\ln {x+1}} = 1\right)}$ e o fato de que $\displaystyle{\int_1^{\infty} \frac{1}{x}}$ é divergente, então a integral em questão é divergente.
\item [b.] $$f(x) = \frac{x^5-3}{\sqrt{x^{20}+x^{10}+1}} = \frac{x^5\left(1-\frac{3}{x^5}\right)}{\sqrt{x^{20}(1+x^{-10}+x^{-20})}}= \underbrace{1 \over x^5}_{g(x)} \underbrace{\frac{1-\frac{3}{x^5}}{1+x^{-10}+x^{-20}}}_{h(x)} = g(x)*h(x)$$ 
$$\lim_{x \rightarrow +\infty} h(x) = 1$$ 
$\displaystyle{\int_{10}^{\infty} g(x)\ dx}$ é convergente. Logo, $\displaystyle{\int_{10}^{\infty} f(x)\ dx}$ é convergente.
\end{enumerate}

\chapter{Sequências e Séries Numéricas} Aulas dadas entre 10/08/2015 e 26/08/2015
\label{chap:c2}

\section{Sequências e limites de sequências}
\label{sec:s21}

\hspace{5mm} Uma sequência  é uma função \footnote{$\mathds{N}^s$ geralmente é da forma $\left\{ n \in \mathds{N} | n \geq q , \ q \ fixo \right\} $ } \begin{align*}
f: \underbrace{\mathds{N}^s \subset \mathds{N}}_n \rightarrow \underbrace{\mathds{R}}_{a_n} \\
(a_n) \ n \in \mathds{N} \\
a_n &= n,\ para \ n \geq 0 \\
a_n &= \frac{1}{n} ,\ para \ n \geq 1 \\
\end{align*}

O que acontece quando n tende a infinito?

\begin{enumerate}
\item $(a_n) \ n \in \mathds{N} a_n = n$
\item $\frac{1}{2}, \ \frac{1}{4}, \ \frac{1}{8}, \ \frac{1}{16}, \ \hdots a_n = \frac{1}{2^n}, \ \  n\geq 1 $
\item 1, -1, 1, -1 , 1 , -1 $\hdots a_n = (-1)^n $ Vai para 0, 1 ou -1?
\end{enumerate}
\begin{definition}[Convergência]
Dizemos que a sequência $(a_n)$ converge para o limite $L \in \mathds{R}$ dado $\epsilon > 0$, existe $n_0 \in \mathds{N}$ tal que $$ n \geq n_0 \Rightarrow |a_n - L| < \epsilon \Longleftrightarrow L-\epsilon < a_n < L + \epsilon $$
\end{definition}

$\displaystyle{\lim_{n \rightarrow + \infty} a_n = + \infty} $ acontece quando $\forall \ R > 0 $ existe $n_0 \in \mathds{N} $ tal que $$n \geq n_0 \Rightarrow a_n > R$$

$\displaystyle{\lim_{n \rightarrow + \infty} a_n = - \infty} $ acontece quando $\forall \ R > 0 $ existe $n_0 \in \mathds{N} $ tal que $$n \geq n_0 \Rightarrow a_n < R$$

Voltando aos exexmplos do começo do capítulo:
\begin{enumerate}
\item \begin {align*} 
a_n &= n \\
\lim_{n \rightarrow + \infty} a_n &= + \infty\\
\end{align*}
\item \begin{align*}
a_n &= \frac{1}{2^n}\\
\lim_{n \rightarrow + \infty} a_n &= 0\\
\end{align*}
Vamos verificar esse fato. Dado $\epsilon > 0, \ \exists \ n_0 \in \mathds{N}$ tal que $\displaystyle{ n \geq n_0  \Rightarrow \left\| \frac{1}{2^n} - 0 \right\| <  \epsilon \Longleftrightarrow \frac{1}{2^n} < \epsilon \Longleftrightarrow 2^n > \frac{1}{\epsilon}}$.   Basta tomarmos $n_0 \in \mathds{N}$ tal que $\displaystyle{2^{n_0} > \frac{1}{\epsilon}}$ e $$ n \geq n_0 \Rightarrow 2^n \geq 2^{n_0} > \frac{1}{\epsilon} \Rightarrow \frac{1}{2^n} < \epsilon$$
\item $a_n = (-1)^n$. O limite dessa expressão não pode ser $\pm \infty$ pois para n tendendo a infinito, teremos $-1 \leq a_n \leq 1 $.  Seja $\displaystyle{0 < \epsilon < \frac{1}{\epsilon} } $. Dado qualquer $L \in \mathds{R}$, o intervalo $]L- \epsilon, L+ \epsilon[ $ tem diâmetro $\displaystyle{2\epsilon < \frac{2}{3}}$. 
Portanto, esse intervalo não comporta  todos os elementos da sequência, logo $$ \not \exists \lim_{n \rightarrow + \infty} (-a)^n$$
\item [\textbf{Adendo}] Seja $(a_n) \ n \in \mathds{N}$  e sejam $n > 0 \land s > 1$ com s fixo. Assim teremos $\displaystyle{\lim_{n \rightarrow + \infty} \frac{1}{n^s} = 0}$
\end{enumerate}

\textbf{Fatos Importantes}
\hspace{5mm} Sejam $(a_n)$ $(b_n)$ duas sequências de números de números reais. Então \begin{enumerate}
\item se $a_n \rightarrow a $ e $b_n \rightarrow b$ $\Rightarrow a_n + b_n \rightarrow a + b$
\item se $a_n \rightarrow a $ e $b_n \rightarrow b$ $\Rightarrow a_nb_n \rightarrow ab$
\item se $a_n \rightarrow a $ e $\lambda \in \mathds{R}$ $\Rightarrow \lambda a_n \rightarrow \lambda a$
\item se $a_n \rightarrow a $ e $b_n \rightarrow b \neq 0$, então $b_n \neq 0$ para n suficientemente grande, então $\displaystyle{\frac{a_n}{b_n} = \frac{a}{b}}$
\end{enumerate}

\begin{lemma}[Critério do Confronto]
Sejam $(a_n)$ $(b_n)$ $(c_n)$ sequências tais que $a_n \leq b_n \leq c_n$. Suponha que existam os limites existam e que seja verdade que $\displaystyle{\lim_{n \rightarrow + \infty} a_n = L = \lim_{n \rightarrow + \infty} c_n}$. Então: $$\lim_{n \rightarrow + \infty} b_n = L$$
\end{lemma}

\begin{demonstration}
Seja um $\epsilon > 0$ suficientemente pequeno. Como $a_n$ e  $c_n$ tendem a $L$, existe $n_0 \in \mathds{N}$ tal que:
\[ n \geq n_0 \Rightarrow
  \begin{cases}
    |a_n - L| < \epsilon \quad \Longleftrightarrow  & \quad L - \epsilon < a_n < L +\epsilon \\
    |c_n - L| < \epsilon \quad \Longleftrightarrow  & \quad L - \epsilon < c_n < L +\epsilon \\
  \end{cases}
\]

Então $n \geq n_0 \Rightarrow L - \epsilon < a_n \leq b_n \leq c_n < L + \epsilon$ ou seja $n \geq n_0 \Rightarrow L- \epsilon < b_n < L + \epsilon \Rightarrow \displaystyle{\lim_{n \rightarrow + \infty} b_n = L}$
\end{demonstration}

\begin{lemma}
Seja $f$ uma função definida num intervalo $I \subset \mathds{R}$, exceto em $c \in I$ e suponha que exista $\displaystyle{\lim_{x \rightarrow c} f(x) = L}$. Seja $(a_n)$ uma sequência que satisfaça \begin{itemize}
\item [a.] $a_n \in I$
\item [b.] $a_n \neq c $
\item [c.] $\displaystyle{\lim_{n \rightarrow + \infty} a_n = c} $
\end{itemize}
Assim, a sequencia $(b_n)_{n \in \mathds{N}} =(f(a_n))_{n \in \mathds{N}}$ é tal que $\displaystyle{\lim_{n \rightarrow + \infty} b_n = L}$ \footnote{Isso equivale a dizer que $\displaystyle{\lim_{n \rightarrow + \infty} f(a_n) = L} $}
\end{lemma}

\begin{demonstration}Dado $\epsilon > 0$ e $\displaystyle{\lim_{x \rightarrow c} f(x) = L}$, existe $\delta > 0$ tal que $$ |x-c| < \delta \Rightarrow |f(x) - L| < \epsilon $$ \hspace{5mm} Como $a_n \rightarrow c$, para tal $\delta > o$ existe $n_0 \in \mathds{N}$ tal que $n \geq n_0 \rightarrow |a_n - c| < \delta$. Então: $$n \geq n_0 \rightarrow |a_n - c| < \delta \Rightarrow |f(a_n) - L| < \epsilon$$ $$ \lim_{n \rightarrow +\infty} f(a_n) = L$$
\end{demonstration}

\begin{obs} Se $f$ está defiida entre $[a;+\infty)$, se $a_n \rightarrow +\infty$ e $\displaystyle{\lim_{x \rightarrow +\infty} f(x) =L}$, então $$\lim_{n \rightarrow +\infty} f(a_n) =L$$
\end{obs}

\section{Exercícios da aula (10/08)}
\label{sec:s22}

\subsection*{Exemplo 1}
\label{subsec:ex221}
\hspace{5mm}Calcular $$\lim_{n \rightarrow +\infty} \frac{(5n-3)^3}{n(n^2+1)}$$
Desenvolvendo o binio de cima, temos $(5n-3)^3 = 125n^3-75n^2+45n -27$. Assim temos $a_n$ como na forma abaixo. Se dividirmos numerador e denominador por $n^3$, ficará fácil determinar o limite em questão.
\begin{align*}
a_n &= \frac{125n^3-75n^2+45n -27}{n^3+n} =  \frac{125 - \frac{75}{n} + \frac{45}{n^2} - \frac{27}{n^3}}{1 + \frac{1}{n^2}} \\
\lim_{n \rightarrow + \infty} a_n &= \lim_{n \rightarrow + \infty} \frac{125 - \frac{75}{n} + \frac{45}{n^2} - \frac{27}{n^3}}{1 + \frac{1}{n^2}} = 125\\
\end{align*}

\subsection*{Exemplo 2}
\label{subsec:ex222}
\hspace{5mm}Calcular $$\lim_{n \rightarrow +\infty} (\sqrt{n+1} - \sqrt{n})$$
\footnote{$\sqrt{n+1} + \sqrt{n} \geq 2\sqrt{n} \Longleftrightarrow \frac{1}{\sqrt{n+1} + \sqrt{n}} \leq \frac{1}{2\sqrt{n}} $}
\begin{align*}
0 \leq a_n = \sqrt{n+1} - \sqrt{n} &= \frac{(\sqrt{n+1} - \sqrt{n})(\sqrt{n+1} + \sqrt{n})}{\sqrt{n+1} + \sqrt{n}} \\
\frac{(n+1) - n}{\sqrt{n+1} + \sqrt{n}} &= \frac{1}{\sqrt{n+1} + \sqrt{n}} \leq \frac{1}{2\sqrt{n}}\\
0 \leq a_n \leq \frac{1}{2\sqrt{n}} & \land \lim_{n \rightarrow +\infty}\frac{1}{2\sqrt{n}} = 0  \\
\end{align*}
\hspace{5mm} Logo, $\displaystyle{\lim_{n \rightarrow +\infty} (\sqrt{n+1} - \sqrt{n}) = 0}$

\subsection*{Exemplo 3}
\label{subsec:ex223}
\hspace{5mm}Demonstrar que $$\lim_{n \rightarrow +\infty} a_n = 1$$ onde $a_n = \sqrt[n]{n},\ n \geq 1$

$a_n = \sqrt[n]{n} = 1 + b_n, \ b_n > 0$, então $a_n \rightarrow 1 \Longleftrightarrow b_n \rightarrow 0$.
$$n = a_n^n =(1+ b_n)^n = 1 + nb_n + \frac{n(n-1)}{2}b_n^2 + \hdots + b_n^n \geq 1 + \frac{n(n-1)}{2}b_n^2$$

Logo $\displaystyle{n-1 \geq \frac{n(n-1)}{2}b_n^2 \Rightarrow b_n^2 \leq \frac{2}{n} \Rightarrow 0 < b_n \leq \frac{\sqrt{2}}{\sqrt{n}}}$

Pelo critério de compração, $b_n \rightarrow 0$ e $a_n \rightarrow 1$.

\subsubsection{Pensando de uma outra forma} Tome $f: [1;+\infty) \rightarrow \mathds{R} \ \ f(x) = x^{\frac{1}{n}}$

\begin{align*}
f(x) = x^{\frac{1}{x}} &= e^{\ln x^{\frac{1}{x}}} =   e^{\frac{\ln x}{x}} \Longleftrightarrow \\
\lim_{x \rightarrow +\infty} \frac{\ln x}{x} &= 0 \Longleftrightarrow \\
\lim_{x \rightarrow +\infty} e^0 &= 1 \\
Logo, \ \sqrt[n]{n} & \rightarrow 1 \\
\end{align*}

\vspace{10mm}
Vamos extender esses conceitos ...

\subsubsection {Mostrar que $\displaystyle{\lim_{n \rightarrow +\infty} \sqrt[n]{a}} = 1 \ \ \forall \ a > 0$}
\label{sssec:ss2232}
\begin{itemize}
\item [\textbf{1}. $ a > 1 $] Para $b_n > 0$ e $a_n = \sqrt[n]{a} = 1 + b_n$ \\ $a = a_n^n = (1 + b_n)^n$. Nesse caso, proceder como no exemplo \ref{subsec:ex223} 
\item [\textbf{2}. $ 0 < a < 1 $] $ 0 < a < 1 \Longleftrightarrow \frac{1}{a} > 1$ Nesse caso teremos que $$ \sqrt[n]{\frac{1}{a}} \rightarrow 1 $$ $$\frac{1}{\sqrt[n]{a}}\rightarrow 1 $$
\end{itemize}

\subsubsection {Calcular $\displaystyle{\lim_{n \rightarrow +\infty} \sqrt[n]{a^n + b^n}}$}
\label{sssec:ss2233}
\hspace{5mm} Dados $0 < a \leq b$ e $n \geq 1$.

\begin{align*}
a_n &= \sqrt[n]{a^n+b^n} > \sqrt[n]{b^n} = b \\
a_n &= \sqrt[n]{a^n+b^n} \leq \sqrt[n]{2b^n} = b\sqrt[n]{2}\\
b &< a_n \leq b\underbrace{\sqrt[n]{2}}_{= 1}\\
\lim_{n \rightarrow + \infty} a_n &= b \\
\end{align*}

\section{Sequências Monótonas}
\label{sec:s23}

\hspace{5mm}Toda sequência convergente é limintada $$a_n \rightarrow a \Rightarrow \exists M \in \mathds{R} \ | \ \forall n \in \mathds{N} \ \ |a_n| \leq M$$ 

De fato, dado $\epsilon = \frac{1}{3}$, então $\exists n_0 \in \mathds{N}$ tal que $$ n \geq n_0 \Rightarrow |a_n - a| < \epsilon$$

Logo, se $n \geq n_0$, temos que $$ |a_n| = |a_n - a + a| \leq |a_n - a| + |a| < |a| + \frac{1}{3}$$

Seja $M = max \left\{ |a_0|, |a_1|, \hdots , |a| +\frac{1}{3} \right\}$. Logo $|a_n| \leq M$

\begin{definition}[Sequências Monótonas]
\hspace{5mm} Pode ser:
\begin{itemize}
\item \underline{Crescente} : se $n \geq m \Rightarrow a_n \geq a_m $
\item \underline{Decrescente} : se $n \geq m \Rightarrow a_n \leq a_m $
\end{itemize}
\end{definition}

\begin{lemma}
Se $(a_n)$ for crescente e limitada, então $\displaystyle{\exists \lim_{n\rightarrow+\infty} a_n}$\footnote{Analogamente, se a sequência for decrescente e limitada inferiormente. A demonstração também é análoga a do caso supra citado}
\end{lemma}

\begin{demonstration}
Se $ (a_n) $ for limitada superiormente, o conjunto$\left\{a_n, n \in \mathds{N}\right\}$ é limitado superiormente e postanto existe $ a = sup \left\{a_n, n \in \mathds{N}\right\}$.

Dado $\epsilon > 0$, existe $n_0 \in \mathds{N}$ existe $a - \epsilon < a_{n_0} \leq a$
\begin{align*}
\forall n \geq n_0 \Rightarrow a - \epsilon < a_{n_0} &\leq a_n \leq a \\
n \geq n_0 \Rightarrow a - \epsilon &< a_n \leq a \\
\lim_{n \rightarrow +\infty} a_n &= a \\
\end{align*}
\end{demonstration}

\begin{definition}[Sequências de Couchy (Critério de Couchy)]
Se $a_n \rightarrow a$, os termos vão ficando arbitrariamente próximos na medida em que n aumenta. 

Dado $\epsilon > 0 $, existe $n_0 \in \mathds{N}$ tal que $n \geq n_0 \Rightarrow |a_n - a| < \frac{\epsilon}{2}$. Logo, se $m, \ n \geq n_0$, então $|a_m - a_n| = |(a_m + a) - (a + a_n)| \leq |a_m - a| + |a_n -a| < \displaystyle{\frac{\epsilon}{2}+\frac{\epsilon}{2}} = \epsilon$

Uma sequência $(a_n)$ é uma sequência de Couchy, se para qualquer $\epsilon > 0$, existir $n_0 \in \mathds{N}$ tal que $m, \ n \geq n_0 \Rightarrow |a_m - a_n| < \epsilon$
\end{definition}

\begin{lemma}
Toda sequência de Couchy é convergente. 
\end{lemma}

\begin{demonstration}
\textbf{Digamos que} $a_n$ satisfaz $m, \ n \geq n_0 \Rightarrow |a_m - a_n| < \epsilon$. De fato, para qualquer $k > 0$, existe $n_k \in \mathds{N}$ tal que para quaisquer $ m, \ n \geq n_k \Rightarrow |a_n - a_m| < \displaystyle{\frac{1}{k}}$.

Toamando $m = n_k$ fixo, temos: $$ n \geq n_k \Rightarrow |a_{n_k} - a_n| < \frac{1}{k} $$ ou mesmo $$ n \geq n_k \Rightarrow a_{n_k} - \frac{1}{k} < a_n < a_{n_k} + 1 $$. Seja $M = max\left\{ |a_0|, |a_1|, \hdots , |a_{n_k}|, |a_{n_k} - \frac{1}{k}|, |a_{n_k} + 1| \right\}$, então $|a_n| \leq M \ \ \forall \ n \in \mathds{N}$

\textbf{Mais algumas considerações:} Seja $A = \left\{ a_n, \ n \in \mathds{N} \right\}$ limitado e $A_n = \left\{ a_n, \ a_{n+1}, \ a_{n+2} \right\}$ também limitado. E por último, consideremos $L_n = sup \ A_n$ e $l_n = inf \ A_n$ onde $l_n \leq L_n$
\begin{align*}
A_n &= \left\{ a_n, \ a_{n+1}, \ a_{n+2}, \hdots \right\} \\
A_{n+1} &= \left\{ a_{n+1}, \ a_{n+2}, \hdots \right\} \\
L_n  &\geq L_{n+1} \\
l_n & \leq l_{n+1} \\
\end{align*}
\begin{itemize}
\item Se $L_n$ for decrescente, então $\displaystyle{\exists \lim_{n \rightarrow +\infty} L_n = L}$
\item Se $l_n$ for crescente, então $\displaystyle{\exists \lim_{n \rightarrow +\infty} l_n = l}$
\end{itemize}
\textbf{Por último ...} $\displaystyle{L = l = \lim_{n \rightarrow +\infty} a_n}$ Como $(a_n)$ é de Couchy, já sabemos que $\forall k \in \mathds{N}, \ k > 0 \Rightarrow \exists n_k \in \mathds{N}$ tal que $n \geq n_k \Rightarrow a_{n_k} - \frac{1}{k} < a_n < a_{n_k} + 1$. Logo, $a_{n_k} - 1 $ e $a_{n_k} + 1 $ são cotas inferior e superior do conjunto $A_{n_k} = \left\{ a_n \ | \ n \geq n_k \right\}$. Para todo $n \geq n_k$, temos $a_{n_k} - \frac{1}{k} \leq l_n \leq L_n \leq a_{n_k} + \frac{1}{k}$. Logo $$ 0 \leq L_n - l_n \leq (a_{n_k} + \frac{1}{k}) - (a_{n_k} - \frac{1}{k}) = \frac{2}{k}$$ Pelo critério de comparação, $ 0 < L - l \leq 0 \Longleftrightarrow L = l = a $. Para todo $n \in \mathds{N}$ temos $ l_n \leq a_n \leq L_n$. Pelo critério de comparação, sabemos que $$ \exists \lim_{n \rightarrow +\infty} a_n = a $$
\end{demonstration}

\section{Subsequências}
\label{sec:s24}
\hspace{5mm} Tome essas sequências:
\begin{align*}
a_n &= \frac{1}{n}, \ n \geq 1 \\
a_{n_k} &= \frac{1}{2k}, \\
a'_{n_k} &= \frac{1}{2k+1} \\
\end{align*}

Dada uma sequência $\displaystyle{a'_{n_k} = \frac{1}{k^2}}$ é uma subsequência de $a_n$ se $\displaystyle{\lim_{n \rightarrow +\infty} a_n = a \Longleftrightarrow \lim_{k \rightarrow +\infty} a_{n_k} = a} \ \forall \ (a_n)$ subsequência de $(a_n)$.

\textbf{Importante: }Sejam $(a_{n_k})$ e $(a'_{n_k})$ subsequências de $(a_n)$ e suponha que $\not \exists \lim_{k \rightarrow +\infty} a_{n_k}$ ou mesmo que $$\lim_{k \rightarrow +\infty} a_{n_k} = a \neq \lim_{k \rightarrow +\infty} a'_{n_k} = b $$. Então $(a_n)$ não é convergente.

\section{Séries Numéricas: Definições e exmplos}
\label{sec:s25}
\hspace{5mm} Exemplos:
\begin{enumerate}
\item $\displaystyle{1 = \frac{1}{2} + \frac{1}{4} + \frac{1}{8} + \frac{1}{16} + \hdots}$ \label{ex:251}
\item $\displaystyle{1 = \frac{9}{10} + \frac{9}{100} + \frac{9}{1000} + \frac{9}{10000} + \hdots = 0,99999 \hdots} $ \label{ex:252}
\item $\displaystyle{1 -1 + 1 - 1 + 1 - 1 + \hdots} $ \label{ex:253} \\
Alguns fazem $(1-1) + (1-1) + (1-1) + \hdots = 0$. \\ Outros fazem $1 + (-1 + 1) + (-1 + 1) + (-1 + 1) + (-1 + 1) + \hdots = 1$. \\ Liebniz propos $\frac{1}{2} = 1 + 1 - 1 + 1 - 1 + 1 - 1 + \hdots$ \\
\end{enumerate}

Seja $(a_n)$ uma sequência  de números reais. A ela associamos a \underline{soma infinita}. $$\sum_{n = 0}^{\infty} a_n = a_0 + a_1 + a_2 + \hdots$$

Consideramos a sequência $(S_m)$ das \underline{somas parciais}
\begin{align*}
S_0 &= a_0 \\
S_1 &= a_0 + a_1 \\
S_2 &= a_0 + a_1 + a_2 \\
\vdots \\
S_m &= a_0 + a_1 + \hdots + a_m = \sum_{n=0}^m a_n \\
\end{align*}


\begin{definition}
Uma série $\sum_{n=0}^{\infty} a_n$ se diz \underline{convergente}. Se a sequência de somas parciais $(S_m)$ é convergente, neste caso, se $$ \lim_{n \rightarrow +\infty} S_n = S \Rightarrow \sum_{n=0}^{+\infty} a_n = S $$ onde S é a \underline{soma} da série. \footnote{$\displaystyle{\sum_{n=0}^{+\infty} a_n = \lim_{m \rightarrow +\infty} \left(\sum_{n=0}^m a_n\right)}$ }
\end{definition}


Voltando aos exemplos:
\begin{itemize}
\item \textbf{Exemplo \ref{ex:251}: } $\displaystyle{\frac{1}{2} + \frac{1}{4} + \frac{1}{8} + \frac{1}{16} + \hdots = \sum_{n=0}^{\infty} \frac{1}{2^{n+1}}}$
\item \textbf{Exemplo \ref{ex:252}: } $\displaystyle{\frac{9}{10} + \frac{9}{100} + \frac{9}{1000} + \frac{9}{10000} + \hdots = \sum_{n=0}^{\infty} \frac{9}{10^{n+1}}}$.
\end{itemize}
Esses são exemplos de \underline{séries geométricas} que são séries do tipo $\displaystyle{\sum_{n=0} a.r^n}$, onde $a, \ r \in \mathds{R}$. No primeiro caso, $a=r=\frac{1}{2}$, e no segundo $a = \frac{9}{10}, \ r = \frac{1}{10}$. \\ $r = 0 \Rightarrow $ converge a 0 \\ $r = 1 \Rightarrow $ diverge.

Supondo $r \neq 0 \land r \neq 1$. Somas parciais:
\begin{align*}
S_m &= a + a.r + a.r^2 + \hdots + a.r^m \\
r.S_m &= a.r + a.r^2 + \hdots + a.r^{m+1} \\
S_m - r.S_m &= a - a.r^{m+1} \\
S_m &= \frac{a(1-r^{m+1})}{1-r} \\
\end{align*}

Se $0 < |r| < 1$, então $$\lim_{m \rightarrow +\infty} S_m = \lim_{m \rightarrow +\infty} \frac{a(1-r^{m+1})}{1-r} = \frac{a}{1-r}$$ Neste caso, $\displaystyle{\sum_{n=0} a.r^n = \frac{a}{1-r}}$.
\begin{enumerate}
\item $\sum_{n=0}^{\infty} \frac{1}{2^{n+1}} = \frac{\frac{1}{2}}{1-\frac{1}{2}} = 1$
\item $\sum_{n=0}^{\infty} \frac{9}{10^{n+1}} = \frac{\frac{9}{10}}{1-\frac{1}{10}} = 1 $
\end{enumerate}

Se $|r| > 1$ a série diverge, pois $\displaystyle{\lim_{m \rightarrow +\infty} S_m = \lim_{m \rightarrow +\infty} \frac{a(1-r^{m+1})}{1-r} = \infty}$.

Exemplo \ref{ex:253}: 
\begin{align*} 
1 - 1 + 1 - 1 + 1 - 1 + 1 + \hdots \\
S_0 &= 1 \\
S_1 &= 1 - 1 = 0 \\
S_2 &= 1 - 1 + 1 = 1 \\
S_3 &= 1 - 1 + 1  - 1 = 0 \\
S_4 &= 1 - 1 + 1  - 1 + 1 = 1 \\
S_{2k} &= 1, \ k \in \mathds{N} \\
S_{2k+1} &= 0, \ k \in \mathds{N} \\
\lim_{k \rightarrow +\infty} S_{2k} &= 1 \\
\lim_{k \rightarrow +\infty} S_{2k+1} &= 0 \\
\end{align*}

$(S_m)$  tem duas subsequências convergindo a limites distintos, logo $\not \exists \displaystyle{\lim_{n \rightarrow +\infty} S_{n}}$ nem $\displaystyle{\sum_{n=0}^{\infty} (-1)^n}$

\textbf{Propriedades Simples:}
\begin{enumerate}
\item Se $\sum a_n = a$ e $\sum b_n = b$, então $\sum (a_n+b_n) = a+b$
\item Se $\sum a_n = a$ e $\lambda \in \mathds{R}$, então $\sum (\lambda a_n) = \lambda a$
\item A convergência (ou não) de uma série não se altera se omitido um termo finito da soma.
\item $\displaystyle{\sum a_n}$ é convergente, então $\displaystyle{\lim_{n \rightarrow +\infty} a_n = 0}$ \footnote{Pode acontecer de o limite ser nulo, mas a série ser divergente}
\item $\displaystyle{\lim_{n \rightarrow +\infty} a_n \neq 0 \Longleftrightarrow \sum a_n}$ é divergente.
\end{enumerate}

\section{Séries de Termos Positivos -- Critérios de Comparação (Parte I)}
\label{sec:s26}
$$\sum_{n=0}^{\infty} a_n \ \ \ a_n > 0 $$

Convergência de sequêncio:
\begin{enumerate}
\item Sequências monótonas \label{261}
\item Critério de Couchy \label{262}
\end{enumerate}

\textbf{Item \ref{261}}: para a série $\sum a_n$ de termos positivos, temos:
\begin{align*}
S_0 &= a_0 \\
S_1 &= a_0 + a_1 > a_0 = S_0 \\
S_2 &= a_0 + a_1 + a_2 > S_1 \\
\end{align*}

A sequência $(S_m)$ é crescente. Podemos afirmar que uma série $\displaystyle{\sum_{n=0}^{\infty} a_n}$ é convergente, se e somente se, a sequência das somas parciais $(S_m)$ for limitada superiormente. \\

\textbf{Item \ref{262}: Critério de Comparação}: Sejam $\sum a_n$ e $\sum b_n$ séries de termos positivos e suponha que $a_n \leq b_n \ \forall \ n$. Então, se $\sum b_n$ converge, então $\sum a_n$ também converge. \footnote{De outra forma, se $a_n$ diverge, então $b_n$ também o faz.}

\begin{demonstration}
Sejam $\displaystyle{S_m = \sum_{n=0}^{m} a_n}$ e  $\displaystyle{S'_m = \sum_{n=0}^{m} b_n}$ sequências  crescentes tais que $S_m \leq S'_m$

Por hipótese, temos $\displaystyle{\lim_{n \rightarrow +\infty} S'_m = S' = \sum b_n}$. Sabendo que $S'_m = sup \ \left\{S'_m \ | \ m \in \mathds{N}\right\}$, temos então que $S_m \leq S'_m \leq S' \ \forall \ m$.

Logo, temos que $S'$ é cota superior do conjunto $\left\{S_m \ | m \in \mathds{N}\right\}$. Como $S_m$ é convergente, então $\displaystyle{\exists \lim_{n \rightarrow + \infty} S_m = S = \left\{S_m \ | m \in \mathds{N}\right\} \Rightarrow \sum a_n}$ converge. 
\end{demonstration}

\section{Exercícios da aula do dia 17/08}
\label{sec:s27}

\subsection*{Calcular $\displaystyle{\sum_{n=1}^{\infty} e^{\frac{1}{n}}} $}
$$ a_n = e^{\frac{1}{n}} \ \ \lim_{n \rightarrow +\infty} e^{\frac{1}{n}} = 1 \neq 0$$ Logo a série converge.

\subsection*{Determinar se a expressão abaixo converge ou não. $$\sum_{n=1}^{\infty} \frac{1}{n}$$}

Ainda que o limite de $a_n$ tenda pra zero quando n vai pra infinito, essa expressão não é convergente.

\begin{align*}
S_1 &= 1 \\
S_2 &= 1 + \frac{1}{2} = \frac{3}{2} \\
S_3 &= 1 + \frac{1}{2} + \frac{1}{3} = \frac{11}{6} \\
S_4 &= 1 + \frac{1}{2} + \frac{1}{3} + \frac{1}{4} > 1 + \frac{1}{2} + \frac{1}{4} + \frac{1}{4} = 2 \\
S_8 &= S_4 + \frac{1}{5} + \frac{1}{6} + \frac{1}{7} + \frac{1}{8} > 2 + \frac{1}{8} + \frac{1}{8} + \frac{1}{8} + \frac{1}{8} = 2 + \frac{1}{2} = \frac{5}{2} \\
S_{2^3} &> \frac{3 + 2}{2} \\
S_{2^n} &> \frac{n + 2}{2} \ \ \forall \ n \in \mathds{N} \\
\lim_{n \rightarrow +\infty} \frac{n+2}{2} = +\infty &\Rightarrow \lim_{n \rightarrow +\infty} S_{2^n} = +\infty \\
\end{align*}

Logo $S_n$ diverge.

\subsection*{Determine a soma:}
\begin{align*}
\frac{1}{1*2} + \frac{1}{2*3} + \frac{1}{3*4} + \hdots = \sum_{n = 1}^{\infty} \frac{1}{n*(n+1)} \\
S_m &= \frac{1}{1*2} + \frac{1}{2*3} + \frac{1}{3*4} + \hdots + \frac{1}{n*(n+1)}\\
 &= \left(1 - \frac{1}{2}\right) + \left(\frac{1}{2} - \frac{1}{3}\right) + \left(\frac{1}{3} - \frac{1}{4}\right) + \left(\frac{1}{n} - \frac{1}{n+1}\right) \\
 &= 1 - \frac{1}{n+1} \\
 \lim_{n \rightarrow +\infty} S_m = \lim_{n \rightarrow +\infty} \left(1 - \frac{1}{n+1} \right) &=  \sum_{n = 1}^{\infty} \frac{1}{n*(n+1)} = 1 \\
\end{align*}

Nesse caso, $S_n$ converge.

\section{Teste seu conhecimento: (pergunte para o professor ou monitor se tiver dúvida)}
\label{sec:s28}

Determinar quais somas são convergentes:
\begin{itemize}
\item [a] $\displaystyle{\sum_{n = 1}^{\infty} \left(\frac{-1}{n}\right)^n}$
\item [b] $\displaystyle{\sum_{n = 0}^{\infty} 10^{-n} }$
\item [c] $\displaystyle{\sum_{n = 0}^{\infty} \frac{3^n}{2^n}}$
\item [d] $\displaystyle{\sum_{n = 0}^{\infty} \frac{n+5}{2n+4}}$
\item [e] $\displaystyle{\sum_{n = 0}^{\infty} \sin (n\pi)}$
\item [e] $\displaystyle{\sum_{n = 0}^{\infty} \cos (n\pi)}$
\item [a] $\displaystyle{\sum_{n = 0}^{\infty}  (2^{-n} + 3^{-n})}$
\end{itemize}

\section{Critério de Comparação no Limite}
\label{sec:s29}
Sejam $\sum a_n$ e $\sum b_n$ séries de termos positivos.
\begin{itemize}
\item [a] Dizemos que $a_n$ é da ordem de $b_n$ \footnote{$a_n = O(b_n)$} se $\exists k \in \mathds{R}\ |\ a_n \leq k\ b_n$ para n suficientemente grande.
\item [b]  Se $a_n = O(b_n) \ \land \ b_n = O(a_n) \Longleftrightarrow a_n\approx b_n$.
\item [c] $\displaystyle{a_n\approx b_n \ \Rightarrow \lim_{n \rightarrow +\infty} \frac{a_n}{b_n} = 0}$ \footnote{Verificar esse detalhe}
\item [exemplo] 
\begin{align*}
a_n &= n    &    a_n &= O(b_n) \\
b_n &= 10n  &   10n \leq \underbrace{20}_K n &\Rightarrow b_n = O(a_n) \\
\end{align*}
\end{itemize}

\begin{lemma}
Sejam $\sum a_n$ e $\sum b_n$ duas séries de termos positivos. Temos:\begin{itemize}
\item se $a_n = O(b_n)$ e se $\sum b_n$ converge, então $\sum a_n$ converge também.
\item se $a_n = O(b_n)$ e se $\sum b_n$ diverge, então $\sum a_n$ diverge também.
\item se $\displaystyle{\lim_{n \rightarrow +\infty} \frac{a_n}{b_n} = c > 0}$ , então $\sum a_n$ converge e $\sum b_n$ também.
\end{itemize}
\end{lemma}

\hspace{5mm} \textbf{Mostre que} $\displaystyle{\sum_{n=1}^{\infty} \frac{3n + 4}{n^2 + 2}}$ diverge.
\begin{align*}
a_n &= \frac{3n + 4}{n^2 + 2} & b_n &= \frac{1}{n} \\
\lim_{n \rightarrow \infty} \frac{a_n}{b_n} &= \lim_{n \rightarrow \infty} \frac{3n^2 + 4n}{n^2 + 2} = \lim_{n \rightarrow \infty} \frac{3 + \frac{4}{n}}{1 + \frac{2}{n^2}} = 3 > 0 \\
\end{align*}

Como $\sum \frac{1}{n}$ diverge, então $\sum_{n = 1}^{\infty} a_n$ diverge também. \vspace{7mm}

\hspace{5mm} \textbf{Mostre que} $\displaystyle{\sum_{n=1}^{\infty} \underbrace{\frac{\cos^2(n\pi)}{n^2}\ \frac{\sqrt{4n-3}}{n^2+1}}_{a_n}}$ convege.

%Tomaremos $b_n = \frac{1}{n^2}$. Se mostrarmos que $a_n \approx b_n$, teremos que $a_n$ converge, pois $b_n$ também o faz. 
\begin{align}
\sqrt{4n-3} < 4n-3 \leq n^2 \label{eq:1} \\
a_n = \frac{1}{n^2} \ \frac{\sqrt{4n-3}}{n^2 + 1} \underbrace{<}_{\ref{eq:1}} \frac{n^2}{n^2(n^2+1)} < \frac{1}{n^2} \\
\end{align}
$\sum a_n$ converge.

\section{Séries de termos Positivos: Critérios de Comparação (Parte II)}
\label{sec:s210}

\hspace{5mm}Os critérios da parte anterior dependem do conhecimento de séries com determinadas propriedades. Nesta seção, veremos métodos "intrínsecos", ou seja, de métodos que dependem somente das próprias séries.

\subsection{Critério da Razão (D'Alambert)}
Seja $\sum a_n$ Uma série de termos positivos e suponha que exista $\displaystyle{\lim_{n \rightarrow \infty} \frac{a_n}{b_n} = l}$ finito ou não. Desse modo temos que: $\left\{ \begin{array}{rlll}
diverge & \hbox{se} &  l > 1 \\
converge & \hbox{se} & 0 \leq l < 1 \\
inconclusivo & \hbox{se} & l = 1 \\
\end{array}\right.$

\begin{demonstration}
Temos dois casos:
\begin{itemize}
\item [a.] $$\lim_{n \rightarrow \infty} \frac{a_{n+1}}{a_n} = l < 1 $$ Existe $r \in \mathds{R}$ tal que $l < r < 1$. Existe $n \in \mathds{N}$  tal que $$ n \geq N \Rightarrow \frac{a_{n+1}}{a_n} < r$$
Temos : \begin{align}
\frac{a_{N+1}}{a_N} < r &\Rightarrow  a_{N+1} < r\ a_N \\
\frac{a_{N+2}}{a_{N+1}} < r &\Rightarrow  a_{N+2} < r\ a_{N+1} < r^2\ a_N \\
\vdots &\Rightarrow \vdots \\
\vdots &\Rightarrow a_{N+K+1} < a_N\  r^{K+1} \label{eq:210}
\end{align}
Como $0 < r < 1$, a série $\displaystyle{\sum_{n=0}^{\infty} r^n}$ converge $\displaystyle{\Rightarrow \ \sum_{n=N}^{\infty} r^n}$ converge.
Pela relação \ref{eq:210}, temos que $\displaystyle{\sum_{n=N}^{\infty} a_n}$ converge,então $\displaystyle{\sum_{n=0}^{\infty} a_n}$ converge também.
\item [b.] $$\lim_{n \rightarrow \infty} \frac{a_{n+1}}{a_n} = l > 1 $$ Existe $n \in \mathds{N}$  tal que $$ n \geq N \Rightarrow \frac{a_{n+1}}{a_n} > 1 \Rightarrow a_{n+1} > a_n $$

\begin{align*}
a_{N+1} &> a_N \\
a_{N+2} &> a_{N+1} > a_N\\
\vdots\\
a_{N+K} &> a_N & \forall \ k \in \mathds{N}\\
\lim_{n \rightarrow \infty} a_n \geq a_N > 0
\end{align*}
Logo $\sum a_n$ converge.
\end{itemize}
\end{demonstration}

\hspace{5mm}\textbf{Exemplo} Analisar a série abaixo: $$\sum_{n=1}^{\infty} \frac{2^n \ n!}{n^n}$$ \footnote{$\displaystyle{\lim_{n \rightarrow \infty} \left( 1 + \frac{1}{n} \right)^n = e}$}
\begin{align*}
a_n &= \frac{2^n \ n!}{n^n} \\
\frac{a_{n+1}}{a_n} &=  \frac{2^{n+1} \ (n+1)!}{(n+1)^{n+1}} \ \frac{n^n}{2^n \ n!} \\
&= 2\ \left( \frac{n}{n+1}\right)^n = \frac{2}{\left( \frac{n+1}{n}\right)^n}\\
\lim_{n \rightarrow \infty} \frac{a_{n+1}}{a_n} &= \frac{2}{e} < 1 
\end{align*}
Pelo critério da Razão, a série converge.

\subsection{Critério da Raíz(De Cauchy)}

\hspace{5mm} Seja $a_n$ uma série de termos positivos tal que $\displaystyle{\lim_{n \rightarrow \infty} \sqrt[n]{a_n} = l}\  $ tal que $\ \sum a_n  \left\{ \begin{array}{rlll}
diverge & \hbox{se} &  l > 1 \\
converge & \hbox{se} & 0 \leq l < 1 \\
inconclusivo & \hbox{se} & l = 1 \\
\end{array}\right.$

Seja $\int a_n$ uma série de números positivos, então se :
\begin{align}
\lim_{n \rightarrow \infty} \frac{a_{n+1}}{a_n} &= l \Rightarrow \label{eq:cauchy1} \\
%esse rótulo foi nomeado com base na sessão, pra facilitar sua referência a algumas linhas a frente xD
\lim_{n \rightarrow \infty} \sqrt[n]{a_n} &= l
\end{align}

Seja $(a_n)$ uma sequência tal que $\displaystyle{\lim_{n \rightarrow \infty} a_n = a}$ então,  $$\lim_{n \rightarrow \infty} \frac{a_1 + a_2 + \cdots + a_n}{n} = a \ \land \ \lim_{n \rightarrow \infty} \sqrt[n]{a_1 * a_2 * \hdots * a_n} = a$$
\begin{align*}
\lim_{n \rightarrow \infty} \sqrt[n]{a_n} \\
\lim_{n \rightarrow \infty} \sqrt[n]{a_n} &= \sqrt[n]{a_1} \ \sqrt[n]{\underbrace{\frac{a_2}{a_1}}_l\ \underbrace{\frac{a_3}{a_2}}_l \ \underbrace{\frac{a_4}{a_3}}_l \ \cdots \ \underbrace{\frac{a_n}{a_{n-1}}}_{l \ (\ref{eq:cauchy1})}}
\end{align*}

\textbf{Decida} se a série abaixo é convergente. $$\sum_{n = 1}^{\infty} \left( \frac{n}{2n+1}\right)^n$$
\begin{align*}
a_n &= \left( \frac{n}{2n+1}\right)^n \\
\lim_{n \rightarrow \infty} \sqrt[n]{a_n} &= \lim_{n \rightarrow \infty}\frac{n}{2n+1} \\
&= \lim_{n \rightarrow \infty}\frac{1}{2+\frac{1}{n}} = \frac{1}{2}
\end{align*}

Logo, a série converge.

\subsection{Critério da Integral}

\hspace{5mm} Seja $\displaystyle{f: [0;+\infty) \rightarrow \mathds{R}}$ uma função positiva e decrescente, e seja $a_n = f(x)$. Desse modo, a série $\displaystyle{\sum_{n=0}^{\infty} a_n}$ converge, se e somente se, $\displaystyle{\int_0^{\infty} f(x) \ dx}$ também for convergente. 

Vale saber que $$\sum_{n=0}^{\infty} a_n = \int_0^{\infty} f(x) \ dx = \lim_{t \rightarrow +\infty} \left( \int_0^t f(x) \ dx \right)$$

\begin{lemma}
Seja $\sum a_n$ uma série de termos positivos e suponha que haja uma função positiva e decrescente $f: [N, +\infty) \rightarrow \mathds{R}$ tal que $\forall n \geq N, \ f(n) = a_n$. Logo $\sum a_n$ convege se, e somente se a integral $\displaystyle{\int_N^{\infty}  f(x) \ dx }$ convergir também.
\end{lemma}

\section{Exercícios da aula do dia 24/08}
\label{sec:211}

\subsection*{Mostrar que a série $$\sum_{n = 1}^{\infty} \frac{1}{n^s}$$ converge para $s > 1$ e diverge para $s \leq 1$}

\begin{itemize}
\item se $\displaystyle{s = 0} \Rightarrow \sum_{n = 1}^{\infty} \frac{1}{n^0} \Longleftrightarrow \sum_{n = 1}^{\infty} 1$ \\ A série diverge.
\item se $\displaystyle{s = 1} \Rightarrow \sum_{n = 1}^{\infty} \frac{1}{n}$ \\ A série também diverge.
\item se $\displaystyle{0 < s < 1 \Rightarrow n^s < n \Longleftrightarrow \frac{1}{n^s} > \frac{1}{n} \ \ \ \vspace{5mm} \sum_{n = 1}^{\infty} \frac{1}{n^s}}$ diverge pelo critério de comparação.
\item se $\displaystyle{s < 0} \Rightarrow \sum_{n = 1}^{\infty} n^{-s} = \infty$ A série diverge \footnote{se $s < 0 \Longleftrightarrow -s > 0$}
\item se $s > 1$, considere $f: [1, \infty) \rightarrow \mathds{R} \ \ \ \ \ f(x) = \frac{1}{x^s}$

\begin{align*}
\int_1^{\infty} \frac{1}{x^s} \ dx = \lim_{t \rightarrow \infty} \left(\int_1^t x^{-s} \ dx \right) &= \lim_{t \rightarrow \infty} \left[\frac{x^{1-s}}{1-s} \right]_1^t \\
\lim_{t \rightarrow \infty} \frac{t^{1-s}-1}{1-s} &= \frac{-1}{1-s} = \frac{1}{s-1}
\end{align*}
Logo, para $s > 1$, a série converge.
\end{itemize}

\subsection*{Avaliar as séries abaixo} \label{avaliar}
\begin{itemize}
\item [a.] $\displaystyle{\sum_{n=2}^{\infty} \frac{1}{(\ln n)^{\ln n}}}$ 
\item [b.] $\displaystyle{\sum_{n=2}^{\infty} \frac{1}{(\ln n)^{\ln(\ln n)}}}$
\end{itemize}
\textbf{resolução item A}
\\
\textbf{Passo 1:} Tome a função $f:[a, +\infty[ \rightarrow \mathds{R} \ | \  f(x) = \displaystyle{\frac{1}{(\ln x)^{\ln x}}}$, podemos usar, primeiramente, o critério da integral onde teremos algo assim: $$\int_2^{\infty} f(x) \ dx = \lim_{t \rightarrow \infty} \int_2^t \displaystyle{\frac{dx}{(\ln x)^{\ln x}}}$$
\textbf{Passo 2:} Mudança de variável
\begin{align}
y &= \ln x \\ \label{211a}
x &= e^y \\ 
dx &= e^y \ dy \label{211b}
\end{align}
Teremos a seguinte integral $$\lim_{t \rightarrow \infty} \int_2^t \displaystyle{\frac{\overbrace{e^y \ dy}^{\ref{211a}}}{\underbrace{y^y}_{\ref{211b}}}} = \int_2^{\infty} \frac{e^y}{y^y} dy$$

\textbf{Passo 3:} Tomamos uma sequência $b_n = \displaystyle{\frac{e^n}{n^n}}$ e com ela podemos usar o critério da raíz. Assim, se $\displaystyle{\exists \lim_{n \rightarrow \infty} \sqrt[n]{b_n}}$ e esse limite estiver entre 0 e 1 ($0 \leq k < 1$), a $b_n$ e consequentemente, $a_n$ serão convergentes, caso contrário, serão divergentes. $$\lim_{n \rightarrow \infty} \sqrt[n]{\frac{e^n}{n^n}} = \lim_{n \rightarrow \infty} \ \frac{e}{n} = 0$$ Logo, as séries são convergentes. \\
\vspace{7mm}

\section{Convergência Absoluta --- Convergência Condicional}

\hspace{5mm} Lembrando que uma sequência $b_n$ é convergente, se e somente se, para um dado $\epsilon$ suficientemente pequeno, houver $N \in \mathds{N}$ tal que \begin{align*}
m,\ n \ \geq N \Rightarrow |b_m - b_n| &\leq \epsilon \\
m \ \geq N \Rightarrow |b_{m+k} - b_m| &\leq \epsilon & \forall \ k = 1,\ 2, 3 \ \hdots
\end{align*}

\begin{theorem}
Seja $\displaystyle{\sum_{n=0}^{\infty} a_n}$ uma série numérica. Se a série $\displaystyle{\sum_{n=0}^{\infty} |a_n|}$ convergir, a primeira série certamente será convergente também.
\end{theorem}

\begin{demonstration}
Sejam \begin{align*}
S_m &= a_0 + a_1 + a_2 + \hdots + a_m \\
S'_m &= |a_0| + |a_1| + |a_2| + \hdots + |a_m|
\end{align*}
Vamos tomar que $S'_m$ seja convergente, queremos provar que $S_m$ converge também.
Dado $\epsilon > 0$ e $S'_m$ convergente, então existe $N \in \mathds{N}$ tal que \begin{align}
m \ \geq N &\Rightarrow |S'_{m+k} - S'_m| \leq \epsilon \\
&= |a_{m+1}| + |a_{m+2}| + \ \hdots \ + |a_{m+k}| < \epsilon \label{212a}
\end{align}

Então, $m \geq N \Rightarrow |S_{m+k} - S_m| = |a_{m+1} +\hdots + a_{m+k}| < \\ < \underbrace{|a_{m+1}| + \hdots + |a_{m+k}| < \epsilon}_{\ref{212a}}$

Pelo critério de Couchy, a sequência $S_m$ é convergente.
\end{demonstration}

\textbf{Exemplo:} $$(*) a_n = \sum_{n=0}^{\infty} \ \frac{\cos n\pi}{n^2} = \sum_{n=0}^{\infty} \ \frac{(-1)^n}{n^2}$$ Em valor absoluto temos $\displaystyle{\sum_{n=1}^{\infty} \ \frac{1}{n^2}}$ que é convergente, logo $\displaystyle{\sum_{n=1}^{\infty} \ \frac{\cos(n\pi)}{n^2}}$ também o é.

\textbf{Considere} o seguinte exemplo: $$(**) \sum_{n=1}^{\infty} \ \frac{(-1)^n}{n}$$ Em valores absolutos temos $\displaystyle{\sum_{n=1}^{\infty} \ \frac{1}{n}}$ que é divergente, mas a série $(**)$ converge como veremos a seguir. 

\begin{definition} São tipos de convergência:
\begin{itemize}
\item Uma série $\sum a_n$ é dita \underline{absolutamente convergente} se $\sum a_n$ e $\sum |a_n|$ forem convergentes. Exemplo $(*)$
\item Uma série $\sum a_n$ é dita \underline{condicionalmente convergente} se $\sum a_n$ convergir, mas $\sum |a_n|$. divergir. Exemplo $(**)$
\end{itemize}
\end{definition}

Para dada $\displaystyle{\sum_{n=0}^{\infty} a_n, \ a_n \neq 0}$ e suponha que exista $$\lim_{n \rightarrow \infty} \ \frac{|a_{n+1}|}{|a_n|} = l$$

\begin{itemize}
\item [a] se $0 \leq l <1$, então $\displaystyle{\sum_{n=0}^{\infty} |a_n|}$ converge, e portanto, $\displaystyle{\sum_{n=0}^{\infty} a_n}$ converge também.
\item [b] se $l > 1$ então $\displaystyle{\sum_{n=0}^{\infty} |a_n|}$ diverge, e portanto, $\displaystyle{\sum_{n=0}^{\infty} a_n}$ também o é. Isso acontece pois se $$ \lim_{n \rightarrow \infty} |a_n| \neq 0 \Rightarrow \lim_{n \rightarrow \infty} a_n \neq 0 \Rightarrow \sum a_n \ \ diverge$$
\end{itemize}

\subsection{Critério de Liebniz}
\label{subsec:2121}

\hspace{5mm}Tomamos uma série $\displaystyle{\sum_{n=0}^{\infty} (-1)^n a_n}$ onde $a_n$ é uma sequência de termos positivos tais que: \begin{enumerate}
\item $a_n$ é decrescente ($a_n \geq a_{n+1}$)
\item $\displaystyle{\lim_{n \rightarrow \infty} a_n = 0}$
\end{enumerate}

Então $\displaystyle{\sum_{n=0}^{\infty} (-1)^n a_n}$ converge. Além disso, temos que \begin{align}
S_m &= \sum_{n=0}^m (-1)^n \ a_n \\
S &= \sum_{n=0}^{\infty} (-1)^n \ a_n \\
\lim_{n \rightarrow \infty} S_m &= S \\
|S-S_m| &\leq a_{m+1} & \forall m \geq 1
\end{align}
\textbf{Voltemos ao exemplo (**)} $ \ \displaystyle{a_n = \frac{1}{n}}$ e tomando dois inteiros tais que $i \leq j$ \begin{itemize}
\item $a_n$ é decrescente. $$ i \leq j \Longleftrightarrow \frac{1}{i} \geq \frac{1}{j} \Longleftrightarrow a_i \geq a_j $$ 
\item $a_n$ é uma série de números positivos
\item $\displaystyle{\lim_{n \rightarrow \infty} a_n = a}$
\item $\displaystyle{|S-S_{99}| = a_{99+1} = a_{100} = 0,01}$ \footnote{Eu não tenho certeza, mas acredito que essa propriedade é uma consequência, e não uma dependência da série ser convergente ou não}
\end{itemize}
$$\sum_{n=1}^\infty \ \frac{(-1)^n}{n} \ converge$$ 

\textbf{Fatos importantes sobre essa série:}\footnote{$n \in \mathds{N} \land n \geq 1$}
\begin{itemize}
\item $S_2 > S_4 > S_6 > \hdots > S_{2n}$
\item $S_1 < S_3 < S_5 < \hdots < S_{2n-1}$
\item Para $k$ par e $l$ ímpar, temos tempre que $S_k > S_l$
\end{itemize}

\begin{align}
S_{2n+2} &= -a_1 + a_2 - \hdots + a_{2n} - a_{2n-1} + a_{2n+2}\\
S_{2n+1} &= -a_1 + a_2 - \hdots - a_{2n-1} + a_{2n} - a_{2n+1}\\
S_{2n+2} - S_{2n} &= -a_{2n+1} + a_{2n+2} \leq 0 \Longleftrightarrow S_{2n+2} \leq S_{2n} \label{212A}\\
S_{2n+1} - S_{2n-1} &= a_{2n} - a_{2n+1} \geq 0 \Longleftrightarrow S_{2n+1} \geq S_{2n-1} \label{212B}\\
S_{2n} - S_{2n-1} &= a_{2n} > 0 \label{212C} \\
l \leq 2n-1 &\land k < 2n
\end{align}

Podemos concluir que $S_l \underbrace{<}_{\ref{212B}} S_{2n-1} \overbrace{<}^{\ref{212C}} S_{2n} \underbrace{<}_{\ref{212A}} S_k$

\begin{obs}{Observações}
\begin{itemize}
\item $S_{2n}$ é decrescente e é limitada inferiormente ($S_{2n} \geq S_1 \ \forall n$). \\
Logo $\exists \displaystyle{\lim_{n \rightarrow \infty} S_{2n} = \alpha = inf \ \left\{S_{2n} | \ n \in \mathds{N}\right\} }$
\item $S_{2n+1}$ é crescente e é limitada superiormente ($S_{2n+1} \leq S_2 \ \forall n$). \\
Logo $\exists \displaystyle{\lim_{n \rightarrow \infty} S_{2n+1} = \beta = sup \ \left\{S_{2n+1} | \ n \in \mathds{N}\right\} }$
\end{itemize}

Desejamos mostrar que $\alpha = \beta$. Para isso, temos que $S_{2n+2} - S_{2n+1} = a_{2n+2} > 0$. 
$$\lim_{n \rightarrow \infty} \overbrace{\left(S_{2n+2} - S_{2n+1}\right)}^{\alpha - \beta} = \lim_{n \rightarrow \infty} a_{2n+2} = 0$$
Logo $\alpha - \beta = 0 \Longleftrightarrow \alpha = \beta$
\end{obs}

\section{Exercícios Resolvidos}
 
Exercício B -- p. \pageref{avaliar}
\textbf{resolução item B}

\textbf{Passo 1:} Considere a função $\displaystyle{f(x) = \frac{1}{\ln n^{\ln(\ln n)}}}$. No intervalo de 2 até infinito teremos o seguinte ($a > 2$): \begin{align*}
\int_2^a f(x) \ dx = \int_2^a \ \frac{dx}{(\ln n)^{\ln(\ln n)}}
\Rightarrow \int_{\ln 2}^{\ln a} \frac{e^y \ dy}{y^{\ln y}} \label{ob:2-11}
\end{align*}

 Mudança de variáveis $(y = \ln x  \ \ \ x = e^y \ \ \ dx = e^y \ dy)$

\vspace{5mm}

\textbf{Passo 2:} Considere agora $g(y) = \frac{e^y}{y^{\ln y}}$ e a sua respectiva série 
\begin{align*}
b_n = g(n) &= \frac{e^n}{n^{\ln n}} = \frac{e^n}{e^{n^{\ln n}}} \\
&= \frac{e^n}{e^{(\ln n)^2}} = e^{n - (\ln n)^2}
\end{align*}
Como $\displaystyle{\LI n - (\ln n)^2 = \infty}$. Então $b_n$ diverge, e pelo critério da Integral, divergem as demais outras operações.

Exercício na lousa do dia (02/Set) -- avaliar a série $\soma{1} \ln \left( 1 + \frac{1}{n}\right)$

\begin{align*}
a_n &= \ln \left( \frac{n+1}{n} \right) = \ln (n+1) - \ln n \\
S_m &= \ln 2 - \underbrace{\ln 1}_{\ln 1 = 0} + \ln 3 - \ln 2 + \cdots + \ln (m+1) - \ln m = \ln (m+1)\\
\LI[m] S_m &= \LI[m] (\ln (m+1))
\end{align*}

Logo a série é divergente

\section{Teste seus conhecimentos}{Avalie se as séries abaixo são divergentes, convergentes absolutas, ou convergentes condicionais. Pergunte ao monitor, se tiver qualquer dúvida.}
%Ideia de subtítulo #1 (caso vc queira se aprofundar em LaTeX )\section{Section Title\\Section Subtitle}
%Ideia de subtítulo #2 \section[Section Title. Section Subtitle]{Section Title\\ {\large Section Subtitle}}
\begin{itemize}
\item [a.] $\displaystyle{\sum_{n=1}^{\infty} \ \frac{\sin(n\theta)}{n^2}, \ \theta \in \mathds{R}}$
\item [b.] $\displaystyle{1-\frac{1}{3}+\frac{1}{5}-\frac{1}{7}+ \frac{1}{9} + \hdots = \sum_{n=0}^{\infty} \ \frac{(-1)^n}{2n+1}}$
\item [c.] $\displaystyle{1-\frac{1}{2}+\frac{2}{3}-\frac{1}{3}+\frac{2}{4}-\frac{1}{4}+\frac{2}{5}-\frac{1}{5}+\frac{2}{6}}-\frac{1}{6}+ \hdots$
\item [d.] $\displaystyle{\sum_{n=1}^{\infty} (-1)^n \ \frac{\ln n}{n}}$
\item [e.] $\displaystyle{\sum_{n=1}^{\infty} \ \frac{(-1)^n}{\ln n}}$
\item [f.] $\displaystyle{\sum_{n=1}^{\infty} \ \frac{(-1)^{n+1}}{\sqrt{n}}}$
\item [g.] $\displaystyle{1+\frac{1}{2}-\frac{1}{4}-\frac{1}{8}+\frac{1}{16}+\frac{1}{32}-\frac{1}{64}-\frac{1}{128}+\hdots}$
\end{itemize}

\chapter{Sequências de Funções} Assunto tratado entre 31/08/2015 até 02/09/2015
\label{chap:c3}

\section{Definições}

\begin{definition} Uma \textbf{sequência de funções} é uma sequência $n \rightarrow f_n$ onde cada $f_n$ é uma função definida em um mesmo conjunto $A \subset \mathds{R}$ Exemplos: \begin{enumerate}
\item $f_n: \mathds{R} \rightarrow \mathds{R} \ | \ f_n(x) = x^n, \ n \geq 1$ \label{ex:302a}
\item $f_n: \mathds{R} \rightarrow \mathds{R} \ | \ f_n(x) = n\ \displaystyle{\sin \left( \frac{x}{n}\right)}$ \label{ex:302b}
\end{enumerate}
\end{definition}

Seja $f_n$ uma sequência definida em $A \in \mathds{R}$. Seja B o subconjunto de A, tal que a seuqência numérica ($f_n(x)$) seja convergente. $B = \left\{x \in A\ | \ \exists \lim_{n \rightarrow \infty} f_n(x) \right\}$

Podemos definir: $f: B \rightarrow \mathds{R} \ | \ f(x) = \displaystyle{\lim_{n \rightarrow \infty} f_n(x)}$ Dado $\epsilon > 0, \ \exists n_0 \in \mathds{N}$ tal que $n \geq n_0 \Rightarrow |f_n(x) - f(x)| < \epsilon$

Voltemos ao exemplo \ref{ex:302a} onde: $f_n(x) = x^n, \ A = \mathds{R}$ e queremos definir o conjunto B.

\begin{itemize}
\item Se $|x| < 1$, então $f_n$ tende a 0.
\item Se $x=1$, $f_n$ tende a 1.
\item Se $x=1 \ \lor \ |x| > 1$: $f_n$ diverge.
\end{itemize}
Logo $B = ]-1;1]\ \land \hbox{f}(x)
= \left\{ \begin{array}{rll}
0 & \hbox{se} &  |x| < 1 \\
1 & \hbox{se} &  x  = 1 \\
\end{array}\right.$

Agora, nos voltamos ao exemplo \ref{ex:302b}
\begin{itemize}
\item se $x=0$, $f_n$ tende a 0
\item caso contrário, temos $$f(x) = \lim_{n \rightarrow \infty} n \sin \left(\frac{x}{n}\right) = \LI x \ \underbrace{\frac{\sin(\frac{x}{n})}{\frac{x}{n}}}_1 = x$$
\end{itemize}

\section*{Mais exemplos} Determine o domínio da função $f$ dada por $\displaystyle{f(x) = \LI f_n(x)}$

\begin{itemize}
\item [a] $$f_n= e^{nx}$$ 
 Não existe $\displaystyle{ \LI f_n(x) = k \ | \ k \neq \infty}$ Logo não existe $f(x)$ nessas condições \ldots
 \item [b] $$f_n= \frac{nx}{1+nx^2}$$
 se $x=0 \Rightarrow f_n(0) = 0 \rightarrow f(0) = 0$, para $x \neq 0$ teremos $\displaystyle{\LI \ \frac{nx}{1+nx^2} = \LI \ \frac{x}{\frac{1}{n}+x^2} = \frac{x}{x^2} \rightarrow f(x) = \frac{1}{x}}$. O seu domínio é $D = \mathds{R} \left\{0\right\}$
 \item [c] $$f_n(x) = \ \left( 1 + \frac{1}{n} \right)^{nx} \rightarrow \LI \ \left[ \left( 1 + \frac{1}{n} \right)^n \right]^x \rightarrow f(x) = e^x$$
\end{itemize}

\begin{definition}[Convergência Uniforme]
Dadas ($f_n$) e $f$ definidas em $A \subset \mathds{R}$, dizemos que $f_n$ \underline{converge unformemente} para $f$ em $A$ se para dado $\epsilon > 0, \exists \ n_0 \in \mathds{N} \ | \ n \geq n_0 \Rightarrow |f_n(x) - f(x)| < \epsilon$
\end{definition}

Voltamos para o exemplo 1 \footnote{p. \pageref{ex:302a}} com $x \in [-\frac{1}{2}; \ \frac{1}{2}]$. Sabemos que essa série converge a 0 neste intervalo. Neste caso, verificaremos que a convergência é uniforme.

Dado $\epsilon > 0, \ \exists \ n_0 \ | \ n \geq n_0 \Rightarrow |x^n-0| < \epsilon \ \forall \ x \in [-\frac{1}{2};\frac{1}{2}]$.

Tomamos que $\displaystyle{|x| < \frac{1}{2} \Rightarrow |x|^n < \frac{1}{2^n} <\epsilon}$. Tomando $\displaystyle{\frac{1}{2^{n_0}} < \epsilon}$ Temos  que para: $$n \geq n_0 \Rightarrow |x|^n < \frac{1}{2^n} < \frac{1}{2^{n_0}} < \epsilon$$

\section*{Exercício:} Determine para que valores de $x \in \mathds{R}$ as funções a seguir estão definidas:
\begin{itemize}
\item [a.] $\displaystyle{\soma{1} \ \frac{n! x^n}{3^n}}$

Tomamos $a_n = \soma{1} \frac{n! x^n}{3^n}$ 
\begin{align*}
\LI \frac{a_{n+1}}{a_n} &= \LI \frac{(n+1)! x^{n+1}}{3^{n+1}}\ \frac{3^n}{n! x^n} \\
&= \LI \frac{x}{3} (n+1) = \frac{x}{3} \LI (n+1) = \infty
\end{align*}
Logo não há nuémros reais para os quais essa função está definida.
\item [b.] $\displaystyle{\soma{1} \ ne^{-xn}}$
\begin{align*}
a_n &= n \ e^{-xn} \\
&= \frac{n}{e^{xn}} \\
\LI \sqrt[n]{\frac{n}{e^{xn}}} &= \LI \frac{\overbrace{\sqrt[n]{n}}^{=1 \footnote{\ref{subsec:ex223}}}}{\sqrt[n]{e^{xn}}} \\
f(x) = \frac{1}{e^x}
\end{align*}
Logo essa função está definida para qualquer valor estritamente positivo de $x$
\end{itemize}

\section{Continuidade, diferenciabilidade e Integrabilidade de uma uma função dada como o limite de uma sequência de funções}

\begin{theorem}[Teorema 1\label{t:3-1}] 
Seja $f_n: B \rightarrow \mathds{R}$ uma sequência de funções e $f: B \rightarrow \mathds{R}$ definida como  $$f(x) = \LI f_n(x)$$ \hspace{5mm}Vamos supor agora que cada $f_n$ seja contínua em B, e que a convergência é uniforme neste universo  Neste caso, $f$ é contínua em B.
\end{theorem}

\begin{theorem}[Teorema 2\label{t:3-2}] 
Seja $f: \ [a;b] \rightarrow \mathds{R}$ dada por $f(x) = \LI f_n (x)$ onde cada $f_n$ é contínua em todos os pontos do domínio da função. Se $f_n$ convergir uniformemente a $f$ no domínio, então: \begin{align*}
\int_b^a f(x) \ dx &= \LI \int_a^b f_n(x) \ dx \\
&= \int_b^a \left( \LI f_n(x) \right) \ dz  \\ 
&= \LI \left[\int_a^b f_n (x) \right] \\
\end{align*}
\end{theorem}

\begin{theorem}[Teorema 3\label{t:3-3}] 
Seja ($f_n$) uma sequência de funções de classe $C_1$ no intervalo $I \subset \mathds{R}$. Sendo $f$ e $g$ funções definidas em $I$ tal  que \begin{align*}
f(x) = \LI f_n (x) \\
g(x) = \LI f'_n(x)
\end{align*}
Se $f'_n$ converge uniformemente para $g$, então $$f'(x) = g(x) \Longleftrightarrow \left( \LI f_n(x)\right)' = \LI f'_n(x)$$
\end{theorem}

\chapter{Série de funções} {Assunto tratado entre 16/09/2015 até 23/09/2015}
\label{chap:c4}
\section{Definição}
\hspace{5mm} Uma série de funções é uma série $\displaystyle{\soma{0} f_n}$ onde cada $f_n$ é uma função definida em um mesmo conjunto $B \in \mathds{R}$. Dizemos que a série $\sum f_n$ converge para a função $s: B \rightarrow \mathds{R}$, se para todo o $x \in B, \ s(x) = \soma{0} f_n(x)$. É o mesmo que dizer que $$\LI[m] \sum_{m=0}^n f_n(x) = s(x) \ \ \  \forall \ x \in B$$
 Para $x > 0$\begin{align*}
 s(x) &= \soma{0} \frac{n}{e^{xn}}\\
 a_n &= {n \over e^{xn}}\\
 \LI \sqrt[n]{a_n} &= \LI {\sqrt[n]{n} \over e^x} = {1 \over e^x} <1
 \end{align*}

A série de funções $\displaystyle{\soma{0} f_n}$ converge uniformemente em B para a função $s: B \rightarrow \mathds{R}$ se dado $\epsilon > 0, \ \exists n_0 \in \mathds{N}$ tal que, para todo $x \in \mathds{B}$ tivermos que $n\geq n_0 \Rightarrow \displaystyle{\left|\sum_{k = 0}^n f_k (x) - s(x)\right| < \epsilon}$. Isso significa que a sequência $s_n = \displaystyle{\sum_{k=0}^n f_k}$ converge unifirmemente para $s$

\section{Critério de Cauchy} {Para convergência uniforme de uma série de funções}

\hspace{5mm} A série de funções $\displaystyle{\soma{0} f_n}$ converge uniformemente em $B$ para a função $\displaystyle{f(x) = \soma{0} f_n(x)}$ se, e somente se, para dado $\epsilon > 0, \ \exists n_0 \in \mathds{N}$ tal que \begin{align*}
m>n \geq n_0 \ &\Rightarrow \left| \sum_{k=0}^m f_k (x) - \sum_{k=0}^n f_k (x)\right|  \Longleftrightarrow & \forall x \in B \\
\Longleftrightarrow m > n \geq n_0 &\Rightarrow |s_m(x) - s_n(x)| < \epsilon
\end{align*}

\section{Critério M de Weierstrass} {Para convergência uniforme de uma série de funções}

Seja $\displaystyle{\soma{0} f_n}$ uma série de funções definidas em $b \subseteq \mathds{R}$ e suponha que exista uma \underline{série numérica} $\displaystyle{\soma[k]{0} M_k}$ tal que \begin{itemize}
\item $\forall \ x \in B \land \ k \in \mathds{N} \rightarrow \ |f_k(x)| \leq M_k$
\item $\displaystyle{\soma[k]{0} M_k}$ é convergente
\end{itemize}

Nestas condições, a série $\displaystyle{\soma{0} f_n}$ converge uniformemente para a função $\displaystyle{s(x) = \soma{0} f_n(x)}$ (\textbf{Demonstração} na foto do link).

\section*{Exemplo 1} {Considere $\displaystyle{\soma{0} {\sin kx \over x^4 + k^4} \ (*) \ x \in \mathds{R}}$}
$$f_k(x) = {\sin kx \over x^4 + k^4} \Rightarrow |f_k(x)| = {|\sin kx| \over x^4 + k^4} \leq {1 \over x^4 + k^4} \leq {1 \over k^4} = M_k$$ Como $\displaystyle{\soma{1} {1 \over k^4}}$ converge, então a série em questão também o faz.

\section*{Exemplo 2}{Faça o mesmo com a série $\displaystyle{\soma{1} {1 \over x^2 + k^2}}$ e justifique a igualdade $$\int_0^1 s(x) \ dx = \soma{1} {1 \over k} \arctan \left( {1 \over k} \right)$$} Tomamos $\displaystyle{f_k(x) = {1 \over x^2 + k^2} \leq {1 \over k^2} = M_k}$. Logo, pelo critério M de Weierstrass, a série converge uniformemente para $s(x) = \displaystyle{\soma{1} {1 \over x^2 + k^2}}$ \begin{align*}
\int_0^1 s(x) \ dx &= \int_0^1 (\LI s_n (x)) \ dx = \LI \int_0^1 s(x) \ dx = \LI \int_0^1 \left(\soma{1} {dx \over x^2 + k^2}\right) \\
&= \LI \left( \sum_{k=1}^n \int_0^1 {dx \over x^2 + k^2}\right) = \LI \sum_{k=1}^n \left[ x \arctan {x \over k^2} - {k^2 \over 2} \ln (1+k^2) \right]_0^1\\
\int \underbrace{\arctan x}_{u} \underbrace{dx}_{dv} &= x \arctan x - \int {x \over 1 + x^2} \ dx \\
&= x \arctan x - {1 \over 2} \int {2x \over 1 + x^2} \ dx \\
&= x \arctan x - {1 \over 2} (\ln{1 + x^2}) \ dx \\
\int {dx \over x^2 + k^2} &= {1 \over k^2} \int \frac{dx}{1 + \frac{x^2}{k^2}} = {1 \over k^2} \int \frac{k \ dy}{1 + y^2} = \frac{1}{k} \arctan{y} = \frac{1}{k} \arctan{\frac{x^2}{k^2}}
\end{align*}

 \section{Continuidade, Integrabilidade e Diferenciabilidade} {de uma função dada como soma de uma série de funções}
 
\begin{theorem}[Teorema 1 -- Continuidade] \label{T1 C4}
 Seja $\defini{s}{B}$ dada por $s(x) = \displaystyle{\soma{1} f_k(x)} $. Se $\sum f_k$ convergir uniformemente para $s$ em $B$ e se cada $f_k$ for contínua, então $s$ é contínua.\footnote{\textbf{Pesquise:} sendo $s(x) = \soma{N} f_n$. Se s(x) for contínua, então $f_n$ também o será?}
 \end{theorem}
 
\begin{theorem}[Teorema 2 -- Integração termo a termo] \label{T2 C4}
Seja $\defini{s}{[a,b]}$ dada por $s(x) = \displaystyle{\soma{1} f_k(x)}$. Se cada $f_i$ for contínua em $[a,b]$ e se a série $\sum f_n$ convergir uniformemente a $s$ em $[a,b]$, então: \begin{align*}
\int_a^b  s(x) \ dx &= \soma{o} \int_a^b  f_n(x) \ dx  \Longleftrightarrow\\
\int_a^b \left( \sum f_k(x)\right) \ dx &= \sum \int_a^b  f_n(x) \ dx \\
\end{align*}

\end{theorem}

\begin{theorem}[Teorema 3 -- Derivação termo a termo] \label{T3 C4}
Seja $\defini{s}{I}$ dada por $s(x) = \displaystyle{\soma{1} f_k(x)} $. Se cada $f_k$ for de classe $C^1$ em $I$ e se a série $\sum f_n$ convergir uniformemente a $s$ em $I$, então: \begin{align*}
s'(x) &= \soma[k]{0} f'_k(x) = \left( \soma[k]{0} f_k(x)\right)'
\end{align*}
\end{theorem}

\section{Testando o que vimos}

\subsection*{\E{1}} {Prove que a função dada é contínua nos subconjuntos de $\mathds{R}$} 

\begin{align*}
(a) \ \ \ f(x) &= \soma{1} {\cos nx^2 \over n^4} & B = \mathds{R}\\
(b) \ \ \ f(x) &= \soma{0} {1 \over 2^{nx}} & B = [0; +\infty)\\
(c) \ \ \ f(x) &= \soma{0} {2^n x^n \over 2^{nx}} & B = \mathds{R} \\
\end{align*}

\subsection*{\E{2}}{Sendo $\displaystyle{f(x) = \soma{1} \sin \left( {x \over n^2}\right)}$ justifique a igualdade $$\int_0^1 f(x) \ dx = \soma{1} n^2 \left(1 -\cos {1 \over n^2}\right)$$

\textbf{Solução:} Para $x \in [0,1]$ temos que $$\sin {x \over n^2} < {x \over n^2} \leq {1 \over n^2}$$ Pelo critério M de Weierstrass, a série é convergente uniformemente em $[0,1]$. Podemos aplicar o teorema \ref{T2 C4} \footnote{\textbf{Bom saber} $\int \sin{x \over n^2}\ dx = n^2 \left(-\cos{x \over n^2}\right)$}. \begin{align*}
\int_0^1 f(x) \ dx = \int_0^1 \left( \soma{1} \sin {x \over n^2} \right) \ dx = \soma{1} \int_0^1 \left( \sin {x \over n^2} \right) \ dx = \soma{1} \left[ -n^2 \cos{x \over n^2} \right]_{x = 0}^{x = 1}\\
\soma{1} \left( -n^2\cos{1 \over n^2} +n^2 \right)
\end{align*}

\subsection*{\E{3}}{Seja $\displaystyle{f(x) = \soma{0}  {x^n \over n!} \ \ (0! = 1)}$ Verifique $$\int_0^x f(t) \ dt = f(x) -1 \ \ \ \forall \ x \in \mathds{R}$$}

\textbf{Resolução:} Fixe $x \in \mathds{R}$ e seja $r > 0$ tal que $|x| \leq r \Longleftrightarrow x \in [-r,r]$. Nesse intervalo, a série em questão converge uniformemente. $\forall \ x \in [-r, r]$ temos: \begin{align*}
\left| \frac{x^n}{n!} \right| = {|x|^n \over n!} &\leq {r^n \over n!} \\
\LI {r^{n+1} \over (n+1)!}*{n! \over r^n} = \LI {r \over n+1} = 0
\end{align*} 
Logo $\DS{\sum {r^n \over n!}}$ é convergente. Pelo critério M de Weierstrass, a série converge uniformemente em $[-r,r]$. Como x está nesse  intervalo,, a série converge uniformemente em $[0,x]$

\subsection*{\E{4}}{Sendo $\displaystyle{f(x) = \soma{1}  {x^6 \over x^2+n^2}}$, justifique a igualdade $$\int_0^1 f(x) \ dx = \frac{1}{2}\soma{1}\ln{1+ \frac{1}{n^2}}$$}

\subsection*{\E{5}}{Seja $\displaystyle{f(x) = \soma{1} \arctan {x \over n^2}}$ Mostre que a série numérica $$\soma{1} \left[ \arctan{1 \over n^2} - \frac{n^2}{2} \ln \left(1+\frac{1}{n^4}\right) \right]$$ tem por soma $\displaystyle{\int_0^1 f(x) \ dx}$ 

\textbf{Resolução:} Reparar que $\forall \ x > 0, \ \arctan{x} < x$. Então $\forall \ x \in [0,1] \land \ n \in \mathds{N}$ temos que $\arctan{x \over n^2} \leq {x \over n^2} \leq {1 \over n^2}$.\vspace{2mm}


Como $\sum{x \over n^2}$ onverge, então, pelo \cmw a série converge unifrmmente em$[0,1]$

\begin{align*}
\int_0^1 \ f(x) \ dx = \soma{1} \int_0^1 \left( \arctan{x \over n^2}\right)dx \\
\int \underbrace{\arctan{x}}_{u}\underbrace{dx}_{dv} &= x\arctan{x} - \int {x \over 1+x^2}dx \\
&= x \arctan{x} - \frac{1}{2}\int {2x \over 1+x^2}dx\\
&= x \arctan{x} - \frac{\ln{(1+x^2)}}{2}\\
\int \arctan{x \over n^2} &= n^2 \left[ {x \over n^2}\arctan{x \over n^2} - \frac{1}{2}\ln{\left(1 + {x^2 \over n^4}\right)}\right]\\
\int_0^1 \ f(x) \ dx = \soma{1} \int_0^1 \left( \arctan{x \over n^2}\right)dx &= \soma{1} \left[x \arctan{x \over n^2} - \frac{n^2}{2}\ln{\left(1 + {x^2 \over n^4}\right)} \right]_0^1 \\
&= \soma{1} \left[\arctan{1 \over n^2} - \frac{n^2}{2}\ln{\left(1 + {1 \over n^4}\right)} \right]
\end{align*}
\footnote{Perguntar o pq do $n^2$}

\chapter{Séries de Potências}{Aulas dias 23/09/2015 - 05/10/2015}
\label{chap:c5}

\section{Definições}
Sejam $(a_n)$ uma sequência, e $x_0 \in \mathds{R}$, a série de funções $$\soma{0} a_n (x-x_0)^n$$ chama-se \textbf{série de potências com coeficientes $a_n$ e centralizada em $x_0$} 
\begin{align*}
f_n(x) = a_n (x-x_0)^n & & x_0 =1\\
\soma{0} {(x-1)^n \over n!} & & a_n = \frac{1}{n!} &  & f_n(x) = {(x-1)^n \over n!}
\end{align*}

\begin{theorem}[Teorema 1]\label{T:5-1}
Se $\displaystyle{\soma{0} a_n x^n}$ for convergente em algum $x \neq 0$, então a série convergirá em qualquer  intervalo  $]-|x|, |x|[$
\end{theorem}

\section*{\textbf{Exemplo:}}{ Verificar que a série $\displaystyle{\soma{0} 2^{nx}}$ converge para $x = -1$ e para $x \in \ ]-1,1]$}

Para $x = -1$ temos $$\soma{0} 2^{-n} = \soma{0} \frac{1}{2^n}$$ Usando o critério da razão ou da raíz (que eu vou escrever abaixo), podemos chegar em um resultado que nos confirma isso. $$\LI \sqrt[n]{\frac{1}{2^n}} = \LI \frac{1}{2} = \frac{1}{2} < 1$$ Pelo Teorema \ref{T:5-1},a série converge para $x \in ]-1,1[$. Se adotarmos $x= 1$, por outro lado, teremos que o limite da série para $n \rightarrow \infty$ é infinito.

\begin{demonstration}
 Se $\displaystyle{\soma{0} a_n x^n}$ converge, então $\displaystyle{\LI a_n  x_1^2 = 0}$
 
 Tomando $\epsilon = 1, \  \exists n_0 \in \mathds{N} \ | \ n \geq n_0 \Rightarrow |a_n x_1^n| \leq 1$. Como $|a_n x_1^n|  = |a_n x_1^n| \left(\frac{|x|}{|x_1|}\right)^n \leq \left|\frac{x}{x_1}\right|^n$. $$ |x| < |x_1| \Longleftrightarrow \left|\frac{x}{x_1}\right| < 1 \Rightarrow \soma{0} \left|\frac{x}{x_1}\right|^n $$ converge $\forall x \ | \ |x| <|x_1|$. Pelo critério da comparação, temos que $\sum a_n x^n$ é absolutamente convergente para esse mesmo intervalo.
\end{demonstration}

\hspace{5mm}\textbf{Exemplo:} $\displaystyle{\soma{1} \frac{x^n}{n}}$ converge $x = -1$. Logo, pelo critério de Liebniz, a série converge absolutamente $\forall x \in ]-1,1[$

\begin{theorem}[Teorema 2]\label{T:5-2}
Seja  $\displaystyle{\soma{0} a_n x^n}$ uma série de potências. Então vale um desses cenários: \begin{itemize}
\item [(a)] a série converge apenas para $x=0$
\item [(b)] a série converge abolutamente $\forall \ x \in \mathds{R}$
\item [(c)] $\exists R > 0$ real tal que a série converge abolutamente $\forall \ x \in ]-R, R[$. Nos extremos, pode ou não convergir.
\end{itemize}

\end{theorem}

$$A = \{x\geq 0 \ | \ \sum a_n x^n \ converge\} \subset [0, +\infty)$$
\begin{itemize}
\item [a.] $A = \{0\}$
\item [b.] $A = [0, +\infty)$
\item [c.] $A \neq \{0\} \land A \neq [0, +\infty)$
\end{itemize}

O número $R$ da parte (c) é chamado de \textbf{raio de convergência da série}. \begin{itemize}
\item se acontecer (a), dizemos que o raio de convergência é zero
\item se acontecer (b), dizemos que o raio de convergência é $\infty$
\end{itemize}

Como calcular o raio de convergência? Seja $\soma{0}a_n x^n$ com $a+n \neq 0$ a partir de um certo $p \in \mathds{N}$. Então $$ R = \LI \frac{|a_n|}{|a_{n+1}|} \  \lor \ \LI \frac{|a_{n+1}|}{|a_n|} = {1 \over R} = \LI \sqrt[n]{|a_n|} $$

\textbf{Resumindo ...} Dada uma série $\displaystyle{\soma{0} a_n x^n}$, 3 coisas podem acontecer. \begin{itemize}
\item Converge para $x = 0$ quando $R = 0$
\item Converge absolutamente para qualquer x real ($R = 0$)
\item $ \exists \ R > 0$ tal que a série converge para $x \in \ ]-R,R[$ e diverge para $ x \in \ ]-\infty,-R[\ \cup \ ]R,+\infty[$ \footnote{Quando $x = \pm R$, a série pode ser convergente ou não}
\end{itemize}

\textbf{Exemplo:} Considere a série $\displaystyle{\soma{0} {x^n \over \sqrt[n]{\left(\frac{5}{3}\right)^n + 2^n}}}$ \vspace{7mm}

\begin{align*}
\sqrt[n]{a_n} &= \frac{1}{\sqrt[n]{\left(\frac{5}{3}\right)^n + 2^n}} \\
0 &< a \leq b \\
\LI \sqrt[n]{a^n + b^n} &= b \\
\LI \frac{1}{\sqrt[n]{\left(\frac{5}{3}\right)^n + 2^n}} &= \frac{1}{2} = \frac{1}{R} \\
R = 2
\end{align*}

\begin{itemize}
\item Converge para $|x| < 2$
\item Diverge para $|x| > 2$
\item Para $x = 2$, a série diverge como vemos abaixo $$\LI \frac{2^n}{\sqrt[n]{\left(\frac{5}{3}\right)^n + 2^n}} = {\infty \over 2} $$
\item Para $x = -2$, a série diverge pelo critério de Liebniz \footnote{Não consigo dizer ao certo, se a série é decresente, mas com certeza $\LI a_n \neq 0$}
\end{itemize}

\section*{Exercícios: \footnote{Essas são as minhas resoluções, e eu sou só um aluno, eu posso estar errado}}
\begin{multicols}{3}
\begin{itemize}
\item [a.] $f(x) = \displaystyle{\soma{0} \frac{x^n}{n+2}}$
\item [b.] $f(x) = \displaystyle{\soma{0} \frac{x^n}{n^2+3}}$
\item [c.] $f(x) = \displaystyle{\soma{1} \frac{x^n}{n^n}}$
\item [d.] $f(x) = \displaystyle{\soma{0} 2^n x^n}$
\item [e.] $f(x) = \displaystyle{\soma{0} n^n x^n}$
\item [f.] $f(x) = \displaystyle{\soma{0} n! x^n}$
\end{itemize}
\end{multicols}

\Resolve

\textbf{Item A
} $a_n = {1 \over n+2}$
\begin{align*}
R = \LI \frac{(n+2)+1}{n+2} = \LI \frac{n+3}{n+2} = \LI \frac{1 + \frac{3}{n}}{1 + \frac{2}{n}} = 1
\end{align*}

Assim, descobrimos que o provável raio de convergência é 1, logo a série converge para todo $x \in ]-1,1[$. se tomarmos $x = -1$, teremos uma série $a_n = {(-1)^n \over n + 2}$ que converge pelo critério de Liebniz. Como a série $\displaystyle{\soma{0} \frac{(1)^n}{n+2}}$ não converge, então, temos que $D_f = [-1,1[$ \vspace{5mm}

\textbf{Item B} $a_n = {1 \over n^2+3}$
\begin{align*}
R = \LI \frac{(n+1)^2+3}{n^2+3} = \LI \frac{n^2 + 2n +4}{n^2+3} = \LI \frac{1 + \frac{2}{n} + \frac{4}{n^2}}{1 + \frac{4}{n^2}} = 1
\end{align*}

O raio de convergência é o mesmo (1). Para $x =-1$ teremos novamente uma série que converge pelo critério de Liebniz, e para $x = 1$, temos $a_n = \frac{1}{n^2+3} \leq \frac{1}{n^2}$ que é convergente. Logo $D_f = [-1,1]$ \vspace{5mm}

\textbf{Item C} $a_n = {1 \over n^n}$
\begin{align*}
{1 \over R} = \LI \sqrt[n]{1 \over n^n} = \LI \frac{1}{n} = 0
\end{align*}

Como $R$ tende ao infinito, $D_f = \mathds{R}$ \vspace{5mm}

\textbf{Item D} $a_n = 2^n$ \vspace{5mm}

\begin{align*}
R = \LI \frac{a_n}{a_{n+1}} = \LI \frac{2^n}{2^{n+1}} = \frac{1}{2}
\end{align*}

Analisando os extremos temos que: \begin{itemize}
\item se $\displaystyle{x = -\frac{1}{2} \Longleftrightarrow a_n = \soma{0} (-1)^n}$ (divergente)
\item se $\displaystyle{x = \frac{1}{2} \Longleftrightarrow a_n = \soma{0} 1^n}$ (divergente)
\end{itemize}

Logo $D_f = ]-\frac{1}{2},\frac{1}{2}[$ \vspace{5mm}

\textbf{Item E} $a_n = n^n$
\begin{align*}
\frac{1}{R} = \LI \sqrt[n]{n^n} = \LI n = \infty \Longleftrightarrow R = 0
\end{align*}

Logo, $D_f = \{0\}$ \vspace{5mm}

\textbf{Item F} $a_n = n!$
\begin{align*}
R = \LI \frac{n!}{(n+1)!} = \LI \frac{1}{n+1} = 0
\end{align*}

Logo, $D_f = \{0\}$

\section{Propriedades de uma função dada como uma série de potências}{Continuidade, Integrabilidade e diferenciabilidade}
$$\frac{1}{1-x}= 1 + x +x^2 + x^3 + \cdots \hspace{10mm} |x| < 1 $$
\hspace{5mm}Seja $\displaystyle{\soma{0} a^n x^n}$ uma série com raio de convergência $R > 0$ (pode ser infinito ou não). Uma série converge uniformemente em todo o intervalo $[-r,r]$ onde $0 < r < R$. Seja um $r$ tal que\footnote{\cmw}: $$r \in [0,R] \Rightarrow \soma{0}|a_n|r^n \ converge$$

\begin{theorem}[Continuidade de Séries de Potências]
Seja a série de potências $\displaystyle{\soma{0} a_n x^n}$ com raio de convergência $R > 0$, então a função $f(x) = \displaystyle{\soma{0} a_n x^n}$ é contínua em $]-R,R[$
\end{theorem}

\begin{corollary}[Integrabilidade -- Diferenciabilidade]
Sendo a função dada no teorema anterior, que é contínua em $\mathds{A} = \{]-R,R[\}$, então, se $x \in \mathds{A}$, então vale que \begin{align*}
\int_0^t f(x) \ dx &= \soma{0} \int_0^t a_n x^n = \soma{0} a_n \frac{t^{n+1}}{n+1} \\
f'(x) &= \soma{1} n a_n x^{n-1}
\end{align*}
\end{corollary}

\section{Funções dadas por uma série de potências}
\hspace{5mm} Procuramos funções que podem ser representadas como séries de potências. A partir da fórmula da soma de uma progressão geométrica de razão $r \ | \ |r| < 1$, podemos afirmar que: \begin{align*}
\frac{1}{1-\underbrace{(-x^2)}_q} &= \frac{1}{1+x^2} = 1 - x^2 + x^4 - x^6 + \cdots \\
\arctan x &= \int_0^x \frac{dx}{1+x^2} = x - \frac{x^3}{3} + \frac{x^5}{5} - \frac{x^7}{7} + \cdots
\end{align*}

\textbf{Importante:} Independentemente que se derive ou integre uma série de potências, o raio de convergência permanece inalterado.

Levando em conta uma função $f(x)$ dada por $\soma{0} a_n x^n$. Podemos dizer que a função é infinitamente diferenciável em $]-R,R[$. Além disso \begin{align*}
f(0) = a_0 \\
f'(0) = 1*a_1 \\
f''(0) = 2*1*a_2 \\
f'''(0) = 3*2*1*a_3 \\
f^{(n)}(0) = n!\ a_n \Longleftrightarrow a_n = \frac{f^{(n)}(0)}{n!}
\end{align*}

Se $f$ for de classe $C^\infty$, então sua série de Taylor é $\displaystyle{\soma{0} \frac{f^{(n)} (0)}{n!}}$.

Conforme já foi visto, se a função $f$ pode ser descrita como uma soma de séries de potências, então $f(x)$ é a soma de sua série de Taylor.

Seja $\displaystyle{\soma{0} a_n x^n}$ uma série de potências com raio de convergência $R > 0$. Se houver $r$ tal que $0 < r < R \land f(x) = 0 \ \forall x \in ]-r,r[ \subset ]-R,R[$ então $f \equiv 0$.

Para todo $x, \ x+h \in ]-r,r[$ temos que \begin{align}
        & \ & a_0 = f(0) = 0 \\
        f'(x) = \LI[h] \frac{f(x+h)-f(x)}{h} = \LI[h] \frac{0-0}{h} = 0 & \ & a_1 = f'(0) = 0 \\
        f''(x) = \LI[h] \frac{f'(x+h)-f'(x)}{h} = \LI[h] \frac{0-0}{h} = 0 & \ & a_2 = \frac{f''(0)}{2!} = 0 \\
        f'''(x) = \LI[h] \frac{f''(x+h)-f''(x)}{h} = \LI[h] \frac{0-0}{h} = 0 & \ & a_3 = \frac{f'''(0)}{3!} = 0\\
        & \ & a_n = \frac{f^{(n)}(0)}{n!} = 0 \ \forall n
    \end{align}

Logo $\displaystyle{\soma{0} a_n x^n = 0 \ \forall x \in \ ]-R,R[}$

Se a função $f(x)$ for constante para um intervalo $]a,b[ \subset ]-R,R[$ então $f$ é constante para todo o intervalo de convergência.
\vspace{5mm}

\textbf{Definição 1:} Seja $f: I \rightarrow \mathds{R}$ uma função infinitamente diferenciável ($C^\infty$) em $I$. A Série de Taylor de $f$ em $a \in I$ é $$\soma{0} \frac{f^{(n)}(a)}{n!} (x-a)^n$$ 

\textbf{Definição 2:} Uma função é \textbf{analítica} em $a \in \mathds{R}$ se existe uma série de potências tal que $$f(x) = \soma{0} \frac{f^{(n)}(a)}{n!} (x-a)^n$$ para todo $x$ num intervalo centrado em $a$. Nesse caso, $a_n = \frac{f^{(n)}(a)}{n!}$. São exemplos de funções analíticas: \begin{align*}
\frac{1}{1-x} = 1 + x + x^2 + x^3 + \cdots & \ & |x| < 1 \\
\ln (1+x) = x - \frac{x^2}{2} + \frac{x^3}{3} - \frac{x^4}{4} + \cdots & \ & |x| < 1 \\
\arctan (x) = x - \frac{x^3}{3} + \frac{x^5}{5} - \frac{x^7}{7} + \cdots & \ & |x| < 1
\end{align*}

\begin{theorem}[Taylor]
Seja $f$ uma função $(n+1)$ vezes diferenciável em um intervalo $I \subset \mathds{R}$. Se $a \in I$ e $x \in I$, temos: \begin{align*}
    f(x) = f(a) + \frac{f'(a)}{1}(x-a) + \frac{f''(a)}{2}(x-a)^2 + \cdots + \frac{f^{(n)}(a)}{n}(x-a)^n + R_{n,a}(x) \\
    \lim_{x \rightarrow a} \frac{R_{n,a}(x)}{|x-a|^n}=0 \\
    R_{n,a}(x) = f(x) - \sum_{k=0}^n \frac{f^{(k)}(a)}{k!}(x-a)^k
    \end{align*}
\end{theorem}

\begin{corollary}
Seja $f$ uma função infinitamente diferenciável em um intervalo $I \subset \mathds{R}$. Se $a \in I$, logo a função é analítica se e somente se $\exists R > 0$ tal que $|x-a| < R \Rightarrow \LI R_{n,a} (x) = 0$
\end{corollary}

\textbf{Exemplo:} $f(x) = \sin x \ \ \ f(0) = 0 \ \ \ f'(0) = 1 \ \ \ f''(0) = 0$ 

Fórmula de Taylor: $\displaystyle{\sin x = x - \frac{x^3}{3} + \frac{x^5}{5} + \cdots \frac{(-1)^n x^{2n+1}}{(2n+1)!} + R_{2n+1,0}(x)}$ $$R_{2n+1, 0}(x) = \int_0^x \frac{\sin^{(2n+1)}t}{(2n+1)!}t^{2n+1}dt \land \LI R_{2n+1,0}(x) = 0 \ \forall \ x \in \mathds{R}$$ Logo $\displaystyle{\sin x = \soma{0} (-1)^n \frac{x^{2n+1}}{(2n+1)!}} \ \forall x \in \mathds{R}$

\section{Séries Binomiais}

Seja $\alpha \in \mathds{R}$, $\alpha \notin \mathds{N}$, então: \begin{align*}
(1+x)^\alpha &= 1 + \soma{1} \frac{\alpha(\alpha - 1) (\alpha - 2) (\alpha - 3) \cdots (\alpha -n + 1)}{n!} \ x^n \ (-1 < x < 1) \\
&= 1 + \soma{1} {\alpha \choose n} \ x^n
\end{align*}

\textbf{Exemplo:} \begin{align*}
    \frac{1}{\sqrt{1+x}} &= (1+x)^{-\frac{1}{2}} \\
    &= 1 + \soma{1} \frac{\left(-\frac{1}{2}\right) \left(-\frac{3}{2}\right) \left(-\frac{5}{2}\right) \cdots \left(-\frac{2n-1}{2}\right)}{n!} \ x^n \\
    &= 1 + \soma{1} \ \left(\frac{-1}{2}\right)^n \frac{1 * 3 * 5 * \cdots * (2n-1)}{n!} \ x^n \\
    &= 1 + \soma{1} \ (-1)^n \frac{1*3*5*\cdots*(2n-1)}{2*4*6*\cdots*(2n)} \ x^n
\end{align*}

\subsection*{\E 1} Desenvolva $\displaystyle{\frac{1}{1+x}}$ centralizado em $x=3$

\begin{align*}
    \frac{1}{1+x} = \frac{1}{1+3+x-3} &= \frac{1}{4 + (x-3)} = \frac{1}{4} \ \frac{1}{1 + \frac{(x-3)}{4}} = \frac{1}{4} \ \frac{1}{1+y} \\
    |y| &= \frac{|x-3|}{4} < 1 \Longleftrightarrow |x-3| < 4 \Longleftrightarrow -1 < x < 7 \\
    \frac{1}{1+y} &= 1 - y + y^2 - y^3 + \cdots \\
    &= 1 - \frac{x-3}{4} + \frac{(x-3)^2}{4^2} - \frac{(x-3)^3}{4^3} + \cdots \\
    &= \soma{0} (-1)^n \frac{(x-3)^n}{4^n} \\
    \frac{1}{1+x} &= \frac{1}{4} \frac{1}{1+y} \\
    &= \soma{0} (-1)^n \frac{(x-3)^n}{4^{n+1}}
\end{align*}

Nos extremos, temos \footnote{conferir com o professor}:\begin{itemize}
\item A série diverge (o limite do termo geral não é zero $$x =-1 \Rightarrow \soma{0}\frac{(-1)^n(-4)^n}{4^{n+1}} = \soma{0} \frac{4^n}{4^{n+1}} = \soma{0} \frac{1}{4} $$
\item A série também diverge (por Liebniz) $$x = 7 \Rightarrow \soma{0}\frac{(-1)^n 4^n}{4^{n+1}} = \soma{0} \frac{(-1)^n}{4} $$
\end{itemize}

$D_f = ]-1,7[$

\subsection*{\E 2} Dada a série de potências abaixo $$f(x) = \soma{0} (-1)^{n+1} (n+1)x^n$$ determine o raio de convergência e identifique a função.

$$a_n = (-1)^n (n+1) \Rightarrow \LI \frac{|a_n|}{|a_{n+1}|} = 1$$. Para $x \in ]-1,1[$, integrando termo a termo, temos \begin{align*}
    F(x) = \int_0^x f(t) \ dt &= \soma{0} (-1)^{n+1} \int_0^x (n+1) t^{n} \ dt \\
    &= \soma{0} (-1)^{n+1} x^{n+1} \\
    &= -x + x^2 - x^3 + x^4 - x^5 + x^6 - \cdots \\
    \frac{1}{1+x} &= 1 - x + x^2 - x^3 + x^4 - x^5 + x^6 - \cdots \\
    F(x) &= \frac{1}{1+x} - 1 = \frac{-x}{1+x} \\
    f(x) = F'(x) &= \frac{-(1+x)-1(-x)}{(1+x)^2} = \frac{-1}{(1+x)^2}
\end{align*}

\subsection*{\E 3} Desenvolver $\sqrt{1+x}$ em séries de potências \begin{align*}
    \sqrt{1 + x} &= (1+x)^{\frac{1}{2}} \ \ \ |x| < 1 \\
    &= 1 + \soma{1} \frac{\frac{1}{2}\left(\frac{1}{2}-1\right)\left(\frac{1}{2} - 2\right)\cdots \left(\frac{1}{2} - n + 1\right)}{n!}x^n \\
    &= 1 + \soma{1} \frac{\left(\frac{1}{2}\right)*\left(-\frac{1}{2}\right)*\left(-\frac{3}{2}\right)*\left(-\frac{5}{2}\right) \cdots \left(-\frac{2n+1}{2}\right)}{n!}x^n \\
    &= 1 + \soma{1} \frac{(-1)^{n-1}}{2^n} \frac{1*3*5\cdots(2n+1)}{n!}x^n \\
    &= 1 + \soma{1} (-1)^{n-1} \frac{1*3*5\cdots(2n+1)}{2*4*6\cdots(2n)} x^n
\end{align*}


\chapter{Séries de Fourier}{Aulas dias 05/10/2015 -- 19/10/2015}
\label{chap:c6}
\section{Observações -- Séries pares e séries ímpares}.

\hspace{5mm} Uma função é dita \underline{par} se $f(-x) = f(x)$, e \underline{ímpar} se $f(-x) = -f(x)$.

Temos também que para qualquer $a > 0$\begin{align*}
    f \ par &\Rightarrow \int_{-a}^{+a} f(x) \ dx = 2 \int_0^a f(x) \ dx \\
    f \ impar &\Rightarrow \int_{-a}^{+a} f(x) \ dx = 0 \\
\end{align*}

Esse material conta com uma sessão de referências, onde se fala um pouco de funções pares e ímpares. Para mais detalhes, veja a página \pageref{ref-3}

\section{As séries}

\hspace{5mm}Dados $(a_n) \ n \geq 0$ e $(b_n) \ n \geq 1$ duas sequências, consideremos as séries trigonométricas: $$\frac{a_0}{2} + \soma{1} [a_n \cos (nx)+ b_n\sin (nx)] \ \ x \in [-\pi, \pi]$$ 

Seja $f:[-\pi,\pi] \rightarrow \mathds{R}$ e sejam $a_n \ (n \geq 0)$ e $b_n \ (n \geq 1)$ tais que $$f(x) = \frac{a_0}{2} + \soma{1} [a_n\cos (nx) + b_n \sin (nx)] \ \ (*)$$ para $x \in [-\pi,\pi]$. Se $f$ é contínua e as séries forem convergentes, então \begin{align}
    a_0 &= \frac{1}{\pi}\int_{-\pi}^\pi f(x) \ dx \\
    a_n &= \frac{1}{\pi}\int_{-\pi}^\pi f(x)\cos (nx) \ dx \\
    b_n &= \frac{1}{\pi}\int_{-\pi}^\pi f(x)\sin (nx) \ dx
\end{align}

A exprressão $(*)$ é a \textbf{série de Fourier de $f(x)$} e $a_0, \ a_n, b_n$ são os \textbf{coeficientes de Fourier de $f$} 

Como a convergência é uniforme no dado intervalo, podemos integrar termo a termo. \begin{align*}
    \IP f(x) \ dx = \IP \frac{a_0}{2} \ dx + \soma{1} \left( a_n \IP \cos (nx) \ dx + b_n \IP \sin (nx) \ dx \right) \\
    \IP \sin (nx) \ dx = 0 \\
    \IP \cos (nx) = \left. \frac{\sin (nx)}{n}\right|_{-\pi}^\pi = 0 \\
    \IP f(x) \ dx = \IP \frac{a_0}{2} \ dx = \pi a_0 \\
    a_0 = \frac{1}{\pi}\IP f(x) \ dx
\end{align*}

O desenvolvimento dos outros termos da série estão nas fotos das aulas do dia 05/10

\textbf{Mudando um pouco de assunto ...} Uma função $f:[-\pi,\pi] \rightarrow \mathds{R}$ é de classe $C^2$ se existirem $f'(x)$ e $f''(x)$ contínuas. 

Essa função será $C^2$ por partes se existir uma partição $K \subset \mathds{R}$ e uma função $f_l(x):K \rightarrow \mathds{R}$ tal que para todo $x \in K \Rightarrow f(x) = f_l(x) \ \forall \ l \in \mathds{N}$

Seja $f:[-\pi,\pi] \rightarrow \mathds{R}$ de classe $C^2$ (por partes ou não) tal que $f(\pi) = f(-\pi)$. Então a série de Fourier de $f$ converge uniformemente em $\mathds{R}$ para $\displaystyle{F(x) = \frac{a_0}{2} + \soma{1} [a_n \cos (nx) + b_n \sin (nx)]}$. Em particular, para $x \in [-\pi, \pi]$ temos $F(x) = f(x)$

\begin{theorem}
Seja $f$ uma função periódica com período $2\pi$ e de classe $C^2$ por partes em $[-\pi,\pi]$ (não necessariamente contínua em $[-\pi,\pi]$). Sejam $a_n$ e $b_n$ os coeficiente de Fourier da função $f$. Então: \begin{itemize}
    \item se $f$ é contínua em $x$: $\displaystyle{f(x) = \frac{a_0}{2} + \soma{1} (a_n \cos (nx) + b_n \sin (nx))}$
    \item se $f$ não é contínua em $x$: $\displaystyle{f(x) = \frac{f(x^+) + f(x^-)}{2} = \frac{a_0}{2} + \soma{1} (a_n \cos (nx) + b_n \sin (nx))}$
\end{itemize}
\end{theorem}

\section*{\E 1} Determine a série de Fourier de $f(x) = x\ -\pi \leq x \leq \pi$ (\textbf{Dica:} calcule os coeficientes)\footnote{conferir a resolução do exercício} \begin{align*}
    a_0 &= \frac{1}{\pi} \IP x \ dx =     \frac{1}{\pi}\left[\frac{x^2}{2}\right]_{-\pi}^{\pi} =\frac{1}{\pi}         \left[\frac{\pi^2}{2}-\frac{(-\pi)^2}{2}\right] = 0 \\
    a_n &= \frac{1}{\pi}\IP x \cos (nx) \ dx = 0 \\
    b_n &= \frac{1}{\pi}\IP x \sin (nx) \ dx = \frac{2}{\pi} \int_0^\pi x \sin (nx)\ dx = \frac{2}{\pi} \left[ \left. -\frac{x \cos (nx)}{n}\right|_0^\pi - \int_0^\pi \frac{\cos (nx)}{n} \ dx \right] \\
    &= \frac{2}{\pi}\left[-\frac{\pi \cos (n\pi)}{n}-\left(\frac{\sin (nx)}{n^2}\right)_0^\pi \right] = -\frac{2 * \cos (n\pi)}{n} \\
    f(x) &= \soma{1} \frac{2*(-1)^n}{n}\sin (nx)
\end{align*}

\section*{\E 2} Determine a série de Fourier de $f(x) = x^2\ -\pi \leq x \leq \pi$ \begin{align*}
    a_0 &= \frac{1}{\pi} \IP x^2 \ dx = \frac{1}{\pi} \left[\frac{\pi^3}{3}-\left[-\frac{(-\pi)^3}{3}\right]\right] = \frac{2\pi^2}{3}\\
    a_n &= \frac{1}{\pi} \IP x^2 \cos (nx) \ dx = \frac{2}{\pi} \int_0^\pi x^2 \cos (nx) \ dx = \frac{2}{\pi}\left( \left[ \frac{x^2 \sin (nx)}{n}\right]_0^\pi- \int_0^\pi \frac{2x \sin (nx)}{n} \ dx\right) \\
    &= \frac{2}{\pi}\left(-\frac{2}{n}\left(\left[-\frac{x \cos (nx)}{n}\right]_0^\pi - \int_0^\pi -\frac{1}{n}\cos (nx)\ dx \right)\right) = \frac{2}{\pi}\left(-\frac{2}{n}\left(-\frac{\pi \cos (n\pi)}{n} + \left[\frac{\sin (nx)}{n^2}\right]_0^\pi\right)\right) \\
    &= \frac{4}{n^2}\cos (n\pi)\\
    b_n &= \frac{1}{\pi} \IP x^2 \sin (nx) \ dx = 0 \\
    f(x) &= \frac{\pi^2}{3}+\soma{1} \frac{(-1)^n * 4}{n^2} \cos (nx)
\end{align*}

\section*{\E 3} Determine a série de Fourier de $f(x) = |x|\ -\pi \leq x \leq \pi$ \begin{align*}
    a_0 &= \frac{1}{\pi} \IP |x| \ dx = \frac{2}{\pi} \int_0^\pi x \ dx = \left.\frac{2}{\pi}\frac{x^2}{2}\right|_0^\pi = \pi \\
    a_n &= \frac{1}{\pi} \IP |x| \cos (nx) \ dx = \frac{2}{\pi} \int_0^\pi x \cos (nx) \ dx = \frac{2}{\pi} \left(\left.\frac{x \sin (nx)}{n}\right|_0^\pi - \int_0^\pi \frac{\sin (nx) }{n} \ dx\right)\\
    &= \frac{2}{\pi}\left[\frac{\cos (nx)}{n^2}\right]_0^\pi = \frac{2}{\pi n^2}[(-1)^n - 1] = \frac{-4}{\pi n^2} \Rightarrow n \ impar\\
    b_n &= \frac{1}{\pi} \IP |x| \sin (nx) = 0 \\
    f(x) &= \frac{\pi}{2} + \soma{1} \frac{-4}{\pi (2n-1)^2}\cos ((2n-1)x)
\end{align*}

\section*{\E 4} Determine uma função não nula $F:[-\pi,\pi] \rightarrow \mathds{R}$ cuja série de Fourier convirja uniformemente para $f(x) = 0$ em $[0,\pi]$.

\textbf{Exemplos de resolução:}

$\hbox{f}(t)
= \left\{ \begin{array}{rll}
\sin x & \hbox{se} &  -\pi \leq x < 0 \\
0 & \hbox{se} & 0 \leq x \leq \pi \\
\end{array}\right.$
\vspace {5mm}
$\hbox{f}(t)
= \left\{ \begin{array}{rll}
x + \pi & \hbox{se} &  -\pi \leq x < -\frac{\pi}{2} \\
-x & \hbox{se} & -\frac{\pi}{2} \leq x < 0 \\
0 & \hbox{se} & 0 \leq x \leq \pi \\
\end{array}\right.$

Note que em ambas resoluções, $f(x) = 0 \ \forall x \in [0,\pi]$

\begin{framed}
\begin{definition}
    \centering{Seja uma função $f:[-\pi,\pi] \rightarrow \mathds{R} \  \land \ f(\pi) = f(-\pi)$ e $C^2$ por partes. Tomando um intervalo $I \subset [-\pi,\pi]$. Se a Série de Fourier da função $f$ convergir  a ela no intervalo $I$, então convirgirá em todo $x \in [-\pi,\pi]$ \label{enunciado}}
\end{definition}
\end{framed}

\chapter{Equações Diferenciais}{Aulas 26/10/2015 -- 00/11/2015}
\label{chap:c7}

\vspace{5mm}
São equações que envolvem uma função e suas derivadas. Exemplos: \begin{align}
    F(t) = m \ \frac{d^2x}{dt^2} \label{Newton} \\
    \frac{\partial^2 E}{\partial x^2} + \frac{\partial^2 E}{\partial y^2} + \frac{\partial^2 E}{\partial z^2} = 4\pi p \label{Maxwell} \\
    y' = y \label{ed1} \\
    y' = y^{\frac{3}{2}} \label{ed2}
\end{align}

\underline{Variáveis independentes} são as variaveis em relação as quais estamos derivando. Exemplos: $t$ no exemplo \ref{Newton} e $x, y, z$ no exemplo \ref{Maxwell}.

\underline{Variáveis dependentes} são aquelas que estão sendo derivadas. Exemplo: $x$ no exemplo \ref{Newton} e $E$ no exemplo \ref{Maxwell}.

\section{Definições}

\begin{align}
    \frac{d^2y}{d x^2} + 5          \left(\frac{dy}{dx}\right)^2 - 7xy = \cos x \label{ed3} \\
    -x^2y^{(IV)} + \cos x \ y'' + y\sqrt{1+x^2} = \sin x \label{ed4}
\end{align}

\hspace{5mm}Equações diferenciais \textbf{ordinárias} (e.d.o): têm apenas uma variável independente. Exemplos: \ref{Newton}, \ref{ed1}, \ref{ed2}.

Equações derivadas parciais (e.d.p): têm mais de uma variada independente. Exemplo: \ref{Maxwell}

A \underline{ordem} de uma \underline{e.d.o.} é a ordem da maior derivada que aparece na equação. \begin{itemize}
    \item a equação \ref{Newton} é de ordem 2.
    \item a equação \ref{ed3} é de ordem 2.
    \item a equação \ref{ed4} é de ordem 4.
\end{itemize}

Defina $F: \mathds{R}^4 \rightarrow \mathds{R}: F(x_1,x_2,x_3,x_4) = x_4 + 5x_3^2 - 7x_1x_2 - \cos x_1$. $F$ é equivalente à equação \ref{ed3}.

De um modo mais geral, dada uma equação diferencial de ordem $n$, podemos representá-la da forma $F(x, y, y', y'', \hdots, y^{(n)}) = 0$ onde $F:\mathds{R}^{n+2} \rightarrow \mathds{R}$

Equações diferenciais lineares -- São equações onde $y$ e suas derivadas aparecem com expoente de no máximo 1 $$ a_0(x)\frac{d^n y}{d x^n}+ a_1(x) \frac{d^{n-1} y}{d x^n} + \cdots + a_n(x)y = f(x) $$ Exemplo: equações \ref{Newton} e \ref{ed4}

Resolvendo, como primeiro exemplo, a equação \ref{ed1}: \begin{align*}
    y' &= y \\
    y &= c*e^x \ \ \ \forall \ x \in \mathds{R} \\
    y' &= c*e^x \\
    y(0) &= 1 \ \ \  (cond.) \\
    y(x) &= c*e^x \\
    y(0) &= 1 \Longleftrightarrow c*e^0 = 1 \Longleftrightarrow c = 1 \\
    y(x) = e^x
\end{align*}

De uma forma mais genérica, teremos um sistema parecido com esse: \begin{align*}
    F(x,y,y',y'') &= 0 \\
    y(x_0) = y_1 \\
    y'(x_0) = y_2
\end{align*}


\section{Equações diferenciais de 1ª ordem}{Alguns métodos de resolução}

\subsection{Equações exatas}

\hspace{5mm} Encontrar equações do tipo \begin{align}
    Pdx + Qdy = 0 \ \ \ \ \left(\frac{dy}{dx} = - \frac{P(x,y)}{Q(x,y)}\right) \label{ed5}
\end{align}

onde $P(x,y)$ e $Q(x,y)$ estão definidas em um domínio $D \subset \mathds{R}$; Por exemplo: \begin{align*}
    y \ dx + \left(x + \frac{2}{y}\right)\ dy = 0 \\
    P(x,y) &= y \\
    Q(x,y) &= x + \frac{2}{y}
\end{align*}

Uma equação como a do exemplo \ref{ed5} é considerada \underline{exata} se existir $\defini{\varphi}{D}$ suficientemente diferenciável tal que \begin{align*}
    \frac{\partial \varphi}{\partial x}(x,y) &= P(x,y) \\
    \frac{\partial \varphi}{\partial y}(x,y) &= Q(x,y) \\
        (x,y) &\in D
\end{align*}

Supondo que a expressão \ref{ed5} seja exata,  então temos que \begin{align*}
    \frac{\partial \varphi}{\partial x}(x,y) \ dx + \frac{\partial \varphi}{\partial y}(x,y)\ dy = 0 \ \lor \ d\varphi = 0 \Rightarrow \\
    \Rightarrow \varphi(x,y) = c \ \ (cte) \\
\end{align*}

No exemplo numérico, temos que $\varphi(x,y) = xy + \ln y^2$. Portanto, as soluções da equação são dadas implicitamente por $\varphi(x,y) = c \lor \ xy + \ln y^2 = c$

Que condições $P$ e $Q$ devem obedecer para a equação \ref{ed5} ser exata? (Considerando que exista uma função $\varphi$) \begin{align*}
    \frac{\partial \varphi}{\partial x} = P & \ & \frac{\partial \varphi}{\partial y} = Q \\
    \frac{\partial^2 \varphi}{\partial y \ \partial x} = \frac{\partial P}{\partial y} &  \  & \frac{\partial^2 \varphi}{\partial x \ \partial y} = \frac{\partial Q}{\partial x} \\
    \ & \frac{\partial P}{\partial y} = \frac{\partial Q}{\partial x} & \
\end{align*}

Considere a equação $P(x,y) \ dx + Q (x,y) \ dy = 0$ e o fato de que $\DS{\frac{\partial P}{\partial y}(x,y) = \frac{\partial Q}{\partial x}(x,y)}$ e que o conjunto $D$ seja simplesmente conexo. Então $$ \varphi(x,y) = \int_\gamma (P(x,y) \ dx + Q(x,y) \ dy) $$ onde $\gamma$ é uma curva diferenciável por partes que une dois pontos pertencentes a $D$

\subsection*{Exemplo 1}{Resolver a equação $$ \left(x + \frac{2}{y}\right) \ dy + y \ dx = 0 $$}
\begin{align*}
    P(x,y) = y \Rightarrow \frac{\partial P}{\partial y} = 1 \\
    Q(x,y) = x + \frac{2}{y} \Rightarrow \frac{\partial Q}{\partial x} = 1
\end{align*}
\begin{figure}
\centering
\includegraphics[width=0.5\textwidth]{new}
\caption{Intervalo de Integração: Resolução do exemplo 1}
\label{fig:diff-1}
\end{figure}

\begin{align*}
    \varphi(x,y) &= \int_b^y \left(a+ \frac{2}{y}\right) \ dy + \int_a^x y \ dx \\
    &= [ay + 2 \ln y]_b^y + xy - ay = ay + \ln y^2 - (ab + \ln b^2) + xy -ay \\
    &= \ln y^2 + xy = k, \ \ ab + \ln b^2 \\
    \varphi (x,y) &= xy +\ln y^2 = c
\end{align*}

Outra forma: Queremos um $\varphi$ tal que:
\begin{align*}
    \frac{\partial \varphi}{\partial x} = y \ &\land \ \frac{\partial \varphi}{\partial y} = x + \frac{2}{y} \\
    \varphi(x,y) = \int \frac{\partial \varphi}{\partial x} \ dx &= \int y \ dx = xy + k(y) \\
    \frac{\partial \varphi}{\partial y} = x + \frac{2}{y} &= x + k'(y) \\
    k'(y) = \frac{2}{y} &\Rightarrow k(y) = 2 \ln y = \ln y^2 \\
    \varphi(x,y) &= xy + \ln y^2 = c
\end{align*}

Vamos tomar um tipo específico de equação diferencial exata. $$f(x) \ dx + g(y) \ dy = 0$$ onde $P(x,y) = f(x)$ e $Q(x,y) = g(y)$. Neste caso, teremos a derivada parcial de $P$ em $y$ é igual à derivada parcial de $Q$ em respeito a $x$ são iguais a zero e $\varphi(x,y) = \int_a^x f(x) \ dx + \int_b^y g(y) \ dy$

\subsection*{Exemplo 2: Resolva a equação $$(\ln x \cos y)dx + (x \tan y)dy = 0 \ \ x > 0 $$}

\hspace{5mm} Dividindo a expressão por $x \ \cos y$ teremos uma expressão parecida com o nosso modelo. \begin{align*}
    \left( \frac{\ln x}{x}\right) \ dx + \left( \frac{\sin y}{\cos^2 y}\right)\ dy = 0 \ \ \cos y \neq 0 \ \land \ x >0 
    \frac{\partial \varphi}{\partial x}(x,y) &= \frac{\ln x}{x}\\
    \frac{\partial \varphi}{\partial y}(x,y) &= \frac{\sin y}{\cos^2 y} \\
    \varphi(x,y) = \int \frac{\partial \varphi}{\partial x}\ dx &= \int \frac{\ln x}{x}\ dx = \frac{(\ln x)^2}{2} + k(y) \\
    \frac{\partial \varphi}{\partial y} &= k'(y) = \frac{\sin y}{\cos^2 y}\\
    k'(y) = \frac{\sin y}{\cos^2 y} &\Rightarrow k(y) = \frac{1}{\cos y} \\
    \varphi(x,y) = \frac{(\ln x)^2}{2}+ \frac{1}{\cos y}
\end{align*}

\subsection*{Exemplo 3}

\hspace{5mm} Dada $P(x,y) \ dx + Q(x,y) \ dy$

\part{Provas}
\chapter{Correção P1}
\label{C:P1}

\section*{\E{1} (2,5)}{\textbf{Fato:} Se $\displaystyle{\LI a_n = a}$ \footnote{pode ser $\pm \infty$} então $$\LI {a_1 + a_2 + a_3 + \cdots + a_n \over n} = a$$ Seja $(b_n)$ uma sequência de números positivos tal que $\displaystyle{\LI b_n = b \geq 0}$ Mostre que $$\LI \sqrt[n]{b_1*b_2\cdots b_n} = b $$}

\Resolve

\begin{align*}
\sqrt[n]{b_1*b_2\cdots b_n} &= e^{\ln{\sqrt[n]{b_1*b_2\cdots b_n}}} = e^{{1 \over n} \ln{b_1} + \ln{b_2} + \cdots + \ln{b_n}}\\
\LI \ln{b_n} &= \ln{b} \Rightarrow \LI {\ln{b_1} + \ln{b_2} + \ln{b_3} + \cdots + \ln{b_n} \over n} = \ln{b} \\
\sqrt[n]{b_1*b_2\cdots b_n} &= e^{\ln{b_1} + \ln{b_2} + \ln{b_3} + \cdots + \ln{b_n} \over n} = e^{\ln{b}} = b \\
\end{align*}
Logo $\displaystyle{\LI \sqrt[n]{b_1*b_2\cdots b_n} = b}$

\newpage

\section*{\E{2} (2,5)} {Determine se são convergentes ou não as séries:}
\begin{multicols}{2}
a) (1,0) $\displaystyle{\soma{2} {n! \ \left({5 \over 2}\right)^n \over n^n}}$

b) (1,5) $\displaystyle{\soma{1} \sin{\left( {1 \over n \sqrt[n]{n^2 + 3}} \right)}}$ \footnote{Sugestão use comparação no limite com $\displaystyle{\soma{1} {1 \over n^{5 \over 3}}}$}
\end{multicols}

\Resolve

\begin{multicols}{2}
\subsection*{Item A}
Temos $\displaystyle{{a_{n+1} \over a_n} = {(n+1)!\ \left({5 \over 2}\right)^{n+1} \over (n+1)^{n+1}}*{n^n \over n! \ \left({5 \over 2}\right)^n}} = \\ = {5 \over 2}*\left({n \over n+1}\right)^n = {{5 \over 2} \over e} < 1$
===
Logo, a série converge pelo Critério da Razão.

\subsection*{Item B}
\begin{align*}
\LI {\sin{\left( {1 \over n \sqrt[n]{n^2 + 3}} \right)} \over {1 \over n^{5 \over 3}}} \\
\LI {\sin{\left( {1 \over n \sqrt[n]{n^2 + 3}} \right)} \over {1 \over n \sqrt[n]{n^2 + 3}}}*{n^{5 \over 3} \over n \sqrt[n]{n^2 + 3}} = 1*1 = 1 \\
\end{align*}

Como a série $\displaystyle{\soma{1} {1 \over n^{5 \over 3}}}$ converge, pelo critério da comparação no limite, a série converge.


Outra forma de constatar que a série  é convergente passa pelo fato de que que $\sin x < x \ \ \forall x > 0$, mas isso eu deixo pra você pensar.
\end{multicols}

\newpage

\section*{\E{3} (2,5)}{Decida se as séries abaixo divergem, convergem condicionalmete ou absolutamente}

\begin{multicols}{3}
a) (1,0) $\displaystyle{\soma{2} {(-1)^n \over n \ln{n}}}$

b) (1,0) $\displaystyle{\soma{0} {(-1)^n(n+1) \over n+2}}$

c) (0,5) $\displaystyle{\soma{1} {\sin\left({n\pi \over 2}\right) \over n^2}}$
\end{multicols}

\Resolve

\begin{multicols}{3}
\subsection*{Item A}
$\displaystyle{\soma{2} \left|{(-1)^n \over n \ln{n}}\right| = \soma{2} {1 \over n \ln{n}}}$ que é diverge. Levaremos em conta uma mudança de variável para o pŕóximo passo.\footnote{$x = e^y \Longleftrightarrow dx = e^y \ dy$} \begin{align*}
\int_2^\infty {dx \over x\ln{x}} &= \int_a^\infty {e^y \ dy \over e^y \ y} \\
&= \int_a^\infty {dy \over y} \\
&= \left[\ln({\ln{y}})\right]_2^\infty \\
&= \infty
\end{align*}

Logo, não converge absolutamente. Agora, consideramos a sequência $a_n = {1 \over n\ln{n}}$ é positiva, descresente e tende a 0 quando n "vai pro infinito". Pelo critério de Liebniz, a sériealternada em questao converge condicionalmente.

\subsection*{Item B}
$\not \exists \LI a_n = \LI {(-1)^n \ (n+1) \over (n+2)}$ Logo, a série diverge.

\subsection*{Item C}
Como $\displaystyle{a_n = \left|{\sin\left({n\pi \over 2}\right) \over n^2}\right| \leq {1 \over n^2}}$ e a sérei $\displaystyle{\soma{1} \frac{1}{n^2}}$  converge pelo critério da comparação, a série converge absolutamente.
\end{multicols}

\newpage

\section*{\E{4} (2,5)}{Determine para que valores \textbf{positivos} de $x \in \mathds{R}$ as funções abaixo estão definidas}

\begin{multicols}{4}
a) $\displaystyle{\soma{0} {n \over e^{xn}}}$

b) $\displaystyle{\soma{1} {n^n x^n \over n!}}$

c) $\displaystyle{\soma{1} {x^n \over n \ln{n}}}$

a) $\displaystyle{\soma{0} {5x^n \over 2^n}}$
\end{multicols}

\Resolve

\begin{multicols}{2}
\section*{\IT{A}}\begin{align*}
a_n &= {n \over e^{xn}} \\
\LI \sqrt[n]{a_n} &= \LI {1 \over e^x} < 1 \Longleftrightarrow\\
\Longleftrightarrow e^x > 0
\end{align*}

Logo $f(x)$ está definida $\forall \ x > 0$

\section*{\IT{B}}\begin{align*}
a_n = {n^n x^n \over n!} &\Rightarrow {a_{n+1} \over a_n} = {(n+1)^{n+1} x^{n+1} \over (n+1)!}*{n! \over n^n x^n}\\
\LI {a_{n+1} \over a_n} &= x*\left({n+1 \over n}\right)^n = x*e < 1
\end{align*}

Logo a função está definida para todo o $\displaystyle{x < \frac{1}{e}}$

\section*{\IT{C}}\begin{align*}
a_n = {x^n \over n \ln{n}} &\Longleftrightarrow \sqrt[n]{a_n} = { x \over \sqrt[n]{n} \sqrt[n]{\ln{n}}}\\
\LI \sqrt[n]{a_n} &= 1 \\
\hbox{f}(x)
&= \left\{ \begin{array}{rll}
converge & \hbox{se} &  x < 1 \\
diverge & \hbox{se} &  x \geq 1 \\
\end{array}\right.
\end{align*}

\section*{\IT{D}}\begin{align*}
a_n &= {5x^n \over 2^n} \\
\LI \sqrt[n]{a_n} &= \LI {\sqrt[n]{5} x \over 2} = {x \over 2} < 1 \\
\hbox{f}(x)
&= \left\{ \begin{array}{rll}
converge & \hbox{se} &  x < 2 \\
diverge & \hbox{se} &  x \geq 2 \\
\end{array}\right.
\end{align*}

\end{multicols}

\chapter{Correção P2}
\label{C:P2}

\section*{\E{1} (2,5)}{São dadas as seguintes séries de potências \begin{align}
    \soma{1} (-1)^n \ 2n\  x^{2n-1} \label{P2-1} \\
    \soma{2} (-1)^n \frac{x^n}{n(n-1)} \label{P2-2}
\end{align}}

a) \textbf{(1,0)} Determione o raio de convergência das séries dadas

b) \textbf{(1,5)} Identifique as funções dadas pelas séries acima

\Resolve

\subsection*{Item A}

\hspace{5mm}No caso da série \ref{P2-1}, não podemos usar o critério da razão do módulo diretamente, pois é necessário que todos os graus de $x$ estejam presentes Então, a opção é integrar a série termo a termo, pois o raio de convergência não se altera se derivarmos ou integrarmos a função. \begin{align*}
    \soma{1} \int_0^x \ (-1)^n \ 2n \ t^{2n-1} \ dt &= \soma{1} \ (-1)^n \ x^{2n} \ \land \ y = x^2 \\
    \soma{1} (-1)^n y^n \ &\land a_n = (-1)^n\\
    \LI \frac{|a_n|}{|a_{n+1}|} &= 1 = R \\
    |y| < 1 \land y = x^2 &\Longleftrightarrow |x^2| < 1 \Rightarrow |x| < 1
\end{align*}

No caso da série \ref{P2-2}, por outro lado, pode-se descobrir o raio de convergência atraves do quociente dos módulos. $$R = \LI \frac{|a_n|}{|a_{n+1}|} = \LI \frac{(n+1)n}{n(n-1)} = 1$$

\subsection*{Item B}

\begin{align*}
    f_1(x) &= \soma{1} \ (-1)^n \ 2n\  x^{2n-1}  \ \ \land |x| < 1 \\
    F_1 (x) &= \int_0^x f_1(t) dt = \soma{1} \ \int_0^x (-1)^n \ 2n x^{2n-1} \ dt = \soma{1} \ (-1)^n \ x^{2n} \\
    &= -x^2 + x^4 -x^6 + x^8 - x^{10} + \cdots \\
    \frac{1}{1 + x^2} &= 1 -x^2 + x^4 -x^6 + x^8 - x^{10} + \cdots \\
    F_1 (x) &= \frac{1}{1 + x^2} - 1 = \frac{-x^2}{1+x^2} \\
    f_1(x) &= F_1'(x) = \frac{-2x(1+x^2) - 2x(-x^2)}{(1+x^2)^2} = \frac{-2x}{(1+x^2)^2} \\
    \hline
    f_2(x) &= \soma{2} (-1)^n \ \frac{x^n}{n(n-1)} \\
    &= \frac{x^2}{2*1}-\frac{x^3}{3*2}+\frac{x^4}{4*3}-\frac{x^5}{5*4}+\cdots \\
    f_2'(x) &= x - \frac {x^2}{2} + \frac{x^3}{3} - \frac{x^4}{4} + \cdots = \ln (1+x) \\
    f_2(x) &= \int_0^x \ln (1+t) \ dt = (1+x) \ln (1+x) - (1+x)
\end{align*}


\newpage

\section*{\E{2} (2,5)}

a) \textbf{(1,0)} Usando a série binomial $$(1 + x)^\alpha = 1 + \soma{1} \frac{\alpha(\alpha - 1) \cdots(\alpha - n + 1)}{n!} \ x^n,\ \ |x| <1$$ desenvolva $\displaystyle{\frac{1}{\sqrt{1-x^2}}}$ em série de potências

b) \textbf{(1,0)} Determine uma série de potências para $\arcsin x, \ |x| < 1$

c) \textbf{(0,5)} Determine uma série numérica para $\frac{\pi}{6}$

\Resolve 

\subsection*{Item A} \begin{align*}
    \frac{1}{\sqrt{1-x^2}} = (1-x^2)^{-\frac{1}{2}} &= 1 + \soma{1} \frac{\left(-\frac{1}{2}\right)\left(-\frac{1}{2} - 1\right)\left(-\frac{1}{2} - 2\right)\cdots\left(-\frac{1}{2} - n+1\right)}{n!}\ (-x^2)^n \\
    &= 1 +\soma{1} \frac{(-1)^n}{2^n} \ \frac{1*3*5\cdots(2n-1)}{n!} \ (-1)^n \ x^{2n} \\
    &= 1 +\soma{1} \frac{1*3*5\cdots(2n-1)}{2^n \ n!} \ x^{2n}
\end{align*}

\subsubsection*{Item B} \begin{align*}
    \arcsin x = \int_0^x \frac{dt}{\sqrt{1-t^2}}\ dt &= \int_0^x 1 + \soma{1} \frac{1*3*5\cdots(2n-1)}{2^n \ n!} \ t^2 \ dt \\ 
    \arcsin x &= x + \soma{1} \frac{1*3*5\cdots(2n-1)}{2^n \ n!} \ \frac{x^{2n+1}}{2n+1}
\end{align*}

\subsection*{Item C} \begin{align*}
    \frac{\pi}{6} &= \arcsin \left(\frac{1}{2}\right) \\
    &= \frac{1}{2} + \soma{1} \frac{1*3*5*\cdots(2n-1)}{2^n \ n!} \ \frac{\frac{1}{2^{2n+1}}}{2n+1} \\
    \frac{\pi}{6} &= \frac{1}{2} + \soma{1} \frac{1*3*5\cdots(2n-1)}{2^{3n+1} \ n! (2n+1)}
\end{align*}

\newpage

\section*{\E{3} (2,5)}{Desenvolva as funções em séries de potências centradas em $a$ dado. Escreva as séries encontradas e os seus domínios}
\begin{multicols}{2}
    \begin{itemize}
       \item [a] \textbf{(1,5)} $\DS{f(x) = \frac{1}{1+2x} \ \ (a = 1)}$
       \item [b] \textbf{(1,0)} $\DS{f(x) = \int_0^x e^{-t^2} \ dt \ \ (a = 0)}$
    \end{itemize}
\end{multicols}

\Resolve

\begin{multicols}{2}
    \subsection*{Item A} 
    \begin{align*}
        f(x) &= \frac{1}{1+2x} = \frac{1}{1+2(x-1) + 2} \\
        &= \frac{1}{3 + 2(x-1)} = \frac{1}{3} \ \frac{1}{1+\frac{2(x-1)}{3}} \\
        \frac{1}{1+\frac{2(x-1)}{3}} &= \soma{0}\ (-1)^n \frac{2^n}{3^n} \ (x-1)^n \\
        f(x) &= \soma{0} \ (-1)^n \ \frac{2^n}{3^{n+1}} \ (x-1)^n \\
         \LI\frac{|a_n|}{|a_{n+1}|} &= \LI \left( \frac{2^n}{3^{n+1}}*\frac{3^{n+2}}{2^{n+1}} \right) = \frac{3}{2} \\
         R &= \frac{3}{2} \Rightarrow -\frac{3}{2} < x-1 < \frac{3}{2} \\
         -\frac{3}{2}+1 < x &< \frac{3}{2}+1 \Longleftrightarrow -\frac{1}{2} < x < \frac{5}{2}
    \end{align*}
    Nos extremos temos \begin{align}
        x - 1 &= -\frac{3}{2} \Rightarrow \soma{0} \ (-1)^n \ \frac{2^n}{3^{n+1}} \ (-1)^n \ \frac{3^n}{2^n} = \soma{0} \frac{1}{3} \label{parte1} \\
        x-1 &= \frac{3}{2} \Rightarrow \soma{0} \ (-1)^n \ \frac{2^n}{3^{n+1}} \ \frac{3^n}{2^n} = \soma{0} \frac{(-1)^n}{3} \label{parte2}
    \end{align}
    As séries \ref{parte1} e \ref{parte2} são divergentes logo o intervalo será aberto dos dois lados. $D_f = ]-\frac{1}{2},\frac{5}{2}[$
    \subsection*{Item B}
    \begin{align*}
        e^x &= \soma{0} \frac{x^n}{n!} \\
        e^{-t^2} &= \soma{0} \frac{(-t^2)^n}{n!} = \soma{0} (-1)^n \frac{t^{2n}}{n!}\\
        \int_0^x e^{-t^2} \ dt &= \soma{0} \int_0^x (-1)^n \frac{t^{2n}}{n!} \ dt = \soma{0} (-1)^n \ \frac{x^{2n+1}}{(2n+1)n!} \\
        D_f = \mathds{R}
    \end{align*}
\end{multicols}

\newpage

\section*{\E{4} (2,5)}

 \hspace{5mm}a) \textbf{(1,0)} Determine uma função $f:[-\pi,\pi] \rightarrow \mathds{R}$ satisfazendo que $f(\pi) = 0 = f(-\pi)$ e cuja série de Fourier convirja uniformemente  em $[-1,1]$ para a função $g(x) =x$. Exmplique porque a convergência acontece (a função $f$ pode ser tomada como ímpar)

b) \textbf{(1,5)}  Determine a série de Fourier da função do item anterior

\Resolve 

\subsection*{Item A}
\hspace{5mm}Levando em conta que a série abaixo segue as condições do postulado da página \pageref{enunciado}, então em um dado intervalo (no caso, $[-1,1]$), como a função converge a $g(x) = x$, então a série de Fourier da função $f$ converge a $g$ em todo o intervalo $[-\pi, \pi]$
\begin{figure}[ht]
\centering
\includegraphics[width=0.5\textwidth]{e41}
\caption{Gráfico Resultate da função encontrada}
\label{fig:res4-p2}
\end{figure}
\begin{center}
    $\hbox{f}(t)
    = \left\{ \begin{array}{rlll}
    (r1) & \frac{x+\pi}{1-\pi} & \hbox{se} &  -\pi \leq x < 1 \\
    & x & \hbox{se} &  -1 \leq x \leq 1 \\
    (r2) & \frac{x-\pi}{1-\pi} & \hbox{se} & 1 < x \leq \pi \\
    \end{array}\right.$
\end{center}

\subsection*{Item B}

\begin{align*}
    a_n &= 0 \ \ \ n \geq 0 \\
    b_n = \frac{2}{\pi}\int_0^\pi f(x) \ \sin (nx) \ dx &= \frac{2}{\pi}\left[\underbrace{\int_0^1 x \sin (nx) \ dx}_A + \underbrace{\int_1^\pi \frac{x-\pi}{1-\pi} \ \sin (nx) \ dx}_B\right] \\
    A = \int_0^1 x \sin (nx) \ dx &= \left.-\frac{x \cos (nx)}{n}\right|_0^1 + \frac{1}{n}\int_0^1 \cos (nx) \\
    &= -\frac{\cos (nx)}{n} + \left.\frac{1}{n^2} \ \sin (nx)\right|_0^1  =\frac{\sin n}{n^2} - \frac{\cos n}{n}\\
    B = \int_1^\pi \frac{x-\pi}{1-\pi} \sin (nx) \ dx &= \frac{1}{1-\pi}\left[ \int_1^\pi \ x \sin (nx) \ dx - \pi \left( \int_0^\pi \sin (nx) \ dx \right) \right] \\
    &= \frac{1}{1-\pi}\left[ -\left.\frac{x \cos (nx)}{n} \right|_1^\pi + \left.\frac{\sin (nx)}{n^2}\right|_1^\pi - \pi \left.\left(-\frac{\cos (nx)}{n}\right)\right|_1^\pi\right] \\
    &= \frac{1}{1-\pi} \left[-\frac{\pi (-1)^n}{n} + \frac{\cos n}{n}-\frac{\sin n}{n^2}+\frac{\pi (-1)^n}{n} + \frac{\pi \cos n}{n}\right] \\
    &= \frac{1}{1-\pi} \left[\frac{(1+\pi) \cos n}{n} - \frac{\sin n}{n^2} \right] \\
    b_n &= \frac{2}{\pi} \left[\frac{\sin n}{n^2} - \frac{\cos n}{n} + \frac{1}{1-\pi} \left(\frac{(1+\pi) \cos n}{n} - \frac{\sin n}{n^2} \right)\right] \\
    &= \frac{2}{\pi} \left[\frac{\cos n}{n} \frac{2\pi}{1-\pi} - \frac{-\pi}{1-\pi} \frac{\sin n}{n^2}\right] = \frac{4}{(1-\pi)} \frac{\cos n}{n} - \frac{2}{(1-\pi)}\frac{\sin n}{n^2}\\
\end{align*}


\chapter{Correção P3}
\label{C:P3}

\section*{\E{1} (2,5) Resolva as equações abaixo:}

\hspace{5mm} a) \textbf{(1,0)} $(\sin x \ln y)\ dy + (y \cot x)\ dx = 0\ \ \ (y > 0 \ \land \ x \neq k\pi \ \ k \in \mathds{Z})$

b) \textbf{(1,5)} $(3xy + 2y)\ dx + (x^2 + 2x)\ dy = 0$

\Resolve

\textbf{Item A} Considerando as condições enunciadas, podemos dividir a equação por $y*\sin x$ e assim obtemos $$ \frac{\ln y}{y}dy + \frac{\cos x}{\sin ^2 x}dx = 0 $$ que é uma equação de variáveis separáveis. Logo $$ \int \frac{\ln y}{y}dy + \int \frac{\cos x}{\sin^2}dx = 0 \Rightarrow \frac{(\ln y)^2}{2} - \frac{1}{\sin x} = 0$$

Solução Geral \begin{align*}
    \varphi (x,y) &= c \\
    \varphi (x,y) &= \frac{(\ln y)^2}{2} - \frac{1}{\sin x}
\end{align*}

\textbf{Item B} \begin{align*}
    P(x,y) &= 3xy + 2y \\
    Q(x,y) &= x^2 + 2x \\
    \difff{P}{y} - \difff{Q}{x} &= 3x +2 - (2x + 2) = x \neq 0 \\
    g(x) = \frac{1}{Q(x,y)}\left(\difff{P}{y} - \difff{Q}{x}\right) &= \frac{x}{x^2 +2x} = \frac{1}{x+2}
\end{align*}

É recomendável, nessa situação, que se aplique o método dos fatores integrantes, multiplicando a equação por um polinômio $\mu(x)$ de forma a obter uma equação exata: \begin{align*}
    \mu(x) &= e^{\int \frac{dx}{x+2}} = e^{\ln (x+2)} = x + 2 \\
    (x+2) (3xy + 2y)dx &+ (x+2) (x^2+2x)dy = 0 \\
    (3x^2y + 8xy + 4y)dx &+ (x^3 + 4x^2 + 4x)dy = 0 \\
    \difff{\varphi}{x} &= 3x^2y + 8xy + 4y \\
    \difff{\varphi}{y} &= x^3 + 4x^2 + 4x \\
    \varphi (x,y) &= \int \difff{\varphi}{x}dx = \int (3x^2y + 8xy + 4y) dx = x^3y + 4x^2y + 4xy + k(y) \\
    \difff{\varphi (x,y)}{y} &= \difff{\varphi}{y} \Rightarrow x^3 + 4x^2 + 4x + k'(y) = x^3 + 4x^2 + 4x \Rightarrow k'(y) = 0 = k(y) \\
    \varphi (x,y) &= x^3y + 4x^2y + 4xy + k\\
    \varphi (x,y) &= C \ \ \ (sol. \ \ geral)
\end{align*}

\newpage

\section*{\E{2} (2,5)}

\hspace{5mm} a) \textbf{(1,0)} Encontre a solução geral de $$y'' + y = 0$$

b) \textbf{(1,5)} Encontre a solução geral de $$y'' + y = \sin x \  e^x$$

\Resolve

\textbf{Item A:} A expressão tem como equação característica $x^2 + 1 = 0$ que tem como raízes $0 \pm i$. Assim, a solução geral dessa equação homogênia é $$y = c_1 \sin x + c_2 \cos x$$

\textbf{Item B:} A partir da resolução do exercício anterior, temos que $y_h(x) = c_1 \sin x + c_2 \cos x$. Para resolver a equação desse item, deveremos considerar o modelo de solução particular proposto pelo professor na hora da prova ($y_p(x) = (A\sin x +  B\cos x)e^x$) e encontrar os coeficientes tais que $y''_p + y_p = \sin x \ e^x$ \begin{align*}
    y_p(x) &= (A\sin x + B\cos x)\ e^x \\
    y'_p(x) &= (A\sin x + B\cos x)\ e^x + (A\cos x - B\sin x)\ e^x \\
    &= ((A-B) \sin x + (A+B) \cos x)\ e^x \\
    y''_p(x) &= ((A-B) \sin x + (A+B) \cos x)e^x + ((A-B) \cos x +
    (-A-B) \sin x)\ e^x \\
    &= ((-2B) \sin x + (2A) \cos x)\ e^x \\
    y''_p + y_p &= ((-2B) \sin x + (2A) \cos x)\ e^x + (A\sin x + B\cos x)\ e^x \\
    &= ((A-2B) \sin x + (2A + B) \cos x)\ e^x = \sin x\ e^x \\
    \begin{cases}
        A - 2B = 1 \\
        2A + B = 0
    \end{cases} &\Longleftrightarrow
    \begin{cases}
        5A = 1 \\
        A - 2B = 1
    \end{cases} \Longleftrightarrow
    A = \frac{1}{5} \ \land \ B = \frac{-2}{5} \\
    y_p &= \frac{\sin x}{5} + \frac{-2 \cos x}{5} \\
    y &= y_h + y_p = \left(c_1 + \frac{1}{5}\right)\ \sin x + \left( c_2 - \frac{2}{5}\right)\ \cos x 
\end{align*}

\newpage

\section*{\E{3} (2,5)}

\hspace{5mm} a) \textbf{(1,0)} Encontre a solução geral de $$y''' - 3y'' + 3y' -y = 0$$

b) \textbf{(1,5)} Encontre a solução geral de $$y''' - 3y'' + 3y' -y = 2x + e^x$$

\Resolve

\textbf{\IT{A}}Temos a equação característica $x^3 - 3x^2 + 3x - 1 = 0 \Longleftrightarrow (x-1)^3 = 0$ da qual $\lambda = 1$ é raís tripla, logo $$y_h(x) = c_1 e^x + c_2 xe^x + c_3x^2e^x$$

\textbf{\IT{B}} Procuramos uma solução particular do formato $$y_p(x) = Ax + B + Cx^3e^x$$ e através do calculo das derivadas desse modelo de solução, descobriremos os valores dos coeficientes. \begin{align*}
    y'_p(x) &= A + Ce^x(x^3 + 3x^2) \\
    y''_p(x) &= Ce^x(x^3 + 6x^2 + 6x) \\
    y'''_p(x) &= Ce^x(x^3 + 9x^2 + 18x + 6) \\
    y'''_p(x) &- 3y''_p(x) + 3 y'_p(x) - y_p(x) = 2x + e^x \\
    -Ax + (3A-B) + 6Ce^x &= 2x + e^x \Rightarrow A = -2 \ \land B = -6 \ \land C = \frac{1}{6} \\
    y_p(x) &= -2x -6 + \frac{1}{6}x^3e^x \\
    y &= \underbrace{c_1 e^x + c_2 xe^x + c_3x^2e^x}_{y_h(x)} \underbrace{-2x -6 + \frac{1}{6}x^3e^x}_{y_p(x)}
\end{align*}

\newpage

\section*{\E{4} (2,5)}{Enunciado a ser inserido depois, pois não lembro dele em detalhes}

\Resolve

\textbf{\IT{A}} Claramente $y_1 = 1$ é uma solução (poderia ser qualquer outra função constante) de $y'' - \underbrace{\frac{1}{x}}_{p(x)}y' = 0$. Para encontrar a outra solução:$$\left|\begin{array}{cc}
1 & y \\
0 & y'
\end{array}\right| = e^{\int \frac{dx}{x}} \Longleftrightarrow y' = e^{\ln x} = x \Longleftrightarrow y_2 = \frac{x^2}{2}$$

Solução geral: $y_h(x) = c_1 + c_2\frac{x^2}{2}$

\textbf{\IT{B}} $$y'' -\frac{1}{x}y' = 3x^2$$ a solução particular está na forma $$y_p(x) = \underbrace{1}_{y_1}*v_1 + \underbrace{\frac{x^2}{2}}_{y_2}*v_2$$ onde: \begin{align*}
    v_1 &= - \int \frac{y_2 * \overbrace{r(x)}^{3x^2}}{W(x)} dx \\
    v_2 &= \int \frac{y_1*r(x)}{W(x)}dx \\
    W(x) &= \left| \begin{array}{cc}
    1 & \frac{x^2}{2} \\
    0 & x
    \end{array}\right| = x \\
    v_1 &= -\int \frac{\frac{x^2}{2}*3x^2}{x}dx = -\frac{3}{2} \int x^3 \ dx= -\frac{3x^4}{8} \\
    v_2 &= \int \frac{3x^2}{x}dx = \int 3x\ dx = \frac{3x^2}{2} \\
    y_p(x) &= -\frac{3x^4}{8} + \frac{3x^4}{4} = \frac{3x^4}{8} \\
    y &= c_1 + c_2\frac{x^2}{2} + \frac{3x^4}{8}
\end{align*}


\chapter{Correção P SUB}
\label{C:SUB}

\section*{\E{1} (2,5)}

\newpage

\section*{\E{2} (2,5)}

\newpage

\section*{\E{3} (2,5)}

\newpage

\section*{\E{4} (2,5)}

\part{Referências}
\label{ref-3}

Antes de mais nada, eu gostaria de dizer que peguei as informações sobre as derivadas e as integrais aqui presentes no livro \cite[p.p. 199-201]{referencia}. 

\section*{Tabela de Paridades}

\hspace{5mm}Dadas funções $f(x)$ e $g(x)$, eu não acredito  \footnote{melhor perguntar para o professor} que possa se pode deduzir a paridade de $f(x) \circ g(x)$ (onde $\circ$ é uma operação binária) se essas operações forem soma ou subtração, mas quando se trata de produto ou quociente, esse trabalho é facilitado.

\begin{table}[ht]
\centering
\begin{tabular}{|c|c|c|}
\hline
$f(x) \circ g(x)$ & par & ímpar \\ \hline 
 par &  par & ímpar \\
 ímpar & ímpar & par \\ \hline
\end{tabular}
\caption{Paridade da função $h(x) = f(x) \circ g(x)$}
\label{tab:par1}
\end{table}

Vale lembrar que uma dada função $f(x)$ é par se $f(-x) = f(x)$ e é impar se $f(-x) = - f(x)$. Aqui vão alguas poucas funções de exemplo. Aproveito a chance para dizer que podem contribuir com mais exemplos. 

\begin{table}[ht]
\centering
\begin{tabular}{|c|c|}
\hline
par & ímpar \\ \hline 
 $x^n | n \in \mathds{Z}_2$ &  $x^n | n \in \mathds{Z} - \mathds{Z}_2$ \\
 $\cos x$ & $\sin{x}$\\ 
 $|x|$ & $ \tan x = \frac{\sin x}{\cos x}$ \\ \hline
\end{tabular}
\caption{Exemplos de funções pares e ímpares}
\label{tab:par2}
\end{table}

\section*{Regras de Diferenciação\footnote{csc = cossecante; cot = cotangente}}
\begin{multicols}{2}
\begin{enumerate}
\item $\displaystyle{\frac{d}{dx} [f(x)g(x)] = f'(x)g(x) + g'(x)f(x)}$
\item $\displaystyle{\frac{d}{dx} \left[\frac{f(x)}{g(x)}\right] = \frac{f'(x)g(x) - g'(x)f(x)}{(g(x))^2}}$
\item $\displaystyle{\frac{d}{dx} f(g(x)) = f'(g(x))g'(x)}$
\item $\displaystyle{\frac{d}{dx} (a^x) = a^x \ln{a}}$
\item $\displaystyle{\frac{d}{dx} \log_a{x} = \frac{1}{x \ln a}}$
\item $\displaystyle{\frac{d}{dx} (\sin x) = \cos x}$
\item $\displaystyle{\frac{d}{dx} (\cos x) = -\sin x}$
\item $\displaystyle{\frac{d}{dx} (\tan x) = \sec^2 x}$
\item $\displaystyle{\frac{d}{dx} (\csc x) = -\csc x \cot x}$
\item $\displaystyle{\frac{d}{dx} (\sec x) = \sec x \tan x}$
\item $\displaystyle{\frac{d}{dx} (\cot x) = -\csc^2 x}$
\item $\displaystyle{\frac{d}{dx} (\arcsin x) = \frac{1}{\sqrt{1-x^2}}}$
\item $\displaystyle{\frac{d}{dx} (\arccos x) = -\frac{1}{\sqrt{1-x^2}}}$
\item $\displaystyle{\frac{d}{dx} (\arctan x) = \frac{1}{1+x^2}}$
\end{enumerate}
\end{multicols}

\section*{Lista de Primitivas\footnote{para as expressões 6 e 7, $n \geq 1$ enquanto, para a expressão 8, $n > 1$}}
\begin{multicols}{2}
\begin{enumerate}
\item $\displaystyle{\int u dv = uv - \int v du}$
\item $\displaystyle{\int \frac{du}{u} = \ln |u| + C}$
\item $\displaystyle{\int e^u du = e^u + C}$
\item $\displaystyle{\int a^u du = {a^u \over \ln a}}$
\item $\displaystyle{\int \tan u \ du = \ln |\sec x| + C}$
\item $\displaystyle{\int \sin^n u \ du = -{1 \over n}\sin^{n-1}u\cos  u + \frac{n-1}{n}\int \sin^{n-2} u \ du}$
\item $\displaystyle{\int \cos^n u \ du = {1 \over n}\cos^{n-1}u\sin  u + \frac{n-1}{n}\int \cos^{n-2} u \ du}$
\item $\displaystyle{\int \tan^n u \ du = \frac{1}{n-1}\tan^{n-1}u - \int \tan^{n-2} u \ du \ n > 1}$
\item $\displaystyle{\int \ln u \ du = u \ln u - u + C}$
\item $\displaystyle{\int \frac{du}{u \ln u} = \ln |\ln u| + C}$
\end{enumerate}
\end{multicols}

\begin{thebibliography}{9}

\bibitem{referencia}
  James Stewart, %autor
  Cálculo, Volume 1, %título
  São Paulo, SP, %cidade
  5ª edição, %edição
  2008.%ano

\end{thebibliography}

\end{document}

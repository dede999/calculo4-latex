\documentclass[12pt,openany]{book}
\usepackage[utf8]{inputenc}
\usepackage[portuguese]{babel}
\usepackage[margin = 2cm]{geometry}
\usepackage[colorlinks=false]{hyperref}
\newtheorem{theorem}{Teorema}[section]
\newtheorem{lemma}{Lema}[section]
\newtheorem{corollary}{Corolário}[theorem]
\newtheorem{definition}{Definição}[section]
\newtheorem{demonstration}{Demonstração}
\newtheorem{obs}{Observação}
\newtheorem{prop}{Proposição}
\usepackage[rightcaption]{sidecap}
\usepackage{graphicx}
\usepackage{amsmath}
\usepackage{dsfont}
\newcommand{\LI}[1][n]{\lim_{{#1} \rightarrow \infty}}
\newcommand{\soma}[2][n]{\sum_{{#1} = #2}^\infty}

%modelo de início de capítulo encontrado no link https://pt.sharelatex.com/learn/Defining_your_own_commands
\makeatletter 
\def\thickhrulefill{\leavevmode \leaders \hrule height 1.2ex \hfill \kern \z@}
\def\@makechapterhead#1{
  \vspace*{10\p@}%
  {\parindent \z@ \centering \reset@font
        \thickhrulefill\quad 
        \scshape\bfseries\textit{\@chapapp{}  \thechapter}  
        \quad \thickhrulefill
        \par\nobreak
        \vspace*{10\p@}%
        \interlinepenalty\@M
        \hrule
        \vspace*{10\p@}%
        \Huge \bfseries #1 \par\nobreak
        \par
        \vspace*{10\p@}%
        \hrule
        \vskip 40\p@
  }}


\title{Calculo IV - IME 2015}
\author{André Luiz A Silveira}
\date{Segundo Semestre -- 2015}

\begin{document}

\maketitle

\tableofcontents

\chapter*{Introdução}
\label{chap:c0}

\hspace{5mm} Eu criei esse livrinho pra ter um resumo e material de estudo. Espero que dê os resultados esperados para mim e vc também que estiver lendo esse material. 

Esse PDF não tem intenção nenhuma de ser uma segunda lousa, quanto menos substituir as aulas, pois selecionei partes que são importantes para mim, mas podem não ser para outros. Essa é unicamente uma forma de me organizar e estudar, e que PODE ajudar a outras pessoas em outros momentos.\\

Professor: 	Antonio Carlos Asperti (\href{mailto:asperti@ime.usp.br}{asperti@ime.usp.br})

Livro usado: Um curso de Cáluclo -- Volumes 2 e 4

Provas:
\begin{description}
\item[P1] 14-Set (2ªf - peso 1)
\item[P2] 21-Out (4ªf - peso 1)
\item[P3] 02-Dez (4ªf - peso 2)
\item[P SUB] 07-Dez (2ªf - peso ?)
\end{description}

\begin{figure}
\centering
\includegraphics[width=1.05\textwidth]{lucky}
\caption{Boa Sorte, pessoal}
\label{fig:lucky}
\end{figure}



\chapter{Integrais Impróprias}
\label{chap:c1}

\hspace{5mm} Assunto tratado entre 03/08/2015 e 05/08/2015
\section{Conceitos}
\label{sec:s11}

\hspace{5mm} São expressões do tipo $\displaystyle{\int_a^{+ \infty}f(x)dx}$, $\displaystyle{\int_{-\infty}^{a} f(x)dx}$, $\displaystyle{\int_{-\infty}^{+ \infty}f(x)dx}$\\
\vspace{15pt}

Definição: Seja $f:{[a,+\infty)}\rightarrow \mathds{R}$ e suponha que $f$ seja integrável em $[a,t]$ para todo $t>a$. Assim definimos $$ \int_a^{+\infty}f(x)dx = \lim_{t\rightarrow\infty}\int_a^t f(x)dx$$ desde que exista o limite e seja finito.

Neste caso, o o limite é a \underline{integral imprópria} de $f$ estendido ao intervalo $[a,+\infty)$.

Se o limite existir e for finito, diremos se tratar de uma \underline{integral convergente}. Caso contrário, é uma \underline{integral divergente}.

Exemplo 1:
\begin{align*}
\int_1^{+\infty}\frac{1}{x^2}dx &= \lim_{t\rightarrow + \infty}\int_1^t \frac{dx}{x^2}=\lim_{t\rightarrow + \infty}\left[\frac{x^{-2+1}}{-2+1}\right]_1^t = \lim_{t\rightarrow + \infty}\left[-\frac{1}{x}\right]_1^t = -\lim_{t\rightarrow + \infty}\left[\frac{1}{t}-1\right]=1
\end{align*}

Exemplo 2:
\begin{align*}
\int_1^{+\infty}\frac{dx}{x}= \lim_{t\rightarrow + \infty}\int_1^t\frac{dx}{x}= \left.\lim_{t\rightarrow + \infty}\ln x \right|_1^t = \lim_{t\rightarrow + \infty}\ln t - 0 = +\infty
\end{align*}

\section{Função definida por uma integral}
\label{sec:s12}

\hspace{5mm} Seja $f \in (-\infty, +\infty) \rightarrow \mathds{R}$ contínua e tal que $\displaystyle{\int_{-\infty}^x f(t)dt}$ exista $\displaystyle{\forall x \in \mathds{R}}$. Podemos afirmar que $F:\mathds{R}\rightarrow\mathds{R}$, $\displaystyle{F(x)=\int_{-\infty}^x f(t)dt}$. Fixe $a \in \mathds{R}$. Então:

\begin{align*}
\int_a^x f(t)dt &= \int_a^b f(t)dt + \int_b^x f(t)dt \\
F(x) = \lim_{a\rightarrow - \infty}\int_a^xf(t)dt &= \underbrace{\int_{-\infty}^b f(t)dt}_{constante} + \underbrace{\int_b^x f(t)dt}_{H(x)}\\
H(x) &= \int_b^x f(t)dt \\
F(x) &= \underbrace{\int_{-\infty}^b f(t)dt}_{constante}+ \underbrace{H(x)}_{derivavel} 
\end{align*}

$H$ é derivável e $H'(x) = f(x)$ (T.F.C.), logo $F$ é derivável e $F'(x)=H'(x)=f(x)$

\vspace{20pt}
Exemplo 1: Dada $f:\mathds{R}\rightarrow\mathds{R}$, $\hbox{f}(t)
= \left\{ \begin{array}{rll}
2 & \hbox{se} &  |t| \leq 1 \\
0 & \hbox{se} &  |t|  > 1 \\
\end{array}\right.$, esboçe o gráfico de $\displaystyle{F(x)=\int_{-\infty}^x f(t)dt}$
\begin{SCfigure}
\centering
\includegraphics[width=0.5\textwidth]{enun}
\caption{gráfico de f(x): exemplo 1}
\label{fig:en1}
\end{SCfigure}

Resolução:
\begin{itemize}
\item se $\displaystyle{x<-1 \Rightarrow F(x) = \int_{-\infty}^x 0\hspace{1mm}dt = 0}$
\item se $\displaystyle{ -1 \leq x < 1 \Rightarrow F(x) = \int_{-\infty}^{-1} f(t) dt + \int_{-1}^x f(t)dt = 0 + \int_{-1}^x \hspace{1mm}dt = 2x+2}$
\item se $\displaystyle{ x \geq 1 \Rightarrow F(x) = \int_{-\infty}^{-1} f(t) dt + \int_{-1}^1 f(t)dt + \int_1^{+\infty}f(t)dt = 0 + 4+ \int_1^x 0 \hspace{1mm}dt = 4}$
\end{itemize}

$\hbox{F}(x)
= \left\{ \begin{array}{rll}
0 & \hbox{se} &  x <-1 \\
2x+2 & \hbox{se} &  -1 \leq x < 1 \\
4 & \hbox{se} & x \geq 1 \\
\end{array}\right.$

\begin{figure}[h]
\centering
\includegraphics[width=0.5\textwidth]{resp}
\caption{Gráfico Resultate da Operação}
\label{fig:res1}
\end{figure}
\vspace{20pt}

\hspace{5mm}Tente você mesmo: Esboçe o gráfico da função $\displaystyle{F(x) = \int_{-\infty}^x f(t)dt}$, onde $\hbox{f}(t)
= \left\{ \begin{array}{rll}
\displaystyle{\frac{1}{t}} & \hbox{se} &  t \geq 1 \\
0 & \hbox{se} &  t < 1 \\
\end{array}\right.$

\section{Critério de Comaparação}
\label{sec:13}

\hspace{5mm} Ferramenta Importante na classificação de integrais impróprias

\begin{theorem}[Critério de Comparação]
Sejam $f$ e $g$ funções integráveis em $[a;t]$ para todo $t\geq a$ e tais que $0\leq f(x)\leq g(x)\ \forall x \geq a$.  Assim temos que:
\begin{itemize}
\item $\displaystyle{\int_a^{\infty} g(x)dx}$ é convergente $\Rightarrow$ $\displaystyle{\int_a^{\infty} f(x)dx}$ é convergente
\item $\displaystyle{\int_a^{\infty} f(x)dx}$ é divergente $\Rightarrow$ $\displaystyle{\int_a^{\infty} g(x)dx}$ é divergente
\end{itemize}
\vspace{10pt}
\begin{corollary}
Seja $f:[a;+\infty) \rightarrow \mathds{R}$ integrável em $[a;t]$ para todo $t \geq a$, então vale que se $\displaystyle{\int_a^{+\infty}|f(x)| dx}$ for convergente, então $\displaystyle{\int_a^{+\infty} f(x) dx}$ também será.
\end{corollary}
\end{theorem}
\vspace{10pt}

Suponha que $F(x)$ seja crescente em $[a;+\infty)$, prove que $\displaystyle{\lim_{x \rightarrow +\infty} F(x) }$ será finito \footnote{nesse caso seria igual a $M = sup\{F(x)| x \in [a;+\infty)\}$ } ou não.

\begin{demonstration}
Sendo $f(x)$ integrável em $[a,t] \ \ \forall t \geq a$ e $f(x) \geq 0$, a função $\displaystyle{F(x) = \int_a^x\ f(t)\ dt}$ é crescente. De fato, sejam $x_1$ e $x_2 \in [a,+\infty)$, então:

\begin{align*}
F(x_2)-F(x_1) &= \int_a^{x_2} f(t)dt - \int_a^{x_1} f(t)dt \\
&= \int_{x_1}^{x_2} f(t)dt \geq 0 \Rightarrow F(x_2) \geq F(x_1)
\end{align*}

Como $\displaystyle{\int_a^{+\infty} g(t)dt}$ é convergente, então $\displaystyle{\exists \lim_{t \rightarrow +\infty} \int_a^t g(x)\ dx} $ (será $M \geq 0$). E como $0 \leq f(x) \leq g(x)$, então $\forall t \geq a \Rightarrow F(x) = \int_a^t f(x)dx \leq \int_a^t g(x)dx \leq \int_a^{+\infty} f(x)dx = M$
\end{demonstration}
\vspace{10pt}

Pelo que se vê acima, $\displaystyle{\lim_{t\rightarrow+\infty} F(t) = \lim_{t\rightarrow+\infty} \int_a^t f(x)dx}$ exite e é finito. Logo, $\displaystyle{\int_a^{+\infty} f(t)dt}$ é \textbf{convergente}.

Interessante destacar que se tivermos a integral indefinida de $\displaystyle{f(x) =  \frac{g(x)}{x^{\alpha}}}$ com $g(x)$ limitado \footnote{$\displaystyle{\lim_{x \rightarrow +\infty} g(x)} = L \neq \infty $}, então sua classificação depende únicamente da classificação de 1/x^\alpha 

\section{Descobrindo se é ou não convergente -- Exemplo de exercícios}
\label{sec:s14}

\subsection*{Exemplo 1}
\label{sub:ex141}

Mostrar que $\displaystyle{\int_0^{+\infty} e^{-x} \cos(\sqrt x) dx} $ é  convergente.
\begin{align*}
f(x) &= e^{-x} \cos(\sqrt x) \\
|f(x)| &= e^{-x}  |\cos(\sqrt x)| \leq e^{-x} = g(x) \\
\end{align*}
Se $\displaystyle{\int_0^\infty \ g(x)dx}$ for convergente, $\displaystyle{\int_0^{+\infty}f(x)dx }$ também o será.
$$ \int_0^{+\infty}e^{-x}dx = \lim_{t \rightarrow +\infty} -e^{-x} |_0^t} = - \lim_{t \rightarrow +\infty} \left[\frac{1}{e^t}-1 \right] = 1$$

$\displaystyle{\int_0^{+\infty}g(x)dx }$ é convergente, logo $\displaystyle{\int_0^{+\infty}f(x)dx }$ também  é convergente.

\subsection{Exemplo 2}
\label{sub:ex142}

Mostre que $$ \int_1^{\infty} \frac{x^2+1}{x^3+1} $$ é divergente.

\begin{align}
\frac{x^2+1}{x^3+1} \geq \frac{x^2}{x^3+1} = \frac{1}{x+\frac{1}{x^2}} = \frac{1}{x(1+\frac{1}{x^2})}\\
x^2 \geq 1 \Rightarrow \frac{1}{x^2} \leq 1 \\
1 + \frac{1}{x^2} \leq 2 \Rightarrow \frac{1}{1+\frac{1}{x^2}} \geq \frac{1}{2}\\
\frac{x^2+1}{x^3+1} \geq \frac{1}{x}\left(\frac{1}{1+\frac{1}{x^2}}\right) \geq \frac{1}{2x}
\end{align}

Como $ \displaystyle{\int_1^{\infty} \frac{1}{2x}} $ é divergente, $\displaystyle{ \int_1^{\infty} \frac{x^2+1}{x^3+1}} $ também é

\subsection{Exemplo 3}
\label{sub:ex143}

\hspace{5mm}Seja $\alpha \in \mathds{R} | \alpha >0 $, então mostre que $\displaystyle{\int_1^\infty \frac{dx}{x^\alpha}} $:
\begin{enumerate}
\item [a.] é convergente se $\alpha > 1$\\
\item [b.] é divergente se $\alpha \in (0,1) $
\end{enumerate}

Para $0  < \alpha \neq 1$ temos que $$\displaystyle{\int_1^\infty \frac{dx}{x^\alpha} = \lim_{t \rightarrow +\infty} \int_1^t x^{-\alpha}\ dx = \lim_{t \rightarrow +\infty} \left[ \frac{x^{-\alpha + 1}}{-\alpha + 1}\right]_1^t = \lim_{t \rightarrow +\infty} \left( \frac{t^{1-\alpha}-1}{1-\alpha}\right)}$$

\begin{enumerate}
\item [a.] $\alpha > 1 \Rightarrow 1 - \alpha < 0$. Quando t tende a infinito, $t^{1-\alpha}$ tende a 0: assim temos uma integral indefinida convergente $$\displaystyle{\int_1^\infty \frac{dx}{x^\alpha}} = \frac{1}{\alpha - 1}$$
\item [b.] $0 < \alpha < 1 \Rightarrow 1 - \alpha > 0$. Quando t tende a infinito, $t^{1-\alpha}$ tende a infinito também: assim temos uma integral indefinida divergente.
\end{enumerate}

\subsection{Exemplo 4}
\label{sub:ex144}

\hspace{5mm} Verifique se são convergentes:
\begin{enumerate}
\item [a.] $\displaystyle{\int_1^{\infty} \frac{\ln x}{x \ln \left(x+1\right)} dx}$
\item [b.] $\displaystyle{\int_{10}^{\infty}\frac{x^5-3}{\sqrt{x^{20}+x^{10}+1}} dx}$
\end{enumerate}

\begin{enumerate}
\item [a.] Seja  $\displaystyle{g(x) = \frac{\ln x}{\kn {x+1}}}$ uma função limitada ($\displaystyle{\lim_{x \rightarrow +\infty} \frac{\ln x}{\ln {x+1}} = 1}$) e o fato de que $\displaystyle{\int_1^{\infty} \frac{1}{x}}$ é divergente, então a integral em questão é divergente.
\item [b.] $$f(x) = \frac{x^5-3}{\sqrt{x^{20}+x^{10}+1}} = \frac{x^5\left(1-\frac{3}{x^5}\right)}{\sqrt{x^{20}(1+x^{-10}+x^{-20})}}= \underbrace{\frac{1}{x^5}}_{g(x)}\ \underbrace{\frac{1-\frac{3}{x^5}}{1+x^{-10}+x^{-20}}}_{h(x)}} = g(x)*h(x) $$ \\ $$\lim_{x \rightarrow +\infty} h(x) = 1$$ \\ $\displaystyle{\int_{10}^{\infty} g(x)\ dx}$ é convergente. Logo, $\displaystyle{\int_{10}^{\infty} f(x)\ dx}$ é convergente.
\end{enumerate}

\chapter{Sequências e Séries Numéricas} Aulas dadas entre 10/08/2015 e 26/08/2015
\label{chap:c2}

\section{Sequências e limites de sequências}
\label{sec:s21}

\hspace{5mm} Uma sequência  é uma função \footnote{$\mathds{N}^s$ geralmente é da forma $\left\{ n \in \mathds{N} | n \geq q , \ q \ fixo \right\} $ } \begin{align*}
f: \underbrace{\mathds{N}^s \subset \mathds{N}}_n \rightarrow \underbrace{\mathds{R}}_{a_n} \\
(a_n) \ n \in \mathds{N} \\
a_n &= n,\ para \ n \geq 0 \\
a_n &= \frac{1}{n} ,\ para \ n \geq 1 \\
\end{align*}

O que acontece quando n tende a infinito?

\begin{enumerate}
\item $(a_n) \ n \in \mathds{N} a_n = n$
\item $\frac{1}{2}, \ \frac{1}{4}, \ \frac{1}{8}, \ \frac{1}{16}, \ \hdots a_n = \frac{1}{2^n}, \ \  n\geq 1 $
\item 1, -1, 1, -1 , 1 , 1 $\hdots a_n = (-1)^n $ Vai para 0, 1 ou -1?
\end{enumerate}
\begin{definition}[Convergência]
Dizemos que a sequência $(a_n)$ converge para o limite $L \in \mathds{R}$ dado $\epsilon > 0$, existe $n_0 \in \mathds{N}$ tal que $$ n \geq n_0 \Rightarrow |a_n - L| < \epsilon \Longleftrightarrow L-\epsilon < a_n < L + \epsilon $$
\end{definition}

$\displaystyle{\lim_{n \rightarrow + \infty} a_n = + \infty} $ acontece quando $\forall \ R > 0 $ existe $n_0 \in \mathds{N} $ tal que $$n \geq n_0 \Rightarrow a_n > R$$

$\displaystyle{\lim_{n \rightarrow + \infty} a_n = - \infty} $ acontece quando $\forall \ R > 0 $ existe $n_0 \in \mathds{N} $ tal que $$n \geq n_0 \Rightarrow a_n < R$$

Voltando aos exexmplos do começo do capítulo:
\begin{enumerate}
\item \begin {align*} 
a_n &= n \\
\lim_{n \rightarrow + \infty} a_n &= + \infty\\
\end{align*}
\item \begin{align*}
a_n &= \frac{1}{2^n}\\
\lim_{n \rightarrow + \infty} a_n &= 0\\
\end{align*}
Vamos verificar esse fato. Dado $\epsilon > 0, \ \exists \ n_0 \in \mathds{N}$ tal que $\displaystyle{ n \geq n_0  \Rightarrow \left\| \frac{1}{2^n} - 0 \right\| <  \epsilon \Longleftrightarrow \frac{1}{2^n} < \epsilon \Longleftrightarrow 2^n > \frac{1}{\epsilon}}$.   Basta tomarmos $n_0 \in \mathds{N}$ tal que $\displaystyle{2^{n_0} > \frac{1}{\epsilon}}$ e $$ n \geq n_0 \Rightarrow 2^n \geq 2^{n_0} > \frac{1}{\epsilon} \Rightarrow \frac{1}{2^n} < \epsilon$$
\item $a_n = (-1)^n$. O limite dessa expressão não pode ser $\pm \infty$ pois para n tendendo a infinito, teremos $-1 \leq a_n \leq 1 $.  Seja $\displaystyle{0 < \epsilon < \frac{1}{\epsilon} } $. Dado qualquer $L \in \mathds{R}}$, o intervalo $]L- \epsilon, L+ \epsilon[ $ tem diâmetro $\displaystyle{2\epsilon < \frac{2}{3}}$. 
Portanto, esse intervalo não comporta  todos os elementos da sequência, logo $$ \nexists \lim_{n \rightarrow + \infty} (-a)^n$$
\item [\textbf{Adendo}] Seja $(a_n) \ n \in \mathds{N}$  e sejam $n > 0 \land s > 1$ com s fixo. Assim teremos $\displaystyle{\lim_{n \rightarrow + \infty} \frac{1}{n^s} = 0}$
\end{enumerate}

\textbf{Fatos Importantes}
\hspace{5mm} Sejam $(a_n)$ $(b_n)$ duas sequências de números de números reais. Então \begin{enumerate}
\item se $a_n \rightarrow a $ e $b_n \rightarrow b$ $\Rightarrow a_n + b_n \rightarrow a + b$
\item se $a_n \rightarrow a $ e $b_n \rightarrow b$ $\Rightarrow a_nb_n \rightarrow ab$
\item se $a_n \rightarrow a $ e $\lambda \in \mathds{R}$ $\Rightarrow \lambda a_n \rightarrow \lambda a$
\item se $a_n \rightarrow a $ e $b_n \rightarrow b \neq 0$, então $b_n \neq 0$ para n suficientemente grande, então $\displaystyle{\frac{a_n}{b_n} = \frac{a}{b}}$
\end{enumerate}

\begin{lemma}[Critério do Confronto]
Sejam $(a_n)$ $(b_n)$ $(c_n)$ sequências tais que $a_n \leq b_n \leq c_n$. Suponha que existam os limites existam e que seja verdade que $\displaystyle{\lim_{n \rightarrow + \infty} a_n = L = \lim_{n \rightarrow + \infty} c_n}$. Então: $$\lim_{n \rightarrow + \infty} b_n = L$$
\end{lemma}

\begin{demonstration}
Seja um $\epsilon > 0$ suficientemente pequeno. Como $a_n$ e  $c_n$ tendem a $L$, existe $n_0 \in \mathds{N}$ tal que:
\[ n \geq n_0) \Rightarrow
  \begin{cases}
    |a_n - L| < \epsilon \quad \Longleftrightarrow  & \quad L - \epsilon < a_n < L +\epsilon \\
    |c_n - L| < \epsilon \quad \Longleftrightarrow  & \quad L - \epsilon < c_n < L +\epsilon \\
  \end{cases}
\]

Então $n \geq n_0 \Rightarrow L - \epsilon < a_n \leq b_n \leq c_n < L + \epsilon$ ou seja $n \geq n_0 \Rightarrow L- \epsilon < b_n < L + \epsilon \Rightarrow \displaystyle{\lim_{n \rightarrow + \infty} b_n = L}$
\end{demonstration}

\begin{lemma}
Seja $f$ uma função definida num intervalo $I \subset \mathds{R}$, exceto em $c \in I$ e suponha que exista $\displaystyle{\lim_{x \rightarrow c} f(x) = L}$. Seja $(a_n)$ uma sequência que satisfaça \begin{itemize}
\item [a.] $a_n \in I$
\item [b.] $a_n \neq c $
\item [c.] $\displaystyle{\lim_{n \rightarrow + \infty} a_n = c} $
\end{itemize}
Assim, a sequencia $(b_n)_{n \in \mathds{N}} =(f(a_n))_{n \in \mathds{N}}$ é tal que $\displaystyle{\lim_{n \rightarrow + \infty} b_n = L}$ \footnote{Isso equivale a dizer que $\displaystyle{\lim_{n \rightarrow + \infty} f(a_n) = L} $}
\end{lemma}

\begin{demonstration}Dado $\epsilon > 0$ e $\displaystyle{\lim_{x \rightarrow c} f(x) = L}$, existe $\delta > 0$ tal que $$ |x-c| < \delta \Rightarrow |f(x) - L| < \epsilon $$ \hspace{5mm} Como $a_n \rightarrow c$, para tal $\delta > o$ existe $n_0 \in \mathds{N}$ tal que $n \geq n_0 \rightarrow |a_n - c| < \delta$. Então: $$n \geq n_0 \rightarrow |a_n - c| < \delta \Rightarrow |f(a_n) - L| < \epsilon$$ $$ \lim_{n \rightarrow +\infty} f(a_n) = L$$
\end{demonstration}

\begin{obs} Se $f$ está defiida entre $[a;+\infty)$, se $a_n \rightarrow +\infty$ e $\displaystyle{\lim_{x \rightarrow +\infty} f(x) =L}$, então $$\lim_{n \rightarrow +\infty} f(a_n) =L$$
\end{obs}

\section{Exercícios da aula (10/08)}
\label{sec:s22}

\subsection*{Exemplo 1}
\label{subsec:ex221}
\hspace{5mm}Calcular $$\lim_{n \rightarrow +\infty} \frac{(5n-3)^3}{n(n^2+1)}$$
Desenvolvendo o binio de cima, temos $(5n-3)^3 = 125n^3-75n^2+45n -27$. Assim temos $a_n$ como na forma abaixo. Se dividirmos numerador e denominador por $n^3$, ficará fácil determinar o limite em questão.
\begin{align*}
a_n &= \frac{125n^3-75n^2+45n -27}{n^3+n} =  \frac{125 - \frac{75}{n} + \frac{45}{n^2} - \frac{27}{n^3}}{1 + \frac{1}{n^2}} \\
\lim_{n \rightarrow + \infty} a_n &= \lim_{n \rightarrow + \infty} \frac{125 - \frac{75}{n} + \frac{45}{n^2} - \frac{27}{n^3}}{1 + \frac{1}{n^2}} = 125\\
\end{align*}

\subsection*{Exemplo 2}
\label{subsec:ex222}
\hspace{5mm}Calcular $$\lim_{n \rightarrow +\infty} (\sqrt{n+1} - \sqrt{n})$$
\footnote{$\sqrt{n+1} + \sqrt{n} \geq 2\sqrt{n} \Longleftrightarrow \frac{1}{\sqrt{n+1} + \sqrt{n}} \leq \frac{1}{2\sqrt{n}} $}
\begin{align*}
0 \leq a_n = \sqrt{n+1} - \sqrt{n} &= \frac{(\sqrt{n+1} - \sqrt{n})(\sqrt{n+1} + \sqrt{n})}{\sqrt{n+1} + \sqrt{n}} \\
\frac{(n+1) - n}{\sqrt{n+1} + \sqrt{n}} &= \frac{1}{\sqrt{n+1} + \sqrt{n}} \leq \frac{1}{2\sqrt{n}}\\
0 \leq a_n \leq \frac{1}{2\sqrt{n}} & \land \lim_{n \rightarrow +\infty}\frac{1}{2\sqrt{n}} = 0  \\
\end{align*}
\hspace{5mm} Logo, $\displaystyle{\lim_{n \rightarrow +\infty} (\sqrt{n+1} - \sqrt{n}) = 0}$

\subsection*{Exemplo 3}
\label{subsec:ex223}
\hspace{5mm}Demonstrar que $$\lim_{n \rightarrow +\infty} a_n = 1$$ onde $a_n = \sqrt[n]{n},\ n \geq 1$

$a_n = \sqrt[n]{n} = 1 + b_n, \ b_n > 0$, então $a_n \rightarrow 1 \Longleftrightarrow b_n \rightarrow 0$.
$$n = a_n^n =(1+ b_n)^n = 1 + nb_n + \frac{n(n-1)}{2}b_n^2 + \hdots + b_n^n \geq 1 + \frac{n(n-1)}{2}b_n^2$$

Logo $\displaystyle{n-1 \geq \frac{n(n-1)}{2}b_n^2 \Rightarrow b_n^2 \leq \frac{2}{n} \Rightarrow 0 < b_n \leq \frac{\sqrt{2}}{\sqrt{n}}}$

Pelo critério de compração, $b_n \rightarrow 0$ e $a_n \rightarrow 1$.

\subsubsection{Pensando de uma outra forma} Tome $f: [1;+\infty) \rightarrow \mathds{R} \ \ f(x) = x^{\frac{1}{n}}$

\begin{align*}
f(x) = x^{\frac{1}{x}} &= e^{\ln x^{\frac{1}{x}}} =   e^{\frac{\ln x}{x}} \Longleftrightarrow \\
\lim_{x \rightarrow +\infty} \frac{\ln x}{x} &= 0 \Longleftrightarrow \\
\lim_{x \rightarrow +\infty} e^0 &= 1 \\
Logo, \ \sqrt[n]{n} & \rightarrow 1 \\
\end{align*}

\vspace{10mm}
Vamos extender esses conceitos ...

\subsubsection {Mostrar que $\displaystyle{\lim_{n \rightarrow +\infty} \sqrt[n]{a}} = 1 \ \ \forall \ a > 0$}
\label{sssec:ss2232}
\begin{itemize}
\item [\textbf{1}. $ a > 1 $] Para $b_n > 0$ e $a_n = \sqrt[n]{a} = 1 + b_n$ \\ $a = a_n^n = (1 + b_n)^n$. Nesse caso, proceder como no exemplo \ref{subsec:ex223} 
\item [\textbf{2}. $ 0 < a < 1 $] $ 0 < a < 1 \Longleftrightarrow \frac{1}{a} > 1$ Nesse caso teremos que $$ \sqrt[n]{\frac{1}{a}} \rightarrow 1 $$ $$\frac{1}{\sqrt[n]{a}}\rightarrow 1 $$
\end{itemize}

\subsubsection {Calcular $\displaystyle{\lim_{n \rightarrow +\infty} \sqrt[n]{a^n + b^n}}$}
\label{sssec:ss2233}
\hspace{5mm} Dados $0 < a \leq b$ e $n \geq 1$.

\begin{align*}
a_n &= \sqrt[n]{a^n+b^n} > \sqrt[n]{b^n} = b \\
a_n &= \sqrt[n]{a^n+b^n} \leq \sqrt[n]{2b^n} = b\sqrt[n]{2}\\
b &< a_n \leq b\underbrace{\sqrt[n]{2}}_{= 1}\\
\lim_{n \rightarrow + \infty} a_n &= b \\
\end{align*}

\section{Sequências Monótonas}
\label{sec:s23}

\hspace{5mm}Toda sequência convergente é limintada $$a_n \rightarrow a \Rightarrow \exists M \in \mathds{R} \ | \ \forall n \in \mathds{N} \ \ |a_n| \leq M$$ 

De fato, dado $\epsilon = \frac{1}{3}$, então $\exists n_0 \in \mathds{N}$ tal que $$ n \geq n_0 \Rightarrow |a_n - a| < \epsilon$$

Logo, se $n \geq n_0$, temos que $$ |a_n| = |a_n - a + a| \leq |a_n - a| + |a| < |a| + \frac{1}{3}$$

Seja $M = max \left\{ |a_0|, |a_1|, \hdots , |a| +\frac{1}{3} \right\}$. Logo $|a_n| \leq M$

\begin{definition}[Sequências Monótonas]
\hspace{5mm} Pode ser:
\begin{itemize}
\item \underline{Crescente} : se $n \geq m \Rightarrow a_n \geq a_m $
\item \underline{Decrescente} : se $n \geq m \Rightarrow a_n \leq a_m $
\end{itemize}
\end{definition}

\begin{lemma}
Se $(a_n)$ for crescente e limitada, então $\displaystyle{\exists \lim_{n\rightarrow+\infty} a_n}$\footnote{Analogamente, se a sequência for decrescente e limitada inferiormente. A demonstração também é análoga a do caso supra citado}
\end{lemma}

\begin{demonstration}
Se $ (a_n) $ for limitada superiormente, o conjunto$\left\{a_n, n \in \mathds{N}\right\}$ é limitado superiormente e postanto existe $ a = sup \left\{a_n, n \in \mathds{N}\right\}$.

Dado $\epsilon > 0$, existe $n_0 \in \mathds{N}$ existe $a - \epsilon < a_{n_0} \leq a$
\begin{align*}
\forall n \geq n_0 \Rightarrow a - \epsilon < a_{n_0} &\leq a_n \leq a \\
n \geq n_0 \Rightarrow a - \epsilon &< a_n \leq a \\
\lim_{n \rightarrow +\infty} a_n &= a \\
\end{align*}
\end{demonstration}

\begin{definition}[Sequências de Couchy (Critério de Couchy)]
Se $a_n \rightarrow a$, os termos vão ficando arbitrariamente próximos na medida em que n aumenta. 

Dado $\epsilon > 0 $, existe $n_0 \in \mathds{N}$ tal que $n \geq n_0 \Rightarrow |a_n - a| < \frac{\epsilon}{2}$. Logo, se $m, \ n \geq n_0$, então $|a_m - a_n| = |(a_m + a) - (a + a_n)| \leq |a_m - a| + |a_n -a| < \displaystyle{\frac{\epsilon}{2}+\frac{\epsilon}{2}} = \epsilon$

Uma sequência $(a_n)$ é uma sequência de Couchy, se para qualquer $\epsilon > 0$, existir $n_0 \in \mathds{N}$ tal que $m, \ n \geq n_0 \Rightarrow |a_m - a_n| < \epsilon$
\end{definition}

\begin{lemma}
Toda sequência de Couchy é convergente. 
\end{lemma}

\begin{demonstration}
\textbf{Diagmos que} $a_n$ satisfaz $m, \ n \geq n_0 \Rightarrow |a_m - a_n| < \epsilon$. De fato, para qualquer $k > 0$, existe $n_k \in \mathds{N}$ tal que para quaisquer $ m, \ n \geq n_k \Rightarrow |a_n - a_m| < \displaystyle{\frac{1}{k}}$.

Toamando $m = n_k$ fixo, temos: $$ n \geq n_k \Rightarrow |a_{n_k} - a_n| < \frac{1}{k} $$ ou mesmo $$ n \geq n_k \Rightarrow a_{n_k} - \frac{1}{k} < a_n < a_{n_k} + 1 $$. Seja $M = max\left\{ |a_0|, |a_1|, \hdots , |a_{n_k}|, |a_{n_k} - \frac{1}{k}|, |a_{n_k} + 1| \right\}$, então $|a_n| \leq M \ \ \forall \ n \in \mathds{N}$

\textbf{Mais algumas considerações:} Seja $A = \left\{ a_n, \ n \in \mathds{N} \right\}$ limitado e $A_n = \left\{ a_n, \ a_{n+1}, \ a_{n+2} \right\}$ também limitado. E por último, consideremos $L_n = sup \ A_n$ e $l_n = inf \ A_n$ onde $l_n \leq L_n$
\begin{align*}
A_n &= \left\{ a_n, \ a_{n+1}, \ a_{n+2}, \hdots \right\} \\
A_{n+1} &= \left\{ a_{n+1}, \ a_{n+2}, \hdots \right\} \\
L_n  &\geq L_{n+1} \\
l_n & \leq l_{n+1} \\
\end{align*}
\begin{itemize}
\item Se $L_n$ for decrescente, então $\displaystyle{\exists \lim_{n \rightarrow +\infty} L_n = L}$
\item Se $l_n$ for crescente, então $\displaystyle{\exists \lim_{n \rightarrow +\infty} l_n = l}$
\end{itemize}
\textbf{Por último ...} $\displaystyle{L = l = \lim_{n \rightarrow +\infty} a_n}$ Como $(a_n)$ é de Couchy, já sabemos que $\forall k \in \mathds{N}, \ k > 0 \Rightarrow \exists n_k \in \mathds{N}$ tal que $n \geq n_k \Rightarrow a_{n_k} - \frac{1}{k} < a_n < a_{n_k} + 1$. Logo, $a_{n_k} - 1 $ e $a_{n_k} + 1 $ são cotas inferior e superior do conjunto $A_{n_k} = \left\{ a_n \ | \ n \geq n_k \right\}$. Para todo $n \geq n_k$, temos $a_{n_k} - \frac{1}{k} \leq l_n \leq L_n \leq a_{n_k} + \frac{1}{k}$. Logo $$ 0 \leq L_n - l_n \leq (a_{n_k} + \frac{1}{k}) - (a_{n_k} - \frac{1}{k}) = \frac{2}{k}$$ Pelo critério de comparação, $ 0 < L - l \leq 0 \Longleftrightarrow L = l = a $. Para todo $n \in \mathds{N}$ temos $ l_n \leq a_n \leq L_n$. Pelo critério de comparação, sabemos que $$ \exists \lim_{n \rightarrow +\infty} a_n = a $$
\end{demonstration}

\section{Subsequências}
\label{sec:s24}
\hspace{5mm} Tome essas sequências:
\begin{align*}
a_n &= \frac{1}{n}, \ n \geq 1 \\
a_{n_k} &= \frac{1}{2k}, \\
a'_{n_k} &= \frac{1}{2k+1} \\
\end{align*}

Dada uma sequência $\displaystyle{a'_{n_k} = \frac{1}{k^2}}$ é uma subsequência de $a_n$ se $\displaystyle{\lim_{n \rightarrow +\infty} a_n = a \Longleftrightarrow \lim_{k \rightarrow +\infty} a_{n_k} = a} \ \forall \ (a_n)$ subsequência de $(a_n)$.

\textbf{Importante: }Sejam $(a_{n_k})$ e $(a'_{n_k})$ subsequências de $(a_n)$ e suponha que $\nexists \lim_{k \rightarrow +\infty} a_{n_k}$ ou mesmo que $$\lim_{k \rightarrow +\infty} a_{n_k} = a \neq \lim_{k \rightarrow +\infty} a'_{n_k} = b $$. Então $(a_n)$ não é convergente.

\section{Séries Numéricas: Definições e exmplos}
\label{sec:s25}
\hspace{5mm} Exemplos:
\begin{enumerate}
\item $\displaystyle{1 = \frac{1}{2} + \frac{1}{4} + \frac{1}{8} + \frac{1}{16} + \hdots}$ \label{ex:251}
\item $\displaystyle{1 = \frac{9}{10} + \frac{9}{100} + \frac{9}{1000} + \frac{9}{10000} + \hdots = 0,99999 \hdots} $ \label{ex:252}
\item $\displaystyle{1 -1 + 1 - 1 + 1 - 1 + \hdots} $ \label{ex:253} \\
Alguns fazem $(1-1) + (1-1) + (1-1) + \hdots = 0$. \\ Outros fazem $1 + (-1 + 1) + (-1 + 1) + (-1 + 1) + (-1 + 1) + \hdots = 1$. \\ Liebniz propos $\frac{1}{2} = 1 + 1 - 1 + 1 - 1 + 1 - 1 + \hdots$ \\
\end{enumerate}

Seja $(a_n)$ uma sequência  de números reais. A ela associamos a \underline{soma infinita}. $$\sum_{n = 0}^{\infty} a_n = a_0 + a_1 + a_2 + \hdots$$

Consideramos a sequência $(S_m)$ das \underline{somas parciais}
\begin{align*}
S_0 &= a_0 \\
S_1 &= a_0 + a_1 \\
S_2 &= a_0 + a_1 + a_2 \\
\vdots \\
S_m &= a_0 + a_1 + \hdots + a_m = \sum_{n=0}^m a_n \\
\end{align*}


\begin{definition}
Uma série $\sum_{n=0}^{\infty} a_n$ se diz \underline{convergente}. Se a sequência de somas parciais $(S_m)$ é convergente, neste caso, se $$ \lim_{n \rightarrow +\infty} S_n = S \Rightarrow \sum_{n=0}^{+\infty} a_n = S $$ onde S é a \underline{soma} da série. \footnote{$\displaystyle{\sum_{n=0}^{+\infty} a_n = \lim_{m \rightarrow +\infty} \left(\sum_{n=0}^m a_n\right)}$ }
\end{definition}


Voltando aos exemplos:
\begin{itemize}
\item \textbf{Exemplo \ref{ex:251}: } $\displaystyle{\frac{1}{2} + \frac{1}{4} + \frac{1}{8} + \frac{1}{16} + \hdots = \sum_{n=0}^{\infty} \frac{1}{2^{n+1}}}$
\item \textbf{Exemplo \ref{ex:252}: } $\displaystyle{\frac{9}{10} + \frac{9}{100} + \frac{9}{1000} + \frac{9}{10000} + \hdots = \sum_{n=0}^{\infty} \frac{9}{10^{n+1}}}$.
\end{itemize}
Esses são exemplos de \underline{séries geométricas} que são séries do tipo $\displaystyle{\sum_{n=0} a.r^n}$, onde $a, \ r \in \mathds{R}$. No primeiro caso, $a=r=\frac{1}{2}$, e no segundo $a = \frac{9}{10}, \ r = \frac{1}{10}$. \\ $r = 0 \Rightarrow $ converge a 0 \\ $r = 1 \Rightarrow $ diverge.

Supondo $r \neq 0 \land r \neq 1$. Somas parciais:
\begin{align*}
S_m &= a + a.r + a.r^2 + \hdots + a.r^m \\
r.S_m &= a.r + a.r^2 + \hdots + a.r^{m+1} \\
S_m - r.S_m &= a - a.r^{m+1} \\
S_m &= \frac{a(1-r^{m+1})}{1-r} \\
\end{align*}

Se $0 < |r| < 1$, então $$\lim_{m \rightarrow +\infty} S_m = \lim_{m \rightarrow +\infty} \frac{a(1-r^{m+1})}{1-r} = \frac{a}{1-r}$$ Neste caso, $\displaystyle{\sum_{n=0} a.r^n = \frac{a}{1-r}}$.
\begin{enumerate}
\item $\sum_{n=0}^{\infty} \frac{1}{2^{n+1}} = \frac{\frac{1}{2}}{1-\frac{1}{2}} = 1$
\item $\sum_{n=0}^{\infty} \frac{9}{10^{n+1}} = \frac{\frac{9}{10}}{1-\frac{1}{10}} = 1 $
\end{enumerate}

Se $|r| > 1$ a série diverge, pois $\displaystyle{\lim_{m \rightarrow +\infty} S_m = \lim_{m \rightarrow +\infty} \frac{a(1-r^{m+1})}{1-r} = \infty}$.

Exemplo \ref{ex:253}: 
\begin{align*} 
1 - 1 + 1 - 1 + 1 - 1 + 1 + \hdots \\
S_0 &= 1 \\
S_1 &= 1 - 1 = 0 \\
S_2 &= 1 - 1 + 1 = 1 \\
S_3 &= 1 - 1 + 1  - 1 = 0 \\
S_4 &= 1 - 1 + 1  - 1 + 1 = 1 \\
S_{2k} &= 1, \ k \in \mathds{N} \\
S_{2k+1} &= 0, \ k \in \mathds{N} \\
\lim_{k \rightarrow +\infty} S_{2k} &= 1 \\
\lim_{k \rightarrow +\infty} S_{2k+1} &= 0 \\
\end{align*}

$(S_m)$  tem duas subsequências convergindo a limites distintos, logo $\nexists \displaystyle{\lim_{n \rightarrow +\infty} S_{n}}$ nem $\displaystyle{\sum_{n=0}^{\infty} (-1)^n}$

\textbf{Propriedades Simples:}
\begin{enumerate}
\item Se $\sum a_n = a$ e $\sum b_n = b$, então $\sum (a_n+b_n) = a+b$
\item Se $\sum a_n = a$ e $\lambda \in \mathds{R}$, então $\sum (\lambda a_n) = \lambda a$
\item A convergência (ou não) de uma série não se altera se omitido um termo finito da soma.
\item $\displaystyle{\sum a_n}$ é convergente, então $\displaystyle{\lim_{n \rightarrow +\infty} a_n = 0}$ \footnote{Pode acontecer de o limite ser nulo, mas a série ser divergente}
\item $\displaystyle{\lim_{n \rightarrow +\infty} a_n \neq 0 \Longleftrightarrow \sum a_n}$ é divergente.
\end{enumerate}

\section{Séries de Termos Positivos -- Critérios de Comparação (Parte I)}
\label{sec:s26}
$$\sum_{n=0}^{\infty} a_n \ \ \ a_n > 0 $$

Convergência de sequêncio:
\begin{enumerate}
\item Sequências monótonas \label{261}
\item Critério de Couchy \label{262}
\end{enumerate}

\textbf{Item \ref{261}}: para a série $\sum a_n$ de termos positivos, temos:
\begin{align*}
S_0 &= a_0 \\
S_1 &= a_0 + a_1 > a_0 = S_0 \\
S_2 &= a_0 + a_1 + a_2 > S_1 \\
\end{align*}

A sequência $(S_m)$ é crescente. Podemos afirmar que uma série $\displaystyle{\sum_{n=0}^{\infty} a_n}$ é convergente, se e somente se, a sequência das somas parciais $(S_m)$ for limitada superiormente. \\

\textbf{Item \ref{262}: Critério de Comparação}: Sejam $\sum a_n$ e $\sum b_n$ séries de termos positivos e suponha que $a_n \leq b_n \ \forall \ n$. Então, se $\sum b_n$ converge, então $\sum a_n$ também converge. \footnote{De outra forma, se $a_n$ diverge, então $b_n$ também o faz.}

\begin{demonstration}
Sejam $\displaystyle{S_m = \sum_{n=0}^{m} a_n}$ e  $\displaystyle{S'_m = \sum_{n=0}^{m} b_n}$ sequências  crescentes tais que $S_m \leq S'_m$

Por hipótese, temos $\displaystyle{\lim_{n \rightarrow +\infty} S'_m = S' = \sum b_n}$. Sabendo que $S'_m = sup \ \left\{S'_m \ | \ m \in \mathds{N}\right\}$, temos então que $S_m \leq S'_m \leq S' \ \forall \ m$.

Logo, temos que $S'$ é cota superior do conjunto $\left\{S_m \ | m \in \mathds{N}\right\}$. Como $S_m$ é convergente, então $\displaystyle{\exists \lim_{n \rightarrow + \infty} S_m = S = \left\{S_m \ | m \in \mathds{N}\right\} \Rightarrow \sum a_n}$ converge. 
\end{demonstration}

\section{Exercícios da aula do dia 17/08}
\label{sec:s27}

\subsection*{Calcular $\displaystyle{\sum_{n=1}^{\infty} e^{\frac{1}{n}}} $}
$$ a_n = e^{\frac{1}{n}} \ \ \lim_{n \rightarrow +\infty} e^{\frac{1}{n}} = 1 \neq 0$$ Logo a série converge.

\subsection*{Determinar se a expressão abaixo converge ou não. $$\sum_{n=1}^{\infty} \frac{1}{n}$$}

Ainda que o limite de $a_n$ tenda pra zero quando n vai pra infinito, essa expressão não é convergente.

\begin{align*}
S_1 &= 1 \\
S_2 &= 1 + \frac{1}{2} = \frac{3}{2} \\
S_3 &= 1 + \frac{1}{2} + \frac{1}{3} = \frac{11}{6} \\
S_4 &= 1 + \frac{1}{2} + \frac{1}{3} + \frac{1}{4} > 1 + \frac{1}{2} + \frac{1}{4} + \frac{1}{4} = 2 \\
S_8 &= S_4 + \frac{1}{5} + \frac{1}{6} + \frac{1}{7} + \frac{1}{8} > 2 + \frac{1}{8} + \frac{1}{8} + \frac{1}{8} + \frac{1}{8} = 2 + \frac{1}{2} = \frac{5}{2} \\
S_{2^3} &> \frac{3 + 2}{2} \\
S_{2^n} &> \frac{n + 2}{2} \ \ \forall \ n \in \mathds{N} \\
\lim_{n \rightarrow +\infty} \frac{n+2}{2} = +\infty &\Rightarrow \lim_{n \rightarrow +\infty} S_{2^n} = +\infty \\
\end{align*}

Logo $S_n$ diverge.

\subsection*{Determine a soma:}
\begin{align*}
\frac{1}{1*2} + \frac{1}{2*3} + \frac{1}{3*4} + \hdots = \sum_{n = 1}^{\infty} \frac{1}{n*(n+1)} \\
S_m &= \frac{1}{1*2} + \frac{1}{2*3} + \frac{1}{3*4} + \hdots + \frac{1}{n*(n+1)}\\
 &= \left(1 - \frac{1}{2}\right) + \left(\frac{1}{2} - \frac{1}{3}\right) + \left(\frac{1}{3} - \frac{1}{4}\right) + \left(\frac{1}{n} - \frac{1}{n+1}\right) \\
 &= 1 - \frac{1}{n+1} \\
 \lim_{n \rightarrow +\infty} S_m = \lim_{n \rightarrow +\infty} \left(1 - \frac{1}{n+1} \right) &=  \sum_{n = 1}^{\infty} \frac{1}{n*(n+1)} = 1 \\
\end{align*}

Nesse caso, $S_n$ converge.

\section{Teste seu conhecimento: (pergunte para o professor ou monitor se tiver dúvida)}
\label{sec:s28}

Determinar quais somas são convergentes:
\begin{itemize}
\item [a] $\displaystyle{\sum_{n = 1}^{\infty} \left(\frac{-1}{n}\right)^n}$
\item [b] $\displaystyle{\sum_{n = 0}^{\infty} 10^{-n} }$
\item [c] $\displaystyle{\sum_{n = 0}^{\infty} \frac{3^n}{2^n}}$
\item [d] $\displaystyle{\sum_{n = 0}^{\infty} \frac{n+5}{2n+4}}$
\item [e] $\displaystyle{\sum_{n = 0}^{\infty} \sin (n\pi)}$
\item [e] $\displaystyle{\sum_{n = 0}^{\infty} \cos (n\pi)}$
\item [a] $\displaystyle{\sum_{n = 0}^{\infty}  (2^{-n} + 3^{-n})}$
\end{itemize}

\section{Critério de Comparação no Limite}
\label{sec:s29}
Sejam $\sum a_n$ e $\sum b_n$ séries de termos positivos.
\begin{itemize}
\item [a] Dizemos que $a_n$ é da ordem de $b_n$ \footnote{$a_n = O(b_n)$} se $\exists k \in \mathds{R}\ |\ a_n \leq k\ b_n$ para n suficientemente grande.
\item [b]  Se $a_n = O(b_n) \ \land \ b_n = O(a_n) \Longleftrightarrow a_n\approx b_n$.
\item [c] $\displaystyle{a_n\approx b_n \ \Rightarrow \lim_{n \rightarrow +\infty} \frac{a_n}{b_n} = 0}$ \footnote{Verificar esse detalhe}
\item [exemplo] 
\begin{align*}
a_n &= n    &    a_n &= O(b_n) \\
b_n &= 10n  &   10n \leq \underbrace{20}_K n &\Rightarrow b_n = O(a_n) \\
\end{align*}
\end{itemize}

\begin{lemma}
Sejam $\sum a_n$ e $\sum b_n$ duas séries de termos positivos. Temos:\begin{itemize}
\item se $a_n = O(b_n)$ e se $\sum b_n$ converge, então $\sum a_n$ converge também.
\item se $a_n = O(b_n)$ e se $\sum b_n$ diverge, então $\sum a_n$ diverge também.
\item se $\displaystyle{\lim_{n \rightarrow +\infty} \frac{a_n}{b_n} = c > 0}$ , então $\sum a_n$ converge e $\sum b_n$ também.
\end{itemize}
\end{lemma}

\hspace{5mm} \textbf{Mostre que} $\displaystyle{\sum_{n=1}^{\infty} \frac{3n + 4}{n^2 + 2}}$ diverge.
\begin{align*}
a_n &= \frac{3n + 4}{n^2 + 2} & b_n &= \frac{1}{n} \\
\lim_{n \rightarrow \infty} \frac{a_n}{b_n} &= \lim_{n \rightarrow \infty} \frac{3n^2 + 4n}{n^2 + 2} = \lim_{n \rightarrow \infty} \frac{3 + \frac{4}{n}}{1 + \frac{2}{n^2}} = 3 > 0 \\
\end{align*}

Como $\sum \frac{1}{n}$ diverge, então $\sum_{n = 1}^{\infty} a_n$ diverge também. \vspace{7mm}

\hspace{5mm} \textbf{Mostre que} $\displaystyle{\sum_{n=1}^{\infty} \underbrace{\frac{\cos^2(n\pi)}{n^2}\ \frac{\sqrt{4n-3}}{n^2+1}}_{a_n}}$ convege.

%Tomaremos $b_n = \frac{1}{n^2}$. Se mostrarmos que $a_n \approx b_n$, teremos que $a_n$ converge, pois $b_n$ também o faz. 
\begin{align}
\sqrt{4n-3} < 4n-3 \leq n^2 \label{eq:1} \\
a_n = \frac{1}{n^2} \ \frac{\sqrt{4n-3}}{n^2 + 1} \underbrace{<}_{\ref{eq:1}} \frac{n^2}{n^2(n^2+1)} < \frac{1}{n^2} \\
\end{align}
$\sum a_n$ converge.

\section{Séries de termos Positivos: Critérios de Comparação (Parte II)}
\label{sec:s210}

\hspace{5mm}Os critérios da parte anterior dependem do conhecimento de séries com determinadas propriedades. Nesta seção, veremos métodos "intrínsecos", ou seja, de métodos que dependem somente das próprias séries.

\subsection{Critério da Razão (D'Alambert)}
Seja $\sum a_n$ Uma série de termos positivos e suponha que exista $\displaystyle{\lim_{n \rightarrow \infty} \frac{a_n}{b_n} = l}$ finito ou não. Desse modo temos que: $\left\{ \begin{array}{rlll}
diverge & \hbox{se} &  l > 1 \\
converge & \hbox{se} & 0 \leq l < 1 \\
inconclusivo & \hbox{se} & l = 1 \\
\end{array}\right.$

\begin{demonstration}
Temos dois casos:
\begin{itemize}
\item [a.] $$\lim_{n \rightarrow \infty} \frac{a_{n+1}}{a_n} = l < 1 $$ Existe $r \in \mathds{R}$ tal que $l < r < 1$. Existe $n \in \mathds{N}$  tal que $$ n \geq N \Rightarrow \frac{a_{n+1}}{a_n} < r$$
Temos : \begin{align}
\frac{a_{N+1}}{a_N} < r &\Rightarrow  a_{N+1} < r\ a_N \\
\frac{a_{N+2}}{a_{N+1}} < r &\Rightarrow  a_{N+2} < r\ a_{N+1} < r^2\ a_N \\
\vdots &\Rightarrow \vdots \\
\vdots &\Rightarrow a_{N+K+1} < a_N\  r^{K+1} \label{eq:210}
\end{align}
Como $0 < r < 1$, a série $\displaystyle{\sum_{n=0}^{\infty} r^n}$ converge $\displaystyle{\Rightarrow \ \sum_{n=N}^{\infty} r^n}$ converge.
Pela relação \ref{eq:210}, temos que $\displaystyle{\sum_{n=N}^{\infty} a_n}$ converge,então $\displaystyle{\sum_{n=0}^{\infty} a_n}$ converge também.
\item [b.] $$\lim_{n \rightarrow \infty} \frac{a_{n+1}}{a_n} = l > 1 $$ Existe $n \in \mathds{N}$  tal que $$ n \geq N \Rightarrow \frac{a_{n+1}}{a_n} > 1 \Rightarrow a_{n+1} > a_n $$

\begin{align*}
a_{N+1} &> a_N \\
a_{N+2} &> a_{N+1} > a_N\\
\vdots\\
a_{N+K} &> a_N & \forall \ k \in \mathds{N}\\
\lim_{n \rightarrow \infty} a_n \geq a_N > 0
\end{align*}
Logo $\sum a_n$ converge.
\end{itemize}
\end{demonstration}

\hspace{5mm}\textbf{Exemplo} Analisar a série abaixo: $$\sum_{n=1}^{\infty} \frac{2^n \ n!}{n^n}$$ \footnote{$\displaystyle{\lim_{n \rightarrow \infty} \left( 1 + \frac{1}{n} \right)^n = e}$}
\begin{align*}
a_n &= \frac{2^n \ n!}{n^n} \\
\frac{a_{n+1}}{a_n} &=  \frac{2^{n+1} \ (n+1)!}{(n+1)^{n+1}} \ \frac{n^n}{2^n \ n!} \\
&= 2\ \left( \frac{n}{n+1}\right)^n = \frac{2}{\left( \frac{n+1}{n}\right)^n}\\
\lim_{n \rightarrow \infty} \frac{a_{n+1}}{a_n} &= \frac{2}{e} < 1 
\end{align*}
Pelo critério da Razão, a série converge.

\subsection{Critério da Raíz(De Cauchy)}

\hspace{5mm} Seja $a_n$ uma série de termos positivos tal que $\displaystyle{\lim_{n \rightarrow \infty} \sqrt[n]{a_n} = l}\  $ tal que $\ \sum a_n  \left\{ \begin{array}{rlll}
diverge & \hbox{se} &  l > 1 \\
converge & \hbox{se} & 0 \leq l < 1 \\
inconclusivo & \hbox{se} & l = 1 \\
\end{array}\right.$

Seja $\int a_n$ uma série de números positivos, então se :
\begin{align}
\lim_{n \rightarrow \infty} \frac{a_{n+1}}{a_n} &= l \Rightarrow \label{eq:cauchy1} \\
%esse rótulo foi nomeado com base na sessão, pra facilitar sua referência a algumas linhas a frente xD
\lim_{n \rightarrow \infty} \sqrt[n]{a_n} &= l
\end{align}

Seja $(a_n)$ uma sequência tal que $\displaystyle{\lim_{n \rightarrow \infty} a_n = a}$ então,  $$\lim_{n \rightarrow \infty} \frac{a_1 + a_2 + \cdots + a_n}{n} = a \ \land \ \lim_{n \rightarrow \infty} \sqrt[n]{a_1 * a_2 * \hdots * a_n} = a$$
\begin{align*}
\lim_{n \rightarrow \infty} \sqrt[n]{a_n} \\
\lim_{n \rightarrow \infty} \sqrt[n]{a_n} &= \sqrt[n]{a_1} \ \sqrt[n]{\underbrace{\frac{a_2}{a_1}}_l\ \underbrace{\frac{a_3}{a_2}}_l \ \underbrace{\frac{a_4}{a_3}}_l \ \cdots \ \underbrace{\frac{a_n}{a_{n-1}}}_{l \ (\ref{eq:cauchy1})}}
\end{align*}

\textbf{Decida} se a série abaixo é convergente. $$\sum_{n = 1}^{\infty} \left( \frac{n}{2n+1}\right)^n$$
\begin{align*}
a_n &= \left( \frac{n}{2n+1}\right)^n \\
\lim_{n \rightarrow \infty} \sqrt[n]{a_n} &= \lim_{n \rightarrow \infty}\frac{n}{2n+1} \\
&= \lim_{n \rightarrow \infty}\frac{1}{2+\frac{1}{n}} = \frac{1}{2}
\end{align*}

Logo, a série converge.

\subsection{Critério da Integral}

\hspace{5mm} Seja $\displaystyle{f: [0;+\infty) \rightarrow \mathds{R}}$ uma função positiva e decrescente, e seja $a_n = f(x)$. Desse modo, a série $\displaystyle{\sum_{n=0}^{\infty} a_n}$ converge, se e somente se, $\displaystyle{\int_0^{\infty} f(x) \ dx}$ também for convergente. 

Vale saber que $$\sum_{n=0}^{\infty} a_n = \int_0^{\infty} f(x) \ dx = \lim_{t \rightarrow +\infty} \left( \int_0^t f(x) \ dx \right)$$

\begin{lemma}
Seja $\sum a_n$ uma série de termos positivos e suponha que haja uma função positiva e decrescente $f: [N, +\infty) \rightarrow \mathds{R}$ tal que $\forall n \geq N, \ f(n) = a_n$. Logo $\sum a_n$ convege se, e somente se a integral $\displaystyle{\int_N^{\infty}  f(x) \ dx }$ convergir também.
\end{lemma}

\section{Exercícios da aula do dia 24/08}
\label{sec:211}

\subsection*{Mostrar que a série $$\sum_{n = 1}^{\infty} \frac{1}{n^s}$$ converge para $s > 1$ e converge para $s \leq 1$}

\begin{itemize}
\item se $\displaystyle{s = 0} \Rightarrow \sum_{n = 1}^{\infty} \frac{1}{n^0} \Longleftrightarrow \sum_{n = 1}^{\infty} 1$ \\ A série diverge.
\item se $\displaystyle{s = 1} \Rightarrow \sum_{n = 1}^{\infty} \frac{1}{n}$ \\ A série também diverge.
\item se $\displaystyle{0 < s < 1 \Rightarrow n^s < n \Longleftrightarrow \frac{1}{n^s} > \frac{1}{n} \ \ \ \vspace{5mm} \sum_{n = 1}^{\infty} \frac{1}{n^s}}$ diverge pelo critério de comparação.
\item se $\displaystyle{s < 0} \Rightarrow \sum_{n = 1}^{\infty} n^{-s} = \infty$ A série diverge \footnote{se $s < 0 \Longleftrightarrow -s > 0$}
\item se $s > 1$, considere $f: [1, \infty) \rightarrow \mathds{R} \ \ \ \ \ f(x) = \frac{1}{x^s}$

\begin{align*}
\int_1^{\infty} \frac{1}{x^s} \ dx = \lim_{t \rightarrow \infty} \left(\int_1^t x^{-s} \ dx \right) &= \lim_{t \rightarrow \infty} \left[\frac{x^{1-s}}{1-s} \right]_1^t \\
\lim_{t \rightarrow \infty} \frac{t^{1-s}-1}{1-s} &= \frac{-1}{1-s} = \frac{1}{s-1}
\end{align*}
Logo, para $s > 1$, a série converge.
\end{itemize}

\subsection*{Avaliar as séries abaixo}
\begin{itemize}
\item [a.] $\displaystyle{\sum_{n=2}^{\infty} \frac{1}{(\ln n)^{\ln n}}}$ 
\item [b.] $\displaystyle{\sum_{n=2}^{\infty} \frac{1}{(\ln n)^{\ln(\ln n)}}}$
\end{itemize}
\textbf{resolução item A\footnote{O item B se resolve de forma semelhamnte e eu deixo para que o leitor o faça}}
\\
\textbf{Passo 1:} Tome a função $f:[a, +\infty[ \rightarrow \mathds{R} \ | \  f(x) = \displaystyle{\frac{1}{(\ln x)^{\ln x}}}$, podemos usar, primeiramente, o critério da integral onde teremos algo assim: $$\int_2^{\infty} f(x) \ dx = \lim_{t \rightarrow \infty} \int_2^t \displaystyle{\frac{dx}{(\ln x)^{\ln x}}}$$
\textbf{Passo 2:} Mudança de variável
\begin{align}
y &= \ln x \\ \label{211a}
x &= e^y \\ 
dx &= e^y \ dy \label{211b}
\end{align}
Teremos a seguinte integral $$\lim_{t \rightarrow \infty} \int_2^t \displaystyle{\frac{\overbrace{e^y \ dy}^{\ref{211a}}}{\underbrace{y^y}_{\ref{211b}}}} = \int_2^{\infty} \frac{e^y}{y^y} dy$$

\textbf{Passo 3:} Tomamos uma sequência $b_n = \displaystyle{\frac{e^n}{n^n}}$ e com ela podemos usar o critério da raíz. Assim, se $\displaystyle{\exists \lim_{n \rightarrow \infty} \sqrt[n]{b_n}}$ e esse limite estiver entre 0 e 1 ($0 \leq k < 1$), a $b_n$ e consequentemente, $a_n$ serão convergentes, caso contrário, serão divergentes. $$\lim_{n \rightarrow \infty} \sqrt[n]{\frac{e^n}{n^n}} = \lim_{n \rightarrow \infty} \ \frac{e}{n} = 0$$ Logo, as séries são convergentes. \\
\vspace{7mm}

\section{Convergência Absoluta --- Convergência Condicional}

\hspace{5mm} Lembrando que uma sequência $b_n$ é convergente, se e somente se, para um dado $\epsilon$ suficientemente pequeno, houver $N \in \mathds{N}$ tal que \begin{align*}
m,\ n \ \geq N \Rightarrow |b_m - b_n| &\leq \epsilon \\
m \ \geq N \Rightarrow |b_{m+k} - b_m| &\leq \epsilon & \forall \ k = 1,\ 2, 3 \ \hdots
\end{align*}

\begin{theorem}
Seja $\displaystyle{\sum_{n=0}^{\infty} a_n}$ uma série numérica. Se a série $\displaystyle{\sum_{n=0}^{\infty} |a_n|}$ convergir, a primeira série certamente será convergente também.
\end{theorem}

\begin{demonstration}
Sejam \begin{align*}
S_m &= a_0 + a_1 + a_2 + \hdots + a_m \\
S'_m &= |a_0| + |a_1| + |a_2| + \hdots + |a_m|
\end{align*}
Vamos tomar que $S'_m$ seja convergente, queremos provar que $S_m$ converge também.
Dado $\epsilon > 0$ e $S'_m$ convergente, então existe $N \in \mathds{N}$ tal que \begin{align}
m \ \geq N &\Rightarrow |S'_{m+k} - S'_m| \leq \epsilon \\
&= |a_{m+1}| + |a_{m+2}| + \ \hdots \ + |a_{m+k}| < \epsilon \label{212a}
\end{align}

Então, $m \geq N \Rightarrow |S_{m+k} - S_m| = |a_{m+1} +\hdots + a_{m+k}| < \\ < \underbrace{|a_{m+1}| + \hdots + |a_{m+k}| < \epsilon}_{\ref{212a}}$

Pelo critério de Couchy, a sequência $S_m$ é convergente.
\end{demonstration}

\textbf{Exemplo:} $$(*) a_n = \sum_{n=0}^{\infty} \ \frac{\cos n\pi}{n^2} = \sum_{n=0}^{\infty} \ \frac{(-1)^n}{n^2}$$ Em valor absoluto temos $\displaystyle{\sum_{n=1}^{\infty} \ \frac{1}{n^2}}$ que é convergente, logo $\displaystyle{\sum_{n=1}^{\infty} \ \frac{\cos(n\pi)}{n^2}}$ também o é.

\textbf{Considere} o seguinte exemplo: $$(**) \sum_{n=1}^{\infty} \ \frac{(-1)^n}{n}$$ Em valores absolutos temos $\displaystyle{\sum_{n=1}^{\infty} \ \frac{1}{n}}$ que é divergente, mas a série $(**)$ converge como veremos a seguir. 

\begin{definition} São tipos de convergência:
\begin{itemize}
\item Uma série $\sum a_n$ é dita \underline{absolutamente convergente} se $\sum a_n$ e $\sum |a_n|$ forem convergentes. Exemplo $(*)$
\item Uma série $\sum a_n$ é dita \underline{condicionalmente convergente} se $\sum a_n$ convergir, mas $\sum |a_n|$. divergir. Exemplo $(**)$
\end{itemize}
\end{definition}

Para dada $\displaystyle{\sum_{n=0}^{\infty} a_n, \ a_n \neq 0}$ e suponha que exista $$\lim_{n \rightarrow \infty} \ \frac{|a_{n+1}|}{|a_n|} = l$$

\begin{itemize}
\item [a] se $0 \leq l <1$, então $\displaystyle{\sum_{n=0}^{\infty} |a_n|}$ converge, e portanto, $\displaystyle{\sum_{n=0}^{\infty} a_n}$ converge também.
\item [b] se $l > 1$ então $\displaystyle{\sum_{n=0}^{\infty} |a_n|}$ diverge, e portanto, $\displaystyle{\sum_{n=0}^{\infty} a_n}$ também o é. Isso acontece pois se $$ \lim_{n \rightarrow \infty} |a_n| \neq 0 \Rightarrow \lim_{n \rightarrow \infty} a_n \neq 0 \Rightarrow \sum a_n \ \ diverge$$
\end{itemize}

\subsection{Critério de Liebniz}
\label{subsec:2121}

\hspace{5mm}Tomamos uma série $\displaystyle{\sum_{n=0}^{\infty} (-1)^n a_n}$ onde $a_n$ é uma sequência de termos positivos tais que: \begin{enumerate}
\item $a_n$ é decrescente ($a_n \geq a_{n+1}$)
\item $\displaystyle{\lim_{n \rightarrow \infty} a_n = 0}$
\end{enumerate}

Então $\displaystyle{\sum_{n=0}^{\infty} (-1)^n a_n}$ converge. Além disso, temos que \begin{align}
S_m &= \sum_{n=0}^m (-1)^n \ a_n \\
S &= \sum_{n=0}^{\infty} (-1)^n \ a_n \\
\lim_{n \rightarrow \infty} S_m &= S \\
|S-S_m| &\leq a_{m+1} & \forall m \geq 1
\end{align}
\textbf{Voltemos ao exemplo (**)} $ \ \displaystyle{a_n = \frac{1}{n}}$ e tomando dois inteiros tais que $i \leq j$ \begin{itemize}
\item $a_n$ é decrescente. $$ i \leq j \Longleftrightarrow \frac{1}{i} \geq \frac{1}{j} \Longleftrightarrow a_i \geq a_j $$ 
\item $a_n$ é uma série de números positivos
\item $\displaystyle{\lim_{n \rightarrow \infty} a_n = a}$
\item $\displaystyle{|S-S_{99}| = a_{99+1} = a_{100} = 0,01}$ \footnote{Eu não tenho certeza, mas acredito que essa propriedade é uma consequência, e não uma dependência da série ser convergente ou não}
\end{itemize}
$$\sum_{n=1}^\infty \ \frac{(-1)^n}{n} \ converge$$ 

\textbf{Fatos importantes sobre essa série:}\footnote{$n \in \mathds{N} \land n \geq 1$}
\begin{itemize}
\item $S_2 > S_4 > S_6 > \hdots > S_{2n}$
\item $S_1 < S_3 < S_5 < \hdots < S_{2n-1}$
\item Para $k$ par e $l$ ímpar, temos tempre que $S_k > S_l$
\end{itemize}

\begin{align}
S_{2n+2} &= -a_1 + a_2 - \hdots + a_{2n} - a_{2n-1} + a_{2n+2}\\
S_{2n+1} &= -a_1 + a_2 - \hdots - a_{2n-1} + a_{2n} - a_{2n+1}\\
S_{2n+2} - S_{2n} &= -a_{2n+1} + a_{2n+2} \leq 0 \Longleftrightarrow S_{2n+2} \leq S_{2n} \label{212A}\\
S_{2n+1} - S_{2n-1} &= a_{2n} - a_{2n+1} \geq 0 \Longleftrightarrow S_{2n+1} \geq S_{2n-1} \label{212B}\\
S_{2n} - S_{2n-1} &= a_{2n} > 0 \label{212C} \\
l \leq 2n-1 &\land k < 2n
\end{align}

Podemos concluir que $S_l \underbrace{<}_{\ref{212B}} S_{2n-1} \overbrace{<}^{\ref{212C}} S_{2n} \underbrace{<}_{\ref{212A}} S_k$

\begin{obs}{Observações}
\begin{itemize}
\item $S_{2n}$ é decrescente e é limitada inferiormente ($S_{2n} \geq S_1 \ \forall n$). \\
Logo $\exists \displaystyle{\lim_{n \rightarrow \infty} S_{2n} = \alpha = inf \ \left\{S_{2n} | \ n \in \mathds{N}\right\} }$
\item $S_{2n+1}$ é crescente e é limitada superiormente ($S_{2n+1} \leq S_2 \ \forall n$). \\
Logo $\exists \displaystyle{\lim_{n \rightarrow \infty} S_{2n+1} = \beta = sup \ \left\{S_{2n+1} | \ n \in \mathds{N}\right\} }$
\end{itemize}

Desejamos mostrar que $\alpha = \beta$. Para isso, temos que $S_{2n+2} - S_{2n+1} = a_{2n+2} > 0$. 
$$\lim_{n \rightarrow \infty} \overbrace{\left(S_{2n+2} - S_{2n+1}\right)}^{\alpha - \beta} = \lim_{n \rightarrow \infty} a_{2n+2} = 0$$
Logo $\alpha - \beta = 0 \Longleftrightarrow \alpha = \beta$
\end{obs}
\newpage

\section*{Teste seus conhecimentos -- Avalie se as séries abaixo são divergentes, convergentes absolutas, ou convergentes condicionais} Pergunte ao monitor, se tiver qualquer dúvida.
%Ideia de subtítulo #1 (caso vc queira se aprofundar em LaTeX )\section{Section Title\\Section Subtitle}
%Ideia de subtítulo #2 \section[Section Title. Section Subtitle]{Section Title\\ {\large Section Subtitle}}
\begin{itemize}
\item [a.] $\displaystyle{\sum_{n=1}^{\infty} \ \frac{\sin(n\theta)}{n^2}, \ \theta \in \mathds{R}}$
\item [b.] $\displaystyle{1-\frac{1}{3}+\frac{1}{5}-\frac{1}{7}+ \frac{1}{9} + \hdots = \sum_{n=0}^{\infty} \ \frac{(-1)^n}{2n+1}}$
\item [c.] $\displaystyle{1-\frac{1}{2}+\frac{2}{3}-\frac{1}{3}+\frac{2}{4}-\frac{1}{4}+\frac{2}{5}-\frac{1}{5}+\frac{2}{6}}-\frac{1}{6}+ \hdots$
\item [d.] $\displaystyle{\sum_{n=1}^{\infty} (-1)^n \ \frac{\ln n}{n}}$
\item [e.] $\displaystyle{\sum_{n=1}^{\infty} \ \frac{(-1)^n}{\ln n}}$
\item [f.] $\displaystyle{\sum_{n=1}^{\infty} \ \frac{(-1)^{n+1}}{\sqrt{n}}}$
\item [g.] $\displaystyle{1+\frac{1}{2}-\frac{1}{4}-\frac{1}{8}+\frac{1}{16}+\frac{1}{32}-\frac{1}{64}-\frac{1}{128}+\hdots}$
\end{itemize}

\chapter{Sequências de Funções} Assunto tratado entre 31/08/2015 até 00/00/2015
\label{chap:c3}

\begin{definition} Uma sequência de funções é uma sequência $n \rightarrow f_n$ onde cada $f_n$ é uma função definida em um mesmo conjunto $A \subset \mathds{R}$ \textbf{Exemplos:} \begin{enumerate}
\item $f_n: \mathds{R} \rightarrow \mathds{R} \ | \ f_n(x) = x^n, \ n \geq 1$ \label{ex:302a}
\item $f_n: \mathds{R} \rightarrow \mathds{R} \ | \ f_n(x) = n\ \displaystyle{\sin \left( \frac{x}{n}\right)}$ \label{ex:302b}
\end{enumerate}
\end{definition}

Seja $f_n$ uma sequência definida em $A \in \mathds{R}$. Seja B o subconjunto de A, tal que a seuqência numérica ($f_n(x)$) seja convergente. $B = \left\{x \in A\ | \ \exists \lim_{n \rightarrow \infty} f_n(x) \right\}$

Podemos definir: $f: B \rightarrow \mathds{R} \ | \ f(x) = \displaystyle{\lim_{n \rightarrow \infty} f_n(x)}$ Dado $\epsilon > 0, \ \exists n_0 \in \mathds{N}$ tal que $n \geq n_0 \Rightarrow |f_n(x) - f(x)| < \epsilon$

Voltemos ao exemplo \ref{ex:302a} onde: $f_n(x) = x^n, \ A = \mathds{R}$ e queremos definir o conjunto B.

\begin{itemize}
\item Se $|x| < 1$, então $f_n$ tende a 0.
\item Se $x=1$, $f_n$ tende a 1.
\item Se $x=1 \ \lor \ |x| > 1$: $f_n$ diverge.
\end{itemize}
Logo $B = ]-1;1]\ \land \hbox{f}(x)
= \left\{ \begin{array}{rll}
0 & \hbox{se} &  |x| < 1 \\
1 & \hbox{se} &  x  = 1 \\
\end{array}\right.$

Agora, nos voltamos ao exemplo \ref{ex:302b}
\begin{itemize}
\item se $x=0$, $f_n$ tende a 0
\item caso contrário, temos $$f(x) = \lim_{n \rightarrow \infty} n \sin \left(\frac{x}{n}\right) = \LI x \ \underbrace{\frac{\sin(\frac{x}{n})}{\frac{x}{n}}}_1 = x$$
\end{itemize}

\section*{Mais exemplos} Determine o domínio da função $f$ dada por $\displaystyle{f(x) = \LI f_n(x)}$

\begin{itemize}
\item [a] $$f_n= e^{nx}$$ 
 Não existe $\displaystyle{ \LI f_n(x) = k \ | \ k \neq \infty}$ Logo não existe $f(x)$ nessas condições \ldots
 \item [b] $$f_n= \frac{nx}{1+nx^2}$$
 se $x=0 \Rightarrow f_n(0) = 0 \rightarrow f(0) = 0$, para $x \neq 0$ teremos $\displaystyle{\LI \ \frac{nx}{1+nx^2} = \LI \ \frac{x}{\frac{1}{n}+x^2} = \frac{x}{x^2} \rightarrow f(x) = \frac{1}{x}}$. O seu domínio é $D = \mathds{R} \left\{0\right\}$
 \item [c] $$f_n(x) = \ \left( 1 + \frac{1}{n} \right)^{nx} \rightarrow \LI \ \left[ \left( 1 + \frac{1}{n} \right)^n \right]^x \rightarrow f(x) = e^x$$
\end{itemize}

\begin{definition}[Convergência Uniforme]
Dadas ($f_n$) e $f$ definidas em $A \subset \mathds{R}$, dizemos que $f_n$ \underline{converge unformemente} para $f$ em $A$ se para dado $\epsilon > 0, \exists \ n_0 \in \mathds{N} \ | \ n \geq n_0 \Rightarrow |f_n(x) - f(x)| < \epsilon$
\end{definition}

Voltamos para o exemplo 1 \footnote{p. \pageref{ex:302a}} com $x \in [-\frac{1}{2}; \ \frac{1}{2}]$. Sabemos que essa série converge a 0 neste intervalo. Neste caso, verificaremos que a convergência é uniforme.

Dado $\epsilon > 0, \ \exists \ n_0 \ | \ n \geq n_0 \Rightarrow |x^n-0| < \epsilon \ \forall \ x \in [-\frac{1}{2};\frac{1}{2}]$.

Tomamos que $\displaystyle{|x| < \frac{1}{2} \Rightarrow |x|^n < \frac{1}{2^n} <\epsilon}$. Tomando $\displaystyle{\frac{1}{2^{n_0}} < \epsilon}$ Temos  que para: $$n \geq n_0 \Rightarrow |x|^n < \frac{1}{2^n} < \frac{1}{2^{n_0}} < \epsilon$$

\section*{Exercício:} Determine para que valores de $x \in \mathds{R}$ as funções a seguir estão definidas:
\begin{itemize}
\item [a.] $\displaystyle{\soma{1} \ \frac{n! x^n}{3^n}}$

Tomamos $a_n = \soma{1} \frac{n! x^n}{3^n}$ 
\begin{align*}
\LI \frac{a_{n+1}}{a_n} &= \LI \frac{(n+1)! x^{n+1}}{3^{n+1}}\ \frac{3^n}{n! x^n} \\
&= \LI \frac{x}{3} (n+1) = \frac{x}{3} \LI (n+1) = \infty
\end{align*}
Logo não há nuémros reais para os quais essa função está definida.
\item [b.] $\displaystyle{\soma{1} \ ne^{-xn}}$
\begin{align*}
a_n &= n \ e^{-xn} \\
&= \frac{n}{e^{xn}} \\
\LI \sqrt[n]{\frac{n}{e^{xn}}} &= \LI \frac{\overbrace{\sqrt[n]{n}}^{=1 \footnote{\ref{subsec:ex223}}}}{\sqrt[n]{e^{xn}}} \\
f(x) = \frac{1}{e^x}
\end{align*}
Logo essa função está definida para qualquer valor estritamente positivo de $x$
\end{itemize}


\end{document}
